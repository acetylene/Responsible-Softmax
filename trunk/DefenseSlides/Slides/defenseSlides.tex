%\documentclass[handout,draft]{beamer}
\documentclass{beamer}
\usetheme{coatneyDis}
\usepackage{amsmath, amsthm, amssymb, amsfonts, mathtools, mathrsfs, bm, bbm}

\usepackage[square, sort, comma, numbers]{natbib}
\usepackage{ifthen}
\usepackage{relsize}
\usepackage{graphicx}  % Required for including images
\usepackage{booktabs} % Top and bottom rules for tables
%\usepackage{caption}
\usepackage{subcaption}
\usepackage{textcomp}
\usepackage{physics}
\usepackage{array}
\usepackage{tikz}
\usetikzlibrary{automata, positioning}
\tikzset{every state/.style={minimum size=0pt}}

\usepackage[toc]{appendix}
\usepackage{fancyvrb}
\usepackage{listings}

%Settings for the listings package. Setup is currently for MATLAB defaults
\lstset{% general command to set parameter(s)
	basicstyle=\tiny, % print whole listing small
	keywordstyle=\color{blue}\bfseries,
	% underlined bold black keywords
	identifierstyle=, % nothing happens
	commentstyle=\color{green}, % white comments
	stringstyle=\color{purple}\ttfamily, % typewriter type for strings
	showstringspaces=false}

\usepackage{placeins}
\usepackage{float}
\usepackage{hyperref}

%\usepackage[lofdepth,lotdepth]{subfig}
% Packages to have pseudocode must be included after hyperref!!!

\usepackage{setspace}
\usepackage{algorithm}
\usepackage{algpseudocode}
\usepackage{chngcntr}


\DeclareMathAlphabet{\mathpzc}{OT1}{pzc}{m}{it}

\def \ga{\alpha} \def \gb{\beta}  \def \gd{\delta} \def \gw{\omega} \def \gW{\Omega}
\def \gt{\theta} \def \gp{\phi} \def \ge{\epsilon} \def \gs{\sigma}
\def \gl{\lambda} \def \gz{\zeta} \def \gr{\rho} \def \GT{\Theta}

%\def \gg{\gamma}

\def \BF{\mathbb{F}}
\def \<{\langle} \def \>{\rangle}
\newcommand{\overbar}[1]{\mkern 1.5mu\overline{\mkern-1.5mu#1\mkern-1.5mu}\mkern 1.5mu}

\newcommand{\oo}{\infty}

\newcommand{\fr}[1]{\mathfrak{#1}}
\renewcommand{\op}[1]{\operatorname{#1}}
\newcommand{\Unit}[1]{{#1}^{\times}}
\newcommand{\cc}[1]{\overline{#1}}
\newcommand{\KX}[1]{\ifthenelse{\equal{#1}{1}}{$K[x]$}{$K[x_1,x_2,\ldots,x_{#1}]$}}

\newcommand{\Ryan}[1]{\ifdraft\textcolor{red}{Ryan says: #1}\fi}
\newcommand{\Marek}[1]{\ifdraft\textcolor{blue}{Marek says: #1}\fi}
\newcommand{\Dave}[1]{\ifdraft\textcolor{plum}{Dave says: #1}\fi}
\newcommand{\Clay}[1]{\ifdraft\textcolor{green}{Clay says: #1}\fi}
\newcommand{\Rob}[1]{\ifdraft\textcolor{orange}{Robert says: #1}\fi}

\newcommand{\R}{\mathbb R}
\newcommand{\HHH}{\mathbb H}
\newcommand{\RR}{\mathcal R}
\newcommand{\SSS}{\mathcal S}
\newcommand{\SSSS}{\mathfrak S}
\newcommand{\CC}{\textrm{C}}
\newcommand{\charr}{\textrm{char}}
\newcommand{\Supp}{\textrm{Supp}}
\newcommand{\CM}{\mathcal M}
\newcommand{\HH}{\textrm{H}}
\newcommand{\htt}{\textrm{ht}}
\newcommand{\Imm}{\textrm{Im}}
\newcommand{\ds}{\displaystyle}

\newcommand{\agmax}{\textrm{argmax}}
\DeclareRobustCommand{\pder1}[2]{\frac{\partial {#1}}{\partial {#2}}}

\newcommand{\N}{\mathbb N}
\newcommand{\ac}{\mathfrak{a}}
\newcommand{\mc}{\mathfrak{m}}
\newcommand{\Pc}{\mathfrak{P}}
\newcommand{\pc}{\mathfrak{p}}

%\newcommand{\MM}{\textrm{M}}
\newcommand{\PG}{\textrm{P}\Gamma_1}
\newcommand{\BA}{\mathbb{A}}
\newcommand{\QQ}{\mathbb Q}
\newcommand{\ZZ}{\mathbb Z}
\newcommand{\KK}{\mathbb K}
\newcommand{\BC}{\mathbb C}
\newcommand{\Prj}{\mathbb P}
\newcommand{\ri}{\mathcal{O}}
\newcommand{\FS}{\mathfrak F}
\newcommand{\Norm}{\textrm{N}}
\newcommand{\End}{\textrm{End}}
\newcommand{\Cl}{\textrm{Cl}}
\newcommand{\Qbar}{\overline{\Q}}

\DeclareMathOperator{\Gal}{Gal}
\DeclareMathOperator{\Frob}{Frob}
\DeclareMathOperator{\GL}{GL}
\DeclareMathOperator{\SL}{SL}
\DeclareMathOperator{\PGL}{PGL}
\DeclareMathOperator{\Aut}{Aut}
\DeclareMathOperator{\Hom}{Hom}
\DeclareMathOperator{\Stab}{Stab}
\DeclareMathOperator{\Fix}{Fix}
\DeclareMathOperator{\Inn}{Inn}
\DeclareMathOperator{\Bil}{Bil}
\DeclareMathOperator{\disc}{Disc}


\DeclareFontFamily{U}{wncy}{}
\DeclareFontShape{U}{wncy}{m}{n}{<->wncyr10}{}
\DeclareSymbolFont{mcy}{U}{wncy}{m}{n}
\DeclareMathSymbol{\Sh}{\mathord}{mcy}{"58} 

\theoremstyle{definition}
\newtheorem{defn}{Definition}[chapter]%[section]
\newtheorem{thm}{Theorem}[chapter]

\theoremstyle{remark}
\newtheorem{rk}[defn]{Remark}
\newtheorem{ex}{Exercise}
\newtheorem{eg}[defn]{Example}
\newtheorem*{soln}{Solution}
\newtheorem{experiment}[defn]{Experiment}
\newtheorem{calc}[defn]{Calculation}

\theoremstyle{plain}
\newtheorem{lemm}[thm]{Lemma}
\newtheorem{prop}[thm]{Proposition}
\newtheorem{cor}[thm]{Corollary}

%\numberwithin{defn}{section}

\newcommand{\Matrix}[1]{\begin{bmatrix} #1 \end{bmatrix}}
 \newcommand{\Vector}[1]{\begin{pmatrix} #1 \end{pmatrix}}

% \newcommand*{\norm}[1]{\mathopen\| #1 \mathclose\|}% use instead of $\|x\|$
% \newcommand*{\abs}[1]{\mathopen| #1 \mathclose|}% use instead of $\|x\|$
 \newcommand*{\normLR}[1]{\left\| #1 \right\|}% use instead of $\|x\|$

 \newcommand*{\SET}[1]  {\ensuremath{\mathcal{#1}}}
 \newcommand*{\FUN}[1]  {\ensuremath{\mathcal{#1}}}
 \newcommand*{\MAT}[1]  {\ensuremath{\boldsymbol{#1}}}
 \newcommand*{\VEC}[1]  {\ensuremath{\boldsymbol{#1}}}
 \newcommand*{\CONST}[1]{\ensuremath{\mathit{#1}}}
 
% \newcommand{\bra}{\langle}
% \newcommand{\ket}{\rangle}
 %%%%%%%%%%%%%%%%%%%%%%%%%%%%%%%%%%%%%%%%%%%%%%%%%%%%%%%
 % commands for quick dissertation writing
 %%%%%%%%%%%%%%%%%%%%%%%%%%%%%%%%%%%%%%%%%%%%%%%%%%%%%%%
 \newcommand*{\Rpi}[2][\bm\pi]{ \(R_{#2}({#1})\)} 
 \newcommand*{\elpi}[2][\bm\pi]{\ell_{#2}({#1})}
 \newcommand*{\RS}{responsible softmax }
 \newcommand*{\DR}{dynamic responsibility }
 %%%%%%%%%%%%%%%%%%%%%%%%%%%%%%%%%%%%%%%%%%%%%%%%%%%%%%%

 \DeclareMathOperator*{\argmax}{arg\,max}
 \DeclareMathOperator*{\diag}{diag}
 \DeclareMathOperator*{\argmin}{arg\,min}
 \DeclareMathOperator*{\vectorize}{vec}
 \DeclareMathOperator*{\reshape}{reshape}

 %-----------------------------------------------------------------------------
 % Differentiation
 \newcommand*{\Nabla}[1]{\nabla_{\!#1}}

 \renewcommand*{\d}{\mathrm{d}}
% \newcommand*{\dd}{\partial}

 \newcommand*{\At}[2]{\ensuremath{\left.#1\right|_{#2}}}
 \newcommand*{\AtZero}[1]{\At{#1}{\pp=\VEC 0}}

 \newcommand*{\diffp}[2]{\ensuremath{\frac{\dd #1}{\dd #2}}}
 \newcommand*{\diffpp}[3]{\ensuremath{\frac{\dd^2 #1}{\dd #2 \dd #3}}}
 \newcommand*{\diffppp}[4]{\ensuremath{\frac{\dd^3 #1}{\dd #2 \dd #3 \dd #4}}}
 \newcommand*{\difff}[2]{\ensuremath{\frac{\d #1}{\d #2}}}
 \newcommand*{\diffff}[3]{\ensuremath{\frac{\d^2 #1}{\d #2 \d #3}}}
 \newcommand*{\difffp}[3]{\ensuremath{\frac{\dd\d #1}{\d #2 \dd #3}}}
 \newcommand*{\difffpp}[4]{\ensuremath{\frac{\dd^2\d #1}{\d #2 \dd #3 \dd #4}}}

 \newcommand*{\diffpAtZero}[2]{\ensuremath{\AtZero{\diffp{#1}{#2}}}}
 \newcommand*{\diffppAtZero}[3]{\ensuremath{\AtZero{\diffpp{#1}{#2}{#3}}}}
 \newcommand*{\difffAt}[3]{\ensuremath{\At{\difff{#1}{#2}}{#3}}}
 \newcommand*{\difffAtZero}[2]{\ensuremath{\AtZero{\difff{#1}{#2}}}}
 \newcommand*{\difffpAtZero}[3]{\ensuremath{\AtZero{\difffp{#1}{#2}{#3}}}}
 \newcommand*{\difffppAtZero}[4]{\ensuremath{\AtZero{\difffpp{#1}{#2}{#3}{#4}}}}

 %-----------------------------------------------------------------------------
 % Defined
 % How should the defined operator look like (:= or ^= ==)
 % (I want back my :=, it is so much better than ^= because (1) it has a
 % direction and (2) everyone here uses it.)
 %
 % Use :=
 \newcommand*{\defined}{\ensuremath{\mathrel{\mathop{:}}=}}
 \newcommand*{\definedRight}{\ensuremath{=\mathrel{\mathop{:}}}}
 % Use ^=
 %\newcommand*{\defined}{\ensuremath{\triangleq}}
 %\newcommand*{\definedRight}{\ensuremath{\triangleq}}
 % Use = with three bars
 %\newcommand*{\defined}{\ensuremath{?}}
 %\newcommand*{\definedRight}{\ensuremath{?}}

%-----------------------------------------------------------------------------
 % Domains
 \newcommand*{\D}{\mathcal{D}}
 \newcommand*{\I}{\mathcal{I}}

 %-----------------------------------------------------------------------------
 % Texture coordinates
 \newcommand*{\rr}{\VEC{r}}

 %-----------------------------------------------------------------------------
 % Parameters
 \newcommand*{\pt}{\VEC{\tau}}
 \newcommand*{\pr}{\VEC{\rho}}
 \newcommand*{\pp}{\VEC{p}}
% \newcommand*{\qq}{\VEC{q}}
 \newcommand*{\xx}{\VEC{x}}
 \newcommand*{\deltaq}{\Delta \qq}
 \newcommand*{\deltap}{\Delta \pp}
 \newcommand*{\zz}{\VEC{z}}
 \newcommand*{\pa}{\VEC{\alpha}}
 \newcommand*{\qa}{\VEC{\alpha}}
% \newcommand*{\pb}{\VEC{\beta}}

 %-----------------------------------------------------------------------------
 % Optimal appearance parameters
 \newcommand*{\pbh}[1]{\ensuremath{\hat{\pb}({#1})}}

 %-----------------------------------------------------------------------------
 % Warp basis
 \newcommand*{\M}[1]{\ensuremath{M({#1})}}
 \newcommand*{\LL}[1]{\ensuremath{L({#1})}}

 %-----------------------------------------------------------------------------
 % Matrices of the texture model
 \newcommand*{\AM}[1]{\ensuremath{\Lambda(#1)}}               % Lambda(beta) 
 \newcommand*{\AMr}[2]{\ensuremath{\Lambda(#1; #2)}}        % Lambda(r, beta)

 \newcommand*{\As}{A}         % Continuous Basis symbol
 \newcommand*{\afs}{a}        % Continuous mean symbol
 \newcommand*{\Ab}[1]{\As(#1)}         % Continuous Basis
 \newcommand*{\af}[1]{\afs(#1)}        % Continuous mean


 %-----------------------------------------------------------------------------
 % Matrices of the shape model
 \newcommand*{\MU}{\VEC{\mu}}
 \newcommand*{\MM}{\MAT{M}}

 %-----------------------------------------------------------------------------
 %% The project out matrix and operator
 \newcommand*{\INT}{\MAT{P}}
 \newcommand*{\INTf}{P}

 %-----------------------------------------------------------------------------
 % The identity matrix
 \newcommand*{\EYEtwo}{\Matrix{1 & 0\\0&1}}
 \newcommand*{\EYE}{\MAT E}
 \newcommand*{\EYEf}{E}

 % Wether to use subscripts or brackets for some function arguments
 % can be decided by commenting out the corresponding functions underneath
 %-----------------------------------------------------------------------------
 % Mapping
 \newcommand*{\Cs}[1]{\ensuremath{C^{#1}}} % C symbol
 \newcommand*{\C}[2]{\ensuremath{C^{#1}(#2)}} % Use C with brackets

 %-----------------------------------------------------------------------------
 % Objective function
 \newcommand*{\Fs}{\ensuremath{F}}              % F symbol
 \newcommand*{\F}[1]{\ensuremath{\Fs(#1)}}       % Use F with brackets    F(q)

 %-----------------------------------------------------------------------------
 % Approximated objective functions
 \newcommand*{\FFs}{\tilde{F}}                     % ~F symbol
 \newcommand*{\FF}[1]{\ensuremath{\FFs(#1)}}       % Use ~F with brackets    F(q)

 %-----------------------------------------------------------------------------
 % residual function
 \newcommand*{\es}{\ensuremath{f}}              % R symbol

 \newcommand*{\e}[1]{\ensuremath{\es(#1)}}         % R(q)
 \newcommand*{\er}[2]{\ensuremath{\es(#1; #2)}}    % R(r; q)

 %-----------------------------------------------------------------------------
 % Approximated residual functions
 \newcommand*{\ees}{\tilde{f}}                       % ~R symbol
 \newcommand*{\ee}[1]{\ensuremath{\ees(#1)}}       % ~R(q)
 \newcommand*{\eer}[2]{\ensuremath{\ees(#2; #1)}}  % ~R(r; q)

 %-----------------------------------------------------------------------------
 % Warps
 \newcommand*{\Vs}{\ensuremath{V}}
 \newcommand*{\VLins}{\ensuremath{\Vs^{\text{Ortho}}}}
 \newcommand{\VModels}{\ensuremath{\Vs^{\text{Model}}}}
 \newcommand*{\Ws}{\ensuremath{W}}

 \newcommand{\V}[1]{\ensuremath{\Vs(#1)}}
 \newcommand{\VModel}[1]{\ensuremath{\VModels(#1)}}
 \newcommand{\Vr}[2]{\ensuremath{\Vs(#1; #2)}}
 \newcommand{\VInvr}[2]{\ensuremath{\Vs^{-1}(#1; #2)}}
 \newcommand{\VrLin}[2]{\ensuremath{\VLins(#1; #2)}}
 \newcommand{\W}[1]{\ensuremath{\Ws(#1)}}
 \newcommand{\Winv}[1]{\ensuremath{\Ws^{-1}(#1)}}
 \newcommand{\Wr}[2]{\ensuremath{\Ws(#1; #2)}}
 \renewcommand{\arraystretch}{0.65}
 
%-----------------------------------------------------------

% set equation numbering to include section and subsection numbers
\numberwithin{equation}{chapter}

\makeatletter
\let\OldStatex\Statex
\renewcommand{\Statex}[1][3]{%
  \setlength\@tempdima{\algorithmicindent}%
  \OldStatex\hskip\dimexpr#1\@tempdima\relax}
\makeatother

\usefonttheme{professionalfonts} % using non standard fonts for beamer
%\usefonttheme{serif} % default family is serif
\usepackage{pgfplots}
\pgfplotsset{compat=1.10}
\usetikzlibrary{cd,shapes,arrows}
\tikzset{>=latex'}

\makeatletter
\newif\ifdraft
\@ifclasswith{beamer}{draft}{\drafttrue}{\draftfalse}
\makeatother

\usepackage{appendixnumberbeamer}
\usepackage{hyperref}

\title[Responsible Softmax]{A Responsibile Softmax Layer in Deep Learning}
\author{Ryan Coatney}
\institute{\tiny{University of Arizona}}

\date{18 June 2018}

\begin{document}
	\maketitle
%	\section{Introduction}
%	\begin{frame}{Abstract}
%		\begin{itemize}
%			\item  \citet{MacKay2002,MML_2019} use \textit{responsibility} in their description of the $K$-means and EM algorithms.
%			\item \alert<2>{\textbf{Dynamic responsibility}} is a related concept that calculates a MLE for class mixture components.
%%			\item Dynamic responsibility is similar to the softmax transfer function used in neural networks.  This connection allows definition of a \alert<3>{\RS} layer for use in classification.
%			\item \alert<3>{\textbf{Responsibility softmax}} enables neural network approximation of underlying distributions.
%			\item Responsibility softmax performs better on imbalanced data than the typical softmax layer.
%		\end{itemize}
%%		We use the concept of responsibility as used in  (and defined in \citet{MacKay2002,MML_2019}) to define a new method of maximum likelihood estimation called \alert<2>{\textbf{dynamic responsibility}}.  is closely related to the softmax function used in neural networks. This relationship may be leveraged to define a new layer called the \alert<3>{\textbf{\RS}} layer.
%%		
%%		\ \\
%%		
%%		The \alert<3>{\textbf{\RS}} layer has an advantage over regular softmax in cases where training and classification is on unbalanced data.  Responsibility softmax layer is also interpretable, but it introduces a new hyperparameter \( C \) that must be tuned for best results.
%	\end{frame}
	
	\section{Clustering and Classification}
	\begin{frame}{Clustering}
		\begin{figure}
			\centering
			\begin{subfigure}{.7\linewidth}
				% This file was created by matlab2tikz.
%
%The latest updates can be retrieved from
%  http://www.mathworks.com/matlabcentral/fileexchange/22022-matlab2tikz-matlab2tikz
%where you can also make suggestions and rate matlab2tikz.
%
\begin{tikzpicture}[scale = .7]

\begin{axis}[%
width=4.521in,
height=0.714in,
at={(0.758in,1.907in)},
scale only axis,
xmin=-28,
xmax=29,
ymin=-3,
ymax=6,
axis background/.style={fill=white},
axis x line*=bottom,
axis y line*=left,
legend style={legend cell align=left, align=left, draw=white!15!black}
]
\addplot[only marks, mark=*, mark options={}, mark size=0.7906pt, color=black, fill=black] table[row sep=crcr]{%
x	y\\
-2.12697932307494	0.734153401779101\\
0.122927902969405	0.343326281739181\\
2.63216147756917	-0.254327350450132\\
-1.75157286210449	0.158295635207253\\
4.27441349221991	0.846805181239452\\
-13.0390647515914	2.58385382612626\\
-12.8962191835208	1.52140009783339\\
8.88868267627099	1.16652701210223\\
17.6121652846523	0.0284510755368321\\
-5.32237756585857	0.655097866112154\\
-0.612345419675211	1.28552435339239\\
0.902331160383546	1.96452881953512\\
0.874869825424029	0.142284653299417\\
-22.1382789336631	0.832793190409463\\
0.586475495983049	-0.832525804626447\\
0.705041791029858	-0.515079395368846\\
-1.04450594719981	-0.147262435740308\\
16.3315637977649	0.477110256498039\\
18.018017152122	1.76404327515109\\
-19.3630024537047	2.01149032796427\\
17.4550912906385	0.319538988623299\\
-20.9896730083877	3.65594753327058\\
-1.18488286562443	-1.16507249030604\\
15.4184999256647	0.83152386335639\\
17.2987680991699	0.0502924317360712\\
-8.93380319123649	2.79179319828156\\
1.26705937469575	-0.867473260315087\\
-1.28438720527871	-0.713347054601585\\
9.10510417136045	1.83776595157136\\
-16.6847219126246	2.68883322594695\\
3.63186642821824	-0.814090659041176\\
18.5805298844266	0.832764559850249\\
-15.3759710202692	0.981067014407197\\
3.14711464411891	-0.322481950660631\\
-2.66614606350415	-0.81636677681243\\
13.7368400873437	-0.16353128552678\\
-21.4184855726264	1.31906368385877\\
7.12114418511638	0.43766322999098\\
19.507988751321	0.0437723624980891\\
-24.1664646761523	2.22569681559814\\
-16.9553906106059	1.35287449366543\\
2.51993009559619	-0.890071012287709\\
-8.150084059313	-0.326401140551159\\
17.1993711733966	1.26448446577924\\
-19.8703419557057	2.03730143411024\\
-1.52213604584959	0.474552418425034\\
16.3262811269147	0.953182799233613\\
-4.11811338337861	-0.194918408619307\\
-16.0532991785198	1.77656100238983\\
-21.147544483623	1.96171869287892\\
10.4173539499661	0.416229729477691\\
-20.4478206961817	1.64646946877365\\
-16.8413288519043	2.73928972359917\\
-19.5926898625094	3.26429398433028\\
-2.08567029493479	1.58862018685686\\
19.6301442962875	-0.574011910441037\\
18.5666799752281	1.98522946051986\\
14.7251490214037	1.13784605885349\\
13.2741394464552	-0.151069476579381\\
19.7549042027541	2.19333508225127\\
-18.9926741023293	1.87124347333019\\
17.4714641450038	0.597151143402294\\
-6.6079222964015	0.762651682985142\\
-17.5201101957292	2.20712888770216\\
-17.1132275615809	2.03969084863212\\
-15.9745545135678	1.57298185013394\\
-16.8927032415142	3.07138091918918\\
-14.3073317612216	3.04376566066148\\
17.5271489410172	0.973202923739207\\
1.91931444197215	-0.305233399040617\\
-18.9283621834201	0.905039250461469\\
19.7436446923	0.194417998993186\\
1.85100245803221	0.121798597267792\\
-20.6981526256077	2.0605043876971\\
-15.7070979848966	1.1416615059414\\
-14.33743596191	3.0377338917311\\
-22.7807585452835	1.29817383619715\\
18.6099159809055	0.683255765856225\\
23.2922016344413	1.99400560113624\\
-16.8097839579789	1.12389141168737\\
-19.8460760759784	2.00555005595308\\
-0.2042031101242	1.24619746275841\\
20.9582517207	0.614385562682709\\
-1.92189047030412	0.100019093319906\\
13.3940092228053	-0.125102704119033\\
-17.5137187641193	1.96083612560374\\
-17.1222284681888	3.32890683803807\\
0.142448697988598	0.093342814247212\\
16.354146626165	0.728147874655163\\
3.98605221014808	1.14956883942006\\
18.8448793560701	0.0229030510036594\\
-8.99755060220678	1.52015401747728\\
2.63553280705291	-0.939661932730858\\
1.12351453006449	1.43470248260702\\
-17.7869324026094	1.44232990683879\\
11.8475614672186	0.433077066828136\\
9.22914549273269	0.979552930295771\\
22.8443184050309	1.09603705904798\\
0.872470481316697	-0.493244756439886\\
14.9278800277285	1.49094705485706\\
-20.5218373633359	2.03704146699485\\
2.35548551922442	-1.02413757243261\\
-19.1168508947509	2.86616596765723\\
-11.6504987722591	0.355011233577846\\
18.0407586248709	0.197226416112856\\
13.0502232670168	-0.000678917350893871\\
-5.56068841260192	0.651666514866178\\
-18.5661635737835	1.82036107176167\\
-14.5648684024875	1.7472471368815\\
23.2384793311994	1.05697993624236\\
3.64427636120491	0.72558231243753\\
21.1780345912468	1.28407465671138\\
18.8938314621014	1.65124207113436\\
-20.9558009809282	2.29438182064043\\
9.05754612856061	2.11573598821182\\
14.4335219567506	0.12861717145351\\
-13.2995965679574	1.18603506901456\\
-15.355388414595	2.28552129200204\\
-20.4427160947005	2.36507745190789\\
3.32687032386843	0.143938803413768\\
2.52958968202736	-0.359283496158863\\
3.57433605400171	0.663738533205764\\
15.9738066140656	1.65626609980807\\
-20.6940885228748	1.97046721642786\\
-20.8993955148484	1.92066864410809\\
8.09721024622322	-0.611171648587698\\
23.6534099624505	0.0414558355825206\\
-2.16803202677901	-0.86138337785408\\
21.441677428739	0.822888233599663\\
15.261495878196	0.93323024809634\\
-5.51602927846288	-1.46425105689166\\
18.2101373700008	1.87174033027409\\
18.155509773026	0.17915689348163\\
6.25890459965704	0.0773331675839121\\
-14.1571379667395	0.708256765225689\\
6.7601413473542	2.28735033003258\\
19.3944035440371	1.26683759064637\\
20.9888471609451	2.11312739610155\\
2.21156293927144	1.26359652704661\\
22.3910505659835	1.03963436158509\\
6.17113714949222	0.499581029724201\\
13.8465226735953	1.63508063545026\\
5.01094534362827	-1.94768754936625\\
15.3103495661314	0.617242628527914\\
4.17369397015448	-0.199454091608638\\
24.8406301916814	-0.0299059505607522\\
-16.1803627926534	2.81859806719155\\
14.3433093866462	-0.0327054928981054\\
-17.4792398563394	1.76692350986799\\
-9.05197724640506	1.82018628882538\\
-15.6508742273441	2.12861025495932\\
-6.87392325124233	-0.726047041941697\\
-0.961470235928968	0.659302657319334\\
1.11561768286045	0.678727026324926\\
15.6884517450919	1.87308089864853\\
-2.20669665539958	-1.83393617244678\\
2.90107954030742	-0.0538228571054752\\
0.030775561275574	-0.505694414308935\\
-14.2450891916411	1.99676758220591\\
-19.3292014258364	1.31084860695427\\
-0.248887709450454	-1.03357022167571\\
13.7899160591559	2.58352660492817\\
-6.31240513420851	0.284799378628209\\
-18.5071565199403	1.83107733246302\\
18.0322650362449	0.44777490923522\\
21.5648747265458	0.585776172591518\\
23.9532440167402	1.28243731966051\\
4.97791519316952	-0.600916978844473\\
7.47636446135275	1.05380851512145\\
20.104162058142	0.542452380501339\\
20.976944981943	1.80350128837074\\
6.14444924191286	0.493848567147238\\
4.98053471281628	-0.992591915463532\\
-6.31493711080174	-0.528423660086781\\
23.9123435040969	1.22387335372774\\
-0.300796098229382	-1.70441880830099\\
12.88503160579	1.59239352154747\\
13.3363144449699	1.99474943299734\\
0.180636022170158	0.00222664574961195\\
-18.0221771013136	1.25226759125618\\
-3.416981264793	0.982464916325299\\
19.5715414584367	2.28646853126353\\
-0.426980860434325	-0.266386116143043\\
16.6949839151162	1.47068028426394\\
-2.29843043769072	0.0428034072020241\\
1.48022093786953	-0.0369553108551366\\
0.215090813586244	0.454357512649916\\
3.71189859955171	0.667925057663898\\
-23.2612528355385	2.582088452385\\
-4.79264380510488	-0.155934649338532\\
16.1909596000321	2.47573874718405\\
-9.53811106905465	1.21143129262326\\
-19.6648767538016	0.994808823993323\\
-5.12862759791042	-0.520330078272704\\
-15.0533521845265	0.490030490326571\\
12.6143887881425	1.7111534897066\\
14.7753853211326	1.0132557478699\\
-1.30048040907759	-0.431937706250186\\
3.23422507161324	-1.3025603140885\\
2.45114322778812	0.317476630131372\\
13.2582951164167	2.22271410151697\\
-8.84971845076545	2.17176324454394\\
-16.8180168857178	1.45576663204976\\
16.4534351894443	0.570465032066382\\
7.33986622212505	0.364469767025394\\
-17.7246315134906	1.82242875133601\\
17.3973815351121	1.34973281429677\\
-15.7921465704791	2.76054262979188\\
16.7916852067438	0.811718729850322\\
23.6142057966694	0.84199278525811\\
22.4577829723654	2.49206622756598\\
13.3224454612036	2.23287543910022\\
11.105584350006	0.620470176134218\\
-0.984604320733126	-1.1513711051391\\
-0.911459934255873	0.475121213330907\\
1.59722012298325	1.23913855668169\\
-12.8630622367322	2.05447012943979\\
-1.22324225553094	-0.0589341694579021\\
-17.1924404357685	2.99558385564624\\
-2.80518274783678	-0.00902135572686707\\
-17.6272401459748	1.69926327140825\\
-14.1891567475565	1.83604723406803\\
-1.18212477059333	0.183005682365753\\
13.8707508201733	1.15839106702986\\
6.26827588273483	0.225574782535944\\
-12.6720022098057	1.371159292207\\
-20.2405176553229	0.556046227232444\\
18.4098771545399	-0.188638341733553\\
-26.0375438316404	2.95267185755971\\
-18.220984946945	2.01807644198578\\
-17.314409185012	2.12798826781049\\
0.252695306713707	0.0212103384108863\\
-4.8296345070935	-0.217075267468112\\
-16.6625017814217	1.42620179366784\\
-18.8169715210963	2.29040632696061\\
0.596009852112165	1.02511255833282\\
-2.13709275859671	-0.889853564462552\\
15.4710071128111	0.610579682826467\\
-9.01792211391009	2.60235462937461\\
6.55587645031129	0.478265197642075\\
12.1623916761271	1.55254313646863\\
14.17342311664	0.933580523259207\\
-2.1816456927356	-0.0607979823126644\\
1.97527355758178	-0.186221009891955\\
-0.71197777236988	1.51547738153152\\
20.2409243606416	1.51760562017255\\
-9.30538967278889	1.82274702568521\\
15.4542607039897	1.68716899234405\\
-20.2544416695237	0.776970463551071\\
2.17164317097379	0.348417453483848\\
18.546042613234	1.13134433433068\\
-1.63377214861198	0.0233836114569938\\
-5.18880612934132	0.0735722801439829\\
20.272293735638	1.62950588032582\\
15.3137376955454	2.59547372884512\\
-11.5770161103664	2.78530405972726\\
-13.6578040906098	2.53764945924901\\
-22.2594718812781	1.45952736923485\\
-2.55311086439217	-1.89757348850271\\
-16.6147545232877	1.38623296795023\\
-11.9336108227027	0.850271133692976\\
19.10038941662	0.746267755092501\\
18.5140798912944	1.16196731602621\\
-14.0820026152571	2.46231536925236\\
17.0033969680974	0.0923861891419633\\
-13.7716433866066	1.66702728548603\\
-21.5465114288155	2.27393038216814\\
-19.266892542561	1.5063570186254\\
17.957293797529	1.20526329536001\\
0.855078802458656	-1.45519140817519\\
15.0913649928761	1.80037076338443\\
14.4283409108264	1.59556331296322\\
-18.3592790884494	1.39610074503546\\
-0.908852166537197	-0.179818610218128\\
1.38930662361828	-1.0126894039102\\
-20.8874869632382	3.5370459111741\\
19.8398389888065	0.473323900134359\\
-3.56518578628586	0.0694002705786866\\
-14.9472070463345	2.09648397420584\\
16.4166007335949	1.28367285196227\\
-0.903231390652451	-0.574647079937109\\
0.451360804117812	-0.445534987556561\\
3.09547678566735	-0.226532505853323\\
-18.6723732578719	1.45088340704962\\
18.4097167174584	1.25606409458656\\
18.0066547612642	0.673202883509209\\
21.2377358237767	0.996552217540614\\
-8.84599216092668	1.46455780243804\\
1.74237052222676	0.20474176535925\\
-14.2643656566544	2.0359986713898\\
13.9932823959732	2.2810748776201\\
18.1689305463369	-0.476035595381123\\
21.3094273166437	1.43598121181515\\
20.4371710628645	1.35885558856491\\
-17.9319893325714	1.67540933733375\\
-19.703206349711	1.81469320025894\\
0.789278993735694	-0.276628818102428\\
-1.88560086380028	-1.41052388252961\\
-12.3124098407243	1.80779432894724\\
-0.937783966196865	0.646631939551512\\
-10.8927388023256	2.37111067588025\\
-9.05820744933176	1.04164084227071\\
-19.256797375881	1.60506823682092\\
18.363715910536	-0.362376451908442\\
5.31405479380758	-0.652492053763823\\
19.1583659796918	1.31306296323931\\
-16.3258057443609	0.777287963460357\\
-19.7900590331145	2.41352568053095\\
-18.9170914580381	1.8718152143672\\
8.99536136719112	1.76274859833112\\
16.5642088765996	0.911737483580825\\
11.0343990909932	0.368501798509585\\
-0.28034321414595	0.391086743553475\\
-2.04916101500488	1.17082765129311\\
-21.5399446430118	1.82042502121844\\
-13.9949450541028	2.9352749074465\\
16.922682720907	0.388459571640427\\
-8.80493044061983	1.53633185203394\\
16.4939570946887	1.52600447864613\\
17.9241467458037	1.20419497507464\\
-23.04895664004	1.63976967797421\\
-17.7696745813826	1.18979757417922\\
23.166335934286	-0.582644707088908\\
-2.27230023287371	0.71458874712357\\
-0.287675662479843	-0.711813345667602\\
-0.360363973361282	-0.197989109684111\\
27.331540620025	0.757557933311197\\
13.1827156854743	2.45728397123268\\
14.3027949751858	0.7553324891235\\
-15.9994364607955	1.08418419031775\\
3.37260053176413	1.31896457301187\\
5.5439344918532	-0.19210767319462\\
-5.05968541613884	-0.335334018370392\\
18.0339424932215	2.32661820778628\\
8.60700215498597	1.1120486202389\\
-7.26158307013133	2.01704240998788\\
-16.6084518187844	1.98483496094024\\
2.47546508347088	-0.952371623973633\\
-16.2460980696874	1.86558534829802\\
-8.86791320426481	3.36046530970861\\
-20.16845812789	1.48182070413641\\
-22.3692119170163	3.52921770601893\\
-0.698131614729261	1.47799550815875\\
-1.33272594858178	-0.362846898895077\\
-17.5905176335567	2.08671647451577\\
-14.8453168268134	2.02048846239236\\
-12.4930155399117	1.25966754217746\\
15.568653564821	1.19830887297365\\
0.324827405578217	0.100716870566698\\
4.70763340065599	-0.261582709569126\\
-8.9790121341234	1.2592543853217\\
-2.43227676137703	-1.00529674280893\\
-3.42556792212039	0.276213723448114\\
12.9191146184821	1.91661960198528\\
-10.3397290342613	2.57818687141448\\
17.1953199307291	0.3874005091279\\
-9.78350537989637	1.46400178823938\\
-1.39637905539436	-0.225707022587075\\
-2.04607986438169	-0.215557883234501\\
26.7269907349407	1.0647149207353\\
-18.3641829752053	2.9967222947365\\
9.35861088149241	0.184108569870373\\
0.365726348718405	-1.1978742231361\\
22.1650585189586	1.06285225424231\\
16.6361575816641	0.475635932289157\\
22.4456245787494	1.09664925459472\\
2.52960646636026	-0.388887344581937\\
-17.4351872477026	2.73216253164856\\
16.4884784363945	2.07652272592363\\
-1.2072541057224	0.374308590663764\\
-0.3456293494986	-0.145244366246396\\
16.129887365198	0.0136479432515966\\
-19.3957506309308	1.13951918220314\\
-24.0369772023756	1.06459681474729\\
17.9361581501116	1.71637095613658\\
21.039049657626	0.351829545338599\\
-24.3538721392332	3.26132825906002\\
-18.0242540116346	0.190996792906936\\
8.9568917100496	0.738627180656787\\
0.294461030552158	-0.427908025119613\\
-18.6920665750535	2.76421624144904\\
16.0513510574982	1.16630018930676\\
-15.6202347031393	0.783212942891168\\
0.0953110219656994	-1.70005083538807\\
-18.1616523223102	2.44001838489488\\
0.74440013857161	0.125261145615376\\
19.3366085813951	-0.124950052767057\\
-16.9543559749638	1.7393735908863\\
-1.08475494287602	1.26553147567952\\
-19.9074165473818	1.75178430304521\\
-13.2038863617528	2.5053026473487\\
-12.7609370672405	0.90319601193845\\
-12.8565793539235	2.63698792513635\\
-12.5603979879597	1.80454105330398\\
-21.1158322955079	1.53149922906964\\
-16.4299803605004	2.71289957054009\\
15.5510029351612	2.13430535609685\\
-5.78790560912164	0.940689180084404\\
-12.4079712589374	2.59085067590129\\
0.741995569408901	1.35097729821223\\
-14.3245048265129	3.36997359881242\\
-8.8245412536474	0.644504466138986\\
18.2027621888613	2.15003988355839\\
16.5285380101286	0.656943261440391\\
0.0456950789312266	-1.57757798407271\\
-2.04961962072843	1.10318927672836\\
-21.7081763745371	0.84136398202266\\
-18.6581535233444	3.10952054457128\\
0.8621464551415	1.16432658152399\\
18.3125820819632	0.706939370903543\\
-15.9421314045719	0.590360840174064\\
-1.07139662572615	1.02027722819588\\
-15.8023465839867	1.95446643338996\\
-14.794783132387	2.38082430885078\\
-3.13764780321976	0.965134649490343\\
0.144893956518119	0.275211885948681\\
1.61730418631774	-1.2610760554924\\
16.2402719973496	0.604288317188662\\
-19.0114324045954	2.41811473613716\\
15.9130048249204	1.52833459773368\\
-0.403811873747295	1.06905385667546\\
-19.337960273146	1.89886097994714\\
20.5611472972269	0.217867377187204\\
17.8022392993506	0.699417957078192\\
15.7789368617815	0.630457904166283\\
-19.7713156159225	2.36150066513532\\
3.27068173235815	1.99413329272505\\
2.34080673117722	-1.22547354678251\\
1.44082177941281	0.427748841757203\\
5.27461449560866	0.665681917910806\\
-13.7175163130788	1.82584031331161\\
20.7882273234874	1.09037313101011\\
-17.0422211463866	0.87364680950016\\
18.3774073265841	1.31004470392943\\
-0.382707098787698	0.0854458716008148\\
-16.2420973542429	1.19496295885525\\
1.21549675330816	1.76344114924328\\
-19.4104519538838	4.14543345619423\\
-2.71521332697903	-0.102543354291735\\
-0.379766638050199	-0.855147739074738\\
15.8316594558143	0.272073501902906\\
1.04457360506462	0.125183188338292\\
0.962720049706309	-0.0800567896487775\\
-25.8489569715911	2.79875211428775\\
-18.3812639685904	1.7586807693835\\
-0.870901740780474	0.499120192804166\\
15.0945414408655	1.55191224157982\\
8.69371609590603	2.81899678836542\\
13.8211720815551	1.85736428078591\\
-6.51317518254709	-0.942033261883492\\
-5.36453330437023	0.066848381796742\\
6.10534226133907	-0.955176071431222\\
-19.1687777013071	1.62710609771755\\
0.195999663312141	-0.212289209885246\\
12.6152921700468	2.06837899542438\\
18.2950063728581	-0.430654941012518\\
19.4028271244944	0.927556874797489\\
-8.4334129488841	2.75632512624446\\
-15.8253944662369	1.68223737191408\\
-16.8370940110564	0.854853059585934\\
-6.1738763898852	0.0526790421663295\\
-0.635265042724267	0.300794393628996\\
13.641355203277	1.00873810257832\\
19.2574938476782	0.683866176061944\\
-22.2539461423157	3.01546795186151\\
-12.3660000572511	1.46661867594727\\
-17.8536260489373	2.64910637072127\\
-20.0068609608008	1.44615028101763\\
-18.6640354368031	3.64209241242855\\
16.5363360864435	-0.273283420489044\\
15.9605422002912	0.14748529314626\\
2.04476138674395	-0.417954355091849\\
15.610420766984	1.63589437139189\\
-6.03344592998271	-0.0635857588602965\\
2.67072088228364	-0.478585218055148\\
-15.7485879890396	0.736956534802068\\
2.38507038121179	-0.582062748571733\\
19.2161825075766	0.81836688418762\\
-20.1060973892671	2.19092272494124\\
18.5807489052253	2.36815805625131\\
-21.9004488771623	1.98921828484528\\
0.708044040305444	-0.435635586393627\\
-11.2549783018629	2.31122336690269\\
24.5690905173359	2.10040148839898\\
-2.57823083051533	1.40005582001389\\
9.56559457487811	0.957885503999155\\
-21.7961066930105	2.51678067987402\\
-19.6878591171879	2.3580626609474\\
-17.1171250701597	0.793253876026244\\
-21.7472411913366	2.31476596263571\\
-0.293399002597656	0.432826955761419\\
15.2701354453626	0.44354746509279\\
23.8133662612195	-0.191602757600895\\
25.6653274924032	2.16872531282788\\
-14.0607500225465	2.92555277320955\\
-22.3842291018574	2.06782043743519\\
2.95165487208015	1.88212968797264\\
15.5862457224728	0.184979417130069\\
-15.7266731682233	2.63279975541401\\
-2.91370012284536	0.093849066181965\\
};

\end{axis}


\end{tikzpicture}%
			\end{subfigure}
			\pause
			\begin{subfigure}{.7\linewidth}
				\ \\
			\end{subfigure}
			\begin{subfigure}{.7\linewidth}
				% This file was created by matlab2tikz.
%
%The latest updates can be retrieved from
%  http://www.mathworks.com/matlabcentral/fileexchange/22022-matlab2tikz-matlab2tikz
%where you can also make suggestions and rate matlab2tikz.
%
\begin{tikzpicture}[scale = .7]

\begin{axis}[%
width=4.521in,
height=0.714in,
at={(0.758in,1.907in)},
scale only axis,
colormap={mymap}{[1pt] rgb(0pt)=(0,0.4470,0.7410); rgb(1pt)=(0.8500,0.3250,0.0980); rgb(2pt)=(0.9290,0.6940,0.1250); rgb(3pt)=(0.4940,0.1840,0.5560); rgb(4pt)=(0.4660,0.6740,0.1880); rgb(5pt)=(0.3010,0.7450,0.9330); rgb(6pt)=(0.6350,0.0780,0.1840)},
xmin=-28,
xmax=29,
ymin=-3,
ymax=6,
axis background/.style={fill=white},
axis x line*=bottom,
axis y line*=left,
legend style={legend cell align=left, align=left, draw=white!15!black},
]
\addplot[scatter, only marks, mark=*, mark size=0.7906pt, scatter src=explicit, scatter/use mapped color={mark options={}, draw=mapped color, fill=mapped color}] table[row sep=crcr, meta=color]{%
x	y	color\\
-2.12697932307494	0.734153401779101    5\\
0.122927902969405	0.343326281739181    5\\
2.63216147756917	-0.254327350450132    5\\
-1.75157286210449	0.158295635207253    5\\
4.27441349221991	0.846805181239452    5\\
-13.0390647515914	2.58385382612626	1\\
-12.8962191835208	1.52140009783339	1\\
8.88868267627099	1.16652701210223     2.3\\
17.6121652846523	0.0284510755368321   3\\
-5.32237756585857	0.655097866112154    5\\
-0.612345419675211	1.28552435339239    5\\
0.902331160383546	1.96452881953512    5\\
0.874869825424029	0.142284653299417    5\\
-22.1382789336631	0.832793190409463	1\\
0.586475495983049	-0.832525804626447    5\\
0.705041791029858	-0.515079395368846    5\\
-1.04450594719981	-0.147262435740308    5\\
16.3315637977649	0.477110256498039   3\\
18.018017152122	1.76404327515109   3\\
-19.3630024537047	2.01149032796427	1\\
17.4550912906385	0.319538988623299   3\\
-20.9896730083877	3.65594753327058	1\\
-1.18488286562443	-1.16507249030604    5\\
15.4184999256647	0.83152386335639   3\\
17.2987680991699	0.0502924317360712   3\\
-8.93380319123649	2.79179319828156         3.6\\
1.26705937469575	-0.867473260315087    5\\
-1.28438720527871	-0.713347054601585    5\\
9.10510417136045	1.83776595157136     2.3\\
-16.6847219126246	2.68883322594695	1\\
3.63186642821824	-0.814090659041176    5\\
18.5805298844266	0.832764559850249   3\\
-15.3759710202692	0.981067014407197	1\\
3.14711464411891	-0.322481950660631    5\\
-2.66614606350415	-0.81636677681243    5\\
13.7368400873437	-0.16353128552678   3\\
-21.4184855726264	1.31906368385877	1\\
7.12114418511638	0.43766322999098    5\\
19.507988751321	0.0437723624980891   3\\
-24.1664646761523	2.22569681559814	1\\
-16.9553906106059	1.35287449366543	1\\
2.51993009559619	-0.890071012287709    5\\
-8.150084059313	-0.326401140551159    5\\
17.1993711733966	1.26448446577924   3\\
-19.8703419557057	2.03730143411024	1\\
-1.52213604584959	0.474552418425034    5\\
16.3262811269147	0.953182799233613   3\\
-4.11811338337861	-0.194918408619307    5\\
-16.0532991785198	1.77656100238983	1\\
-21.147544483623	1.96171869287892	1\\
10.4173539499661	0.416229729477691   3\\
-20.4478206961817	1.64646946877365	1\\
-16.8413288519043	2.73928972359917	1\\
-19.5926898625094	3.26429398433028	1\\
-2.08567029493479	1.58862018685686    5\\
19.6301442962875	-0.574011910441037   3\\
18.5666799752281	1.98522946051986   3\\
14.7251490214037	1.13784605885349   3\\
13.2741394464552	-0.151069476579381   3\\
19.7549042027541	2.19333508225127   3\\
-18.9926741023293	1.87124347333019	1\\
17.4714641450038	0.597151143402294   3\\
-6.6079222964015	0.762651682985142    5\\
-17.5201101957292	2.20712888770216	1\\
-17.1132275615809	2.03969084863212	1\\
-15.9745545135678	1.57298185013394	1\\
-16.8927032415142	3.07138091918918	1\\
-14.3073317612216	3.04376566066148	1\\
17.5271489410172	0.973202923739207   3\\
1.91931444197215	-0.305233399040617    5\\
-18.9283621834201	0.905039250461469	1\\
19.7436446923	0.194417998993186   3\\
1.85100245803221	0.121798597267792    5\\
-20.6981526256077	2.0605043876971	1\\
-15.7070979848966	1.1416615059414	1\\
-14.33743596191	3.0377338917311	1\\
-22.7807585452835	1.29817383619715	1\\
18.6099159809055	0.683255765856225   3\\
23.2922016344413	1.99400560113624   3\\
-16.8097839579789	1.12389141168737	1\\
-19.8460760759784	2.00555005595308	1\\
-0.2042031101242	1.24619746275841    5\\
20.9582517207	0.614385562682709   3\\
-1.92189047030412	0.100019093319906    5\\
13.3940092228053	-0.125102704119033   3\\
-17.5137187641193	1.96083612560374	1\\
-17.1222284681888	3.32890683803807	1\\
0.142448697988598	0.093342814247212    5\\
16.354146626165	0.728147874655163   3\\
3.98605221014808	1.14956883942006    5\\
18.8448793560701	0.0229030510036594   3\\
-8.99755060220678	1.52015401747728         3.6\\
2.63553280705291	-0.939661932730858    5\\
1.12351453006449	1.43470248260702    5\\
-17.7869324026094	1.44232990683879	1\\
11.8475614672186	0.433077066828136   3\\
9.22914549273269	0.979552930295771     2.3\\
22.8443184050309	1.09603705904798   3\\
0.872470481316697	-0.493244756439886    5\\
14.9278800277285	1.49094705485706   3\\
-20.5218373633359	2.03704146699485	1\\
2.35548551922442	-1.02413757243261    5\\
-19.1168508947509	2.86616596765723	1\\
-11.6504987722591	0.355011233577846	1\\
18.0407586248709	0.197226416112856   3\\
13.0502232670168	-0.000678917350893871   3\\
-5.56068841260192	0.651666514866178    5\\
-18.5661635737835	1.82036107176167	1\\
-14.5648684024875	1.7472471368815	1\\
23.2384793311994	1.05697993624236   3\\
3.64427636120491	0.72558231243753    5\\
21.1780345912468	1.28407465671138   3\\
18.8938314621014	1.65124207113436   3\\
-20.9558009809282	2.29438182064043	1\\
9.05754612856061	2.11573598821182     2.3\\
14.4335219567506	0.12861717145351   3\\
-13.2995965679574	1.18603506901456	1\\
-15.355388414595	2.28552129200204	1\\
-20.4427160947005	2.36507745190789	1\\
3.32687032386843	0.143938803413768    5\\
2.52958968202736	-0.359283496158863    5\\
3.57433605400171	0.663738533205764    5\\
15.9738066140656	1.65626609980807   3\\
-20.6940885228748	1.97046721642786	1\\
-20.8993955148484	1.92066864410809	1\\
8.09721024622322	-0.611171648587698    5\\
23.6534099624505	0.0414558355825206   3\\
-2.16803202677901	-0.86138337785408    5\\
21.441677428739	0.822888233599663   3\\
15.261495878196	0.93323024809634   3\\
-5.51602927846288	-1.46425105689166    5\\
18.2101373700008	1.87174033027409   3\\
18.155509773026	0.17915689348163   3\\
6.25890459965704	0.0773331675839121    5\\
-14.1571379667395	0.708256765225689	1\\
6.7601413473542	2.28735033003258   3\\
19.3944035440371	1.26683759064637   3\\
20.9888471609451	2.11312739610155   3\\
2.21156293927144	1.26359652704661    5\\
22.3910505659835	1.03963436158509   3\\
6.17113714949222	0.499581029724201    5\\
13.8465226735953	1.63508063545026   3\\
5.01094534362827	-1.94768754936625    5\\
15.3103495661314	0.617242628527914   3\\
4.17369397015448	-0.199454091608638    5\\
24.8406301916814	-0.0299059505607522   3\\
-16.1803627926534	2.81859806719155	1\\
14.3433093866462	-0.0327054928981054   3\\
-17.4792398563394	1.76692350986799	1\\
-9.05197724640506	1.82018628882538         3.6\\
-15.6508742273441	2.12861025495932	1\\
-6.87392325124233	-0.726047041941697    5\\
-0.961470235928968	0.659302657319334    5\\
1.11561768286045	0.678727026324926    5\\
15.6884517450919	1.87308089864853   3\\
-2.20669665539958	-1.83393617244678    5\\
2.90107954030742	-0.0538228571054752    5\\
0.030775561275574	-0.505694414308935    5\\
-14.2450891916411	1.99676758220591	1\\
-19.3292014258364	1.31084860695427	1\\
-0.248887709450454	-1.03357022167571    5\\
13.7899160591559	2.58352660492817   3\\
-6.31240513420851	0.284799378628209    5\\
-18.5071565199403	1.83107733246302	1\\
18.0322650362449	0.44777490923522   3\\
21.5648747265458	0.585776172591518   3\\
23.9532440167402	1.28243731966051   3\\
4.97791519316952	-0.600916978844473    5\\
7.47636446135275	1.05380851512145    5\\
20.104162058142	0.542452380501339   3\\
20.976944981943	1.80350128837074   3\\
6.14444924191286	0.493848567147238    5\\
4.98053471281628	-0.992591915463532    5\\
-6.31493711080174	-0.528423660086781    5\\
23.9123435040969	1.22387335372774   3\\
-0.300796098229382	-1.70441880830099    5\\
12.88503160579	1.59239352154747    5\\
13.3363144449699	1.99474943299734   3\\
0.180636022170158	0.00222664574961195    5\\
-18.0221771013136	1.25226759125618	1\\
-3.416981264793	0.982464916325299    5\\
19.5715414584367	2.28646853126353   3\\
-0.426980860434325	-0.266386116143043    5\\
16.6949839151162	1.47068028426394   3\\
-2.29843043769072	0.0428034072020241    5\\
1.48022093786953	-0.0369553108551366    5\\
0.215090813586244	0.454357512649916    5\\
3.71189859955171	0.667925057663898    5\\
-23.2612528355385	2.582088452385	1\\
-4.79264380510488	-0.155934649338532    5\\
16.1909596000321	2.47573874718405   3\\
-9.53811106905465	1.21143129262326         3.6\\
-19.6648767538016	0.994808823993323	1\\
-5.12862759791042	-0.520330078272704    5\\
-15.0533521845265	0.490030490326571	1\\
12.6143887881425	1.7111534897066   3\\
14.7753853211326	1.0132557478699   3\\
-1.30048040907759	-0.431937706250186    5\\
3.23422507161324	-1.3025603140885    5\\
2.45114322778812	0.317476630131372    5\\
13.2582951164167	2.22271410151697   3\\
-8.84971845076545	2.17176324454394         3.6\\
-16.8180168857178	1.45576663204976	1\\
16.4534351894443	0.570465032066382   3\\
7.33986622212505	0.364469767025394    5\\
-17.7246315134906	1.82242875133601	1\\
17.3973815351121	1.34973281429677   3\\
-15.7921465704791	2.76054262979188	1\\
16.7916852067438	0.811718729850322   3\\
23.6142057966694	0.84199278525811   3\\
22.4577829723654	2.49206622756598   3\\
13.3224454612036	2.23287543910022   3\\
11.105584350006	0.620470176134218   3\\
-0.984604320733126	-1.1513711051391    5\\
-0.911459934255873	0.475121213330907    5\\
1.59722012298325	1.23913855668169    5\\
-12.8630622367322	2.05447012943979	1\\
-1.22324225553094	-0.0589341694579021    5\\
-17.1924404357685	2.99558385564624	1\\
-2.80518274783678	-0.00902135572686707    5\\
-17.6272401459748	1.69926327140825	1\\
-14.1891567475565	1.83604723406803	1\\
-1.18212477059333	0.183005682365753    5\\
13.8707508201733	1.15839106702986   3\\
6.26827588273483	0.225574782535944    5\\
-12.6720022098057	1.371159292207	1\\
-20.2405176553229	0.556046227232444	1\\
18.4098771545399	-0.188638341733553   3\\
-26.0375438316404	2.95267185755971	1\\
-18.220984946945	2.01807644198578	1\\
-17.314409185012	2.12798826781049	1\\
0.252695306713707	0.0212103384108863    5\\
-4.8296345070935	-0.217075267468112    5\\
-16.6625017814217	1.42620179366784	1\\
-18.8169715210963	2.29040632696061	1\\
0.596009852112165	1.02511255833282    5\\
-2.13709275859671	-0.889853564462552    5\\
15.4710071128111	0.610579682826467   3\\
-9.01792211391009	2.60235462937461         3.6\\
6.55587645031129	0.478265197642075    5\\
12.1623916761271	1.55254313646863   3\\
14.17342311664	0.933580523259207   3\\
-2.1816456927356	-0.0607979823126644    5\\
1.97527355758178	-0.186221009891955    5\\
-0.71197777236988	1.51547738153152    5\\
20.2409243606416	1.51760562017255   3\\
-9.30538967278889	1.82274702568521         3.6\\
15.4542607039897	1.68716899234405   3\\
-20.2544416695237	0.776970463551071	1\\
2.17164317097379	0.348417453483848    5\\
18.546042613234	1.13134433433068   3\\
-1.63377214861198	0.0233836114569938    5\\
-5.18880612934132	0.0735722801439829    5\\
20.272293735638	1.62950588032582   3\\
15.3137376955454	2.59547372884512   3\\
-11.5770161103664	2.78530405972726	1\\
-13.6578040906098	2.53764945924901	1\\
-22.2594718812781	1.45952736923485	1\\
-2.55311086439217	-1.89757348850271    5\\
-16.6147545232877	1.38623296795023	1\\
-11.9336108227027	0.850271133692976	1\\
19.10038941662	0.746267755092501   3\\
18.5140798912944	1.16196731602621   3\\
-14.0820026152571	2.46231536925236	1\\
17.0033969680974	0.0923861891419633   3\\
-13.7716433866066	1.66702728548603	1\\
-21.5465114288155	2.27393038216814	1\\
-19.266892542561	1.5063570186254	1\\
17.957293797529	1.20526329536001   3\\
0.855078802458656	-1.45519140817519    5\\
15.0913649928761	1.80037076338443   3\\
14.4283409108264	1.59556331296322   3\\
-18.3592790884494	1.39610074503546	1\\
-0.908852166537197	-0.179818610218128    5\\
1.38930662361828	-1.0126894039102    5\\
-20.8874869632382	3.5370459111741	1\\
19.8398389888065	0.473323900134359   3\\
-3.56518578628586	0.0694002705786866    5\\
-14.9472070463345	2.09648397420584	1\\
16.4166007335949	1.28367285196227   3\\
-0.903231390652451	-0.574647079937109    5\\
0.451360804117812	-0.445534987556561    5\\
3.09547678566735	-0.226532505853323    5\\
-18.6723732578719	1.45088340704962	1\\
18.4097167174584	1.25606409458656   3\\
18.0066547612642	0.673202883509209   3\\
21.2377358237767	0.996552217540614   3\\
-8.84599216092668	1.46455780243804         3.6\\
1.74237052222676	0.20474176535925    5\\
-14.2643656566544	2.0359986713898	1\\
13.9932823959732	2.2810748776201   3\\
18.1689305463369	-0.476035595381123   3\\
21.3094273166437	1.43598121181515   3\\
20.4371710628645	1.35885558856491   3\\
-17.9319893325714	1.67540933733375	1\\
-19.703206349711	1.81469320025894	1\\
0.789278993735694	-0.276628818102428    5\\
-1.88560086380028	-1.41052388252961    5\\
-12.3124098407243	1.80779432894724	1\\
-0.937783966196865	0.646631939551512    5\\
-10.8927388023256	2.37111067588025	1\\
-9.05820744933176	1.04164084227071	1\\
-19.256797375881	1.60506823682092	1\\
18.363715910536	-0.362376451908442   3\\
5.31405479380758	-0.652492053763823    5\\
19.1583659796918	1.31306296323931   3\\
-16.3258057443609	0.777287963460357	1\\
-19.7900590331145	2.41352568053095	1\\
-18.9170914580381	1.8718152143672	1\\
8.99536136719112	1.76274859833112     2.3\\
16.5642088765996	0.911737483580825   3\\
11.0343990909932	0.368501798509585    5\\
-0.28034321414595	0.391086743553475    5\\
-2.04916101500488	1.17082765129311    5\\
-21.5399446430118	1.82042502121844	1\\
-13.9949450541028	2.9352749074465	1\\
16.922682720907	0.388459571640427   3\\
-8.80493044061983	1.53633185203394         3.6\\
16.4939570946887	1.52600447864613   3\\
17.9241467458037	1.20419497507464   3\\
-23.04895664004	1.63976967797421	1\\
-17.7696745813826	1.18979757417922	1\\
23.166335934286	-0.582644707088908   3\\
-2.27230023287371	0.71458874712357    5\\
-0.287675662479843	-0.711813345667602    5\\
-0.360363973361282	-0.197989109684111    5\\
27.331540620025	0.757557933311197   3\\
13.1827156854743	2.45728397123268   3\\
14.3027949751858	0.7553324891235   3\\
-15.9994364607955	1.08418419031775	1\\
3.37260053176413	1.31896457301187    5\\
5.5439344918532	-0.19210767319462    5\\
-5.05968541613884	-0.335334018370392    5\\
18.0339424932215	2.32661820778628   3\\
8.60700215498597	1.1120486202389     2.3\\
-7.26158307013133	2.01704240998788    5\\
-16.6084518187844	1.98483496094024	1\\
2.47546508347088	-0.952371623973633    5\\
-16.2460980696874	1.86558534829802	1\\
-8.86791320426481	3.36046530970861	1\\
-20.16845812789	1.48182070413641	1\\
-22.3692119170163	3.52921770601893	1\\
-0.698131614729261	1.47799550815875    5\\
-1.33272594858178	-0.362846898895077    5\\
-17.5905176335567	2.08671647451577	1\\
-14.8453168268134	2.02048846239236	1\\
-12.4930155399117	1.25966754217746	1\\
15.568653564821	1.19830887297365   3\\
0.324827405578217	0.100716870566698    5\\
4.70763340065599	-0.261582709569126    5\\
-8.9790121341234	1.2592543853217         3.6\\
-2.43227676137703	-1.00529674280893    5\\
-3.42556792212039	0.276213723448114    5\\
12.9191146184821	1.91661960198528   3\\
-10.3397290342613	2.57818687141448	1\\
17.1953199307291	0.3874005091279   3\\
-9.78350537989637	1.46400178823938	1\\
-1.39637905539436	-0.225707022587075    5\\
-2.04607986438169	-0.215557883234501    5\\
26.7269907349407	1.0647149207353   3\\
-18.3641829752053	2.9967222947365	1\\
9.35861088149241	0.184108569870373    5\\
0.365726348718405	-1.1978742231361    5\\
22.1650585189586	1.06285225424231   3\\
16.6361575816641	0.475635932289157   3\\
22.4456245787494	1.09664925459472   3\\
2.52960646636026	-0.388887344581937    5\\
-17.4351872477026	2.73216253164856	1\\
16.4884784363945	2.07652272592363   3\\
-1.2072541057224	0.374308590663764    5\\
-0.3456293494986	-0.145244366246396    5\\
16.129887365198	0.0136479432515966   3\\
-19.3957506309308	1.13951918220314	1\\
-24.0369772023756	1.06459681474729	1\\
17.9361581501116	1.71637095613658   3\\
21.039049657626	0.351829545338599   3\\
-24.3538721392332	3.26132825906002	1\\
-18.0242540116346	0.190996792906936	1\\
8.9568917100496	0.738627180656787     2.3\\
0.294461030552158	-0.427908025119613    5\\
-18.6920665750535	2.76421624144904	1\\
16.0513510574982	1.16630018930676   3\\
-15.6202347031393	0.783212942891168	1\\
0.0953110219656994	-1.70005083538807    5\\
-18.1616523223102	2.44001838489488	1\\
0.74440013857161	0.125261145615376    5\\
19.3366085813951	-0.124950052767057   3\\
-16.9543559749638	1.7393735908863	1\\
-1.08475494287602	1.26553147567952    5\\
-19.9074165473818	1.75178430304521	1\\
-13.2038863617528	2.5053026473487	1\\
-12.7609370672405	0.90319601193845	1\\
-12.8565793539235	2.63698792513635	1\\
-12.5603979879597	1.80454105330398	1\\
-21.1158322955079	1.53149922906964	1\\
-16.4299803605004	2.71289957054009	1\\
15.5510029351612	2.13430535609685   3\\
-5.78790560912164	0.940689180084404    5\\
-12.4079712589374	2.59085067590129	1\\
0.741995569408901	1.35097729821223    5\\
-14.3245048265129	3.36997359881242	1\\
-8.8245412536474	0.644504466138986    5\\
18.2027621888613	2.15003988355839   3\\
16.5285380101286	0.656943261440391   3\\
0.0456950789312266	-1.57757798407271    5\\
-2.04961962072843	1.10318927672836    5\\
-21.7081763745371	0.84136398202266	1\\
-18.6581535233444	3.10952054457128	1\\
0.8621464551415	1.16432658152399    5\\
18.3125820819632	0.706939370903543   3\\
-15.9421314045719	0.590360840174064	1\\
-1.07139662572615	1.02027722819588    5\\
-15.8023465839867	1.95446643338996	1\\
-14.794783132387	2.38082430885078	1\\
-3.13764780321976	0.965134649490343    5\\
0.144893956518119	0.275211885948681    5\\
1.61730418631774	-1.2610760554924    5\\
16.2402719973496	0.604288317188662   3\\
-19.0114324045954	2.41811473613716	1\\
15.9130048249204	1.52833459773368   3\\
-0.403811873747295	1.06905385667546    5\\
-19.337960273146	1.89886097994714	1\\
20.5611472972269	0.217867377187204   3\\
17.8022392993506	0.699417957078192   3\\
15.7789368617815	0.630457904166283   3\\
-19.7713156159225	2.36150066513532	1\\
3.27068173235815	1.99413329272505    5\\
2.34080673117722	-1.22547354678251    5\\
1.44082177941281	0.427748841757203    5\\
5.27461449560866	0.665681917910806    5\\
-13.7175163130788	1.82584031331161	1\\
20.7882273234874	1.09037313101011   3\\
-17.0422211463866	0.87364680950016	1\\
18.3774073265841	1.31004470392943   3\\
-0.382707098787698	0.0854458716008148    5\\
-16.2420973542429	1.19496295885525	1\\
1.21549675330816	1.76344114924328    5\\
-19.4104519538838	4.14543345619423	1\\
-2.71521332697903	-0.102543354291735    5\\
-0.379766638050199	-0.855147739074738    5\\
15.8316594558143	0.272073501902906   3\\
1.04457360506462	0.125183188338292    5\\
0.962720049706309	-0.0800567896487775    5\\
-25.8489569715911	2.79875211428775	1\\
-18.3812639685904	1.7586807693835	1\\
-0.870901740780474	0.499120192804166    5\\
15.0945414408655	1.55191224157982   3\\
8.69371609590603	2.81899678836542     2.3\\
13.8211720815551	1.85736428078591   3\\
-6.51317518254709	-0.942033261883492    5\\
-5.36453330437023	0.066848381796742    5\\
6.10534226133907	-0.955176071431222    5\\
-19.1687777013071	1.62710609771755	1\\
0.195999663312141	-0.212289209885246    5\\
12.6152921700468	2.06837899542438   3\\
18.2950063728581	-0.430654941012518   3\\
19.4028271244944	0.927556874797489   3\\
-8.4334129488841	2.75632512624446         3.6\\
-15.8253944662369	1.68223737191408	1\\
-16.8370940110564	0.854853059585934	1\\
-6.1738763898852	0.0526790421663295    5\\
-0.635265042724267	0.300794393628996    5\\
13.641355203277	1.00873810257832   3\\
19.2574938476782	0.683866176061944   3\\
-22.2539461423157	3.01546795186151	1\\
-12.3660000572511	1.46661867594727	1\\
-17.8536260489373	2.64910637072127	1\\
-20.0068609608008	1.44615028101763	1\\
-18.6640354368031	3.64209241242855	1\\
16.5363360864435	-0.273283420489044   3\\
15.9605422002912	0.14748529314626   3\\
2.04476138674395	-0.417954355091849    5\\
15.610420766984	1.63589437139189   3\\
-6.03344592998271	-0.0635857588602965    5\\
2.67072088228364	-0.478585218055148    5\\
-15.7485879890396	0.736956534802068	1\\
2.38507038121179	-0.582062748571733    5\\
19.2161825075766	0.81836688418762   3\\
-20.1060973892671	2.19092272494124	1\\
18.5807489052253	2.36815805625131   3\\
-21.9004488771623	1.98921828484528	1\\
0.708044040305444	-0.435635586393627    5\\
-11.2549783018629	2.31122336690269	1\\
24.5690905173359	2.10040148839898   3\\
-2.57823083051533	1.40005582001389    5\\
9.56559457487811	0.957885503999155     2.3\\
-21.7961066930105	2.51678067987402	1\\
-19.6878591171879	2.3580626609474	1\\
-17.1171250701597	0.793253876026244	1\\
-21.7472411913366	2.31476596263571	1\\
-0.293399002597656	0.432826955761419    5\\
15.2701354453626	0.44354746509279   3\\
23.8133662612195	-0.191602757600895   3\\
25.6653274924032	2.16872531282788   3\\
-14.0607500225465	2.92555277320955	1\\
-22.3842291018574	2.06782043743519	1\\
2.95165487208015	1.88212968797264    5\\
15.5862457224728	0.184979417130069   3\\
-15.7266731682233	2.63279975541401	1\\
-2.91370012284536	0.093849066181965    5\\
};

\end{axis}

\end{tikzpicture}%
			\end{subfigure}
		\end{figure}
	\end{frame}

	\begin{frame}{Classification}
		\begin{figure}
			\centering
			\begin{subfigure}{.7\linewidth}
				% This file was created by matlab2tikz.
%
%The latest updates can be retrieved from
%  http://www.mathworks.com/matlabcentral/fileexchange/22022-matlab2tikz-matlab2tikz
%where you can also make suggestions and rate matlab2tikz.
%
\begin{tikzpicture}[scale = .7]

\begin{axis}[%
width=4.521in,
height=0.714in,
at={(0.758in,1.907in)},
scale only axis,
colormap={mymap}{[1pt] rgb(0pt)=(0,0.4470,0.7410); rgb(1pt)=(0.8500,0.3250,0.0980); rgb(2pt)=(0.9290,0.6940,0.1250); rgb(3pt)=(0.4940,0.1840,0.5560); rgb(4pt)=(0.4660,0.6740,0.1880); rgb(5pt)=(0.3010,0.7450,0.9330); rgb(6pt)=(0.6350,0.0780,0.1840)},
xmin=-28,
xmax=29,
ymin=-3,
ymax=6,
axis background/.style={fill=white},
axis x line*=bottom,
axis y line*=left,
legend style={legend cell align=left, align=left, draw=white!15!black},
]
\addplot[scatter, only marks, mark=*, mark size=0.7906pt, scatter src=explicit, scatter/use mapped color={mark options={}, draw=mapped color, fill=mapped color}] table[row sep=crcr, meta=color]{%
x	y	color\\
-2.12697932307494	0.734153401779101    5\\
0.122927902969405	0.343326281739181    5\\
2.63216147756917	-0.254327350450132    5\\
-1.75157286210449	0.158295635207253    5\\
4.27441349221991	0.846805181239452    5\\
-13.0390647515914	2.58385382612626	1\\
-12.8962191835208	1.52140009783339	1\\
8.88868267627099	1.16652701210223     2.3\\
17.6121652846523	0.0284510755368321   3\\
-5.32237756585857	0.655097866112154    5\\
-0.612345419675211	1.28552435339239    5\\
0.902331160383546	1.96452881953512    5\\
0.874869825424029	0.142284653299417    5\\
-22.1382789336631	0.832793190409463	1\\
0.586475495983049	-0.832525804626447    5\\
0.705041791029858	-0.515079395368846    5\\
-1.04450594719981	-0.147262435740308    5\\
16.3315637977649	0.477110256498039   3\\
18.018017152122	1.76404327515109   3\\
-19.3630024537047	2.01149032796427	1\\
17.4550912906385	0.319538988623299   3\\
-20.9896730083877	3.65594753327058	1\\
-1.18488286562443	-1.16507249030604    5\\
15.4184999256647	0.83152386335639   3\\
17.2987680991699	0.0502924317360712   3\\
-8.93380319123649	2.79179319828156         3.6\\
1.26705937469575	-0.867473260315087    5\\
-1.28438720527871	-0.713347054601585    5\\
9.10510417136045	1.83776595157136     2.3\\
-16.6847219126246	2.68883322594695	1\\
3.63186642821824	-0.814090659041176    5\\
18.5805298844266	0.832764559850249   3\\
-15.3759710202692	0.981067014407197	1\\
3.14711464411891	-0.322481950660631    5\\
-2.66614606350415	-0.81636677681243    5\\
13.7368400873437	-0.16353128552678   3\\
-21.4184855726264	1.31906368385877	1\\
7.12114418511638	0.43766322999098    5\\
19.507988751321	0.0437723624980891   3\\
-24.1664646761523	2.22569681559814	1\\
-16.9553906106059	1.35287449366543	1\\
2.51993009559619	-0.890071012287709    5\\
-8.150084059313	-0.326401140551159    5\\
17.1993711733966	1.26448446577924   3\\
-19.8703419557057	2.03730143411024	1\\
-1.52213604584959	0.474552418425034    5\\
16.3262811269147	0.953182799233613   3\\
-4.11811338337861	-0.194918408619307    5\\
-16.0532991785198	1.77656100238983	1\\
-21.147544483623	1.96171869287892	1\\
10.4173539499661	0.416229729477691   3\\
-20.4478206961817	1.64646946877365	1\\
-16.8413288519043	2.73928972359917	1\\
-19.5926898625094	3.26429398433028	1\\
-2.08567029493479	1.58862018685686    5\\
19.6301442962875	-0.574011910441037   3\\
18.5666799752281	1.98522946051986   3\\
14.7251490214037	1.13784605885349   3\\
13.2741394464552	-0.151069476579381   3\\
19.7549042027541	2.19333508225127   3\\
-18.9926741023293	1.87124347333019	1\\
17.4714641450038	0.597151143402294   3\\
-6.6079222964015	0.762651682985142    5\\
-17.5201101957292	2.20712888770216	1\\
-17.1132275615809	2.03969084863212	1\\
-15.9745545135678	1.57298185013394	1\\
-16.8927032415142	3.07138091918918	1\\
-14.3073317612216	3.04376566066148	1\\
17.5271489410172	0.973202923739207   3\\
1.91931444197215	-0.305233399040617    5\\
-18.9283621834201	0.905039250461469	1\\
19.7436446923	0.194417998993186   3\\
1.85100245803221	0.121798597267792    5\\
-20.6981526256077	2.0605043876971	1\\
-15.7070979848966	1.1416615059414	1\\
-14.33743596191	3.0377338917311	1\\
-22.7807585452835	1.29817383619715	1\\
18.6099159809055	0.683255765856225   3\\
23.2922016344413	1.99400560113624   3\\
-16.8097839579789	1.12389141168737	1\\
-19.8460760759784	2.00555005595308	1\\
-0.2042031101242	1.24619746275841    5\\
20.9582517207	0.614385562682709   3\\
-1.92189047030412	0.100019093319906    5\\
13.3940092228053	-0.125102704119033   3\\
-17.5137187641193	1.96083612560374	1\\
-17.1222284681888	3.32890683803807	1\\
0.142448697988598	0.093342814247212    5\\
16.354146626165	0.728147874655163   3\\
3.98605221014808	1.14956883942006    5\\
18.8448793560701	0.0229030510036594   3\\
-8.99755060220678	1.52015401747728         3.6\\
2.63553280705291	-0.939661932730858    5\\
1.12351453006449	1.43470248260702    5\\
-17.7869324026094	1.44232990683879	1\\
11.8475614672186	0.433077066828136   3\\
9.22914549273269	0.979552930295771     2.3\\
22.8443184050309	1.09603705904798   3\\
0.872470481316697	-0.493244756439886    5\\
14.9278800277285	1.49094705485706   3\\
-20.5218373633359	2.03704146699485	1\\
2.35548551922442	-1.02413757243261    5\\
-19.1168508947509	2.86616596765723	1\\
-11.6504987722591	0.355011233577846	1\\
18.0407586248709	0.197226416112856   3\\
13.0502232670168	-0.000678917350893871   3\\
-5.56068841260192	0.651666514866178    5\\
-18.5661635737835	1.82036107176167	1\\
-14.5648684024875	1.7472471368815	1\\
23.2384793311994	1.05697993624236   3\\
3.64427636120491	0.72558231243753    5\\
21.1780345912468	1.28407465671138   3\\
18.8938314621014	1.65124207113436   3\\
-20.9558009809282	2.29438182064043	1\\
9.05754612856061	2.11573598821182     2.3\\
14.4335219567506	0.12861717145351   3\\
-13.2995965679574	1.18603506901456	1\\
-15.355388414595	2.28552129200204	1\\
-20.4427160947005	2.36507745190789	1\\
3.32687032386843	0.143938803413768    5\\
2.52958968202736	-0.359283496158863    5\\
3.57433605400171	0.663738533205764    5\\
15.9738066140656	1.65626609980807   3\\
-20.6940885228748	1.97046721642786	1\\
-20.8993955148484	1.92066864410809	1\\
8.09721024622322	-0.611171648587698    5\\
23.6534099624505	0.0414558355825206   3\\
-2.16803202677901	-0.86138337785408    5\\
21.441677428739	0.822888233599663   3\\
15.261495878196	0.93323024809634   3\\
-5.51602927846288	-1.46425105689166    5\\
18.2101373700008	1.87174033027409   3\\
18.155509773026	0.17915689348163   3\\
6.25890459965704	0.0773331675839121    5\\
-14.1571379667395	0.708256765225689	1\\
6.7601413473542	2.28735033003258   3\\
19.3944035440371	1.26683759064637   3\\
20.9888471609451	2.11312739610155   3\\
2.21156293927144	1.26359652704661    5\\
22.3910505659835	1.03963436158509   3\\
6.17113714949222	0.499581029724201    5\\
13.8465226735953	1.63508063545026   3\\
5.01094534362827	-1.94768754936625    5\\
15.3103495661314	0.617242628527914   3\\
4.17369397015448	-0.199454091608638    5\\
24.8406301916814	-0.0299059505607522   3\\
-16.1803627926534	2.81859806719155	1\\
14.3433093866462	-0.0327054928981054   3\\
-17.4792398563394	1.76692350986799	1\\
-9.05197724640506	1.82018628882538         3.6\\
-15.6508742273441	2.12861025495932	1\\
-6.87392325124233	-0.726047041941697    5\\
-0.961470235928968	0.659302657319334    5\\
1.11561768286045	0.678727026324926    5\\
15.6884517450919	1.87308089864853   3\\
-2.20669665539958	-1.83393617244678    5\\
2.90107954030742	-0.0538228571054752    5\\
0.030775561275574	-0.505694414308935    5\\
-14.2450891916411	1.99676758220591	1\\
-19.3292014258364	1.31084860695427	1\\
-0.248887709450454	-1.03357022167571    5\\
13.7899160591559	2.58352660492817   3\\
-6.31240513420851	0.284799378628209    5\\
-18.5071565199403	1.83107733246302	1\\
18.0322650362449	0.44777490923522   3\\
21.5648747265458	0.585776172591518   3\\
23.9532440167402	1.28243731966051   3\\
4.97791519316952	-0.600916978844473    5\\
7.47636446135275	1.05380851512145    5\\
20.104162058142	0.542452380501339   3\\
20.976944981943	1.80350128837074   3\\
6.14444924191286	0.493848567147238    5\\
4.98053471281628	-0.992591915463532    5\\
-6.31493711080174	-0.528423660086781    5\\
23.9123435040969	1.22387335372774   3\\
-0.300796098229382	-1.70441880830099    5\\
12.88503160579	1.59239352154747    5\\
13.3363144449699	1.99474943299734   3\\
0.180636022170158	0.00222664574961195    5\\
-18.0221771013136	1.25226759125618	1\\
-3.416981264793	0.982464916325299    5\\
19.5715414584367	2.28646853126353   3\\
-0.426980860434325	-0.266386116143043    5\\
16.6949839151162	1.47068028426394   3\\
-2.29843043769072	0.0428034072020241    5\\
1.48022093786953	-0.0369553108551366    5\\
0.215090813586244	0.454357512649916    5\\
3.71189859955171	0.667925057663898    5\\
-23.2612528355385	2.582088452385	1\\
-4.79264380510488	-0.155934649338532    5\\
16.1909596000321	2.47573874718405   3\\
-9.53811106905465	1.21143129262326         3.6\\
-19.6648767538016	0.994808823993323	1\\
-5.12862759791042	-0.520330078272704    5\\
-15.0533521845265	0.490030490326571	1\\
12.6143887881425	1.7111534897066   3\\
14.7753853211326	1.0132557478699   3\\
-1.30048040907759	-0.431937706250186    5\\
3.23422507161324	-1.3025603140885    5\\
2.45114322778812	0.317476630131372    5\\
13.2582951164167	2.22271410151697   3\\
-8.84971845076545	2.17176324454394         3.6\\
-16.8180168857178	1.45576663204976	1\\
16.4534351894443	0.570465032066382   3\\
7.33986622212505	0.364469767025394    5\\
-17.7246315134906	1.82242875133601	1\\
17.3973815351121	1.34973281429677   3\\
-15.7921465704791	2.76054262979188	1\\
16.7916852067438	0.811718729850322   3\\
23.6142057966694	0.84199278525811   3\\
22.4577829723654	2.49206622756598   3\\
13.3224454612036	2.23287543910022   3\\
11.105584350006	0.620470176134218   3\\
-0.984604320733126	-1.1513711051391    5\\
-0.911459934255873	0.475121213330907    5\\
1.59722012298325	1.23913855668169    5\\
-12.8630622367322	2.05447012943979	1\\
-1.22324225553094	-0.0589341694579021    5\\
-17.1924404357685	2.99558385564624	1\\
-2.80518274783678	-0.00902135572686707    5\\
-17.6272401459748	1.69926327140825	1\\
-14.1891567475565	1.83604723406803	1\\
-1.18212477059333	0.183005682365753    5\\
13.8707508201733	1.15839106702986   3\\
6.26827588273483	0.225574782535944    5\\
-12.6720022098057	1.371159292207	1\\
-20.2405176553229	0.556046227232444	1\\
18.4098771545399	-0.188638341733553   3\\
-26.0375438316404	2.95267185755971	1\\
-18.220984946945	2.01807644198578	1\\
-17.314409185012	2.12798826781049	1\\
0.252695306713707	0.0212103384108863    5\\
-4.8296345070935	-0.217075267468112    5\\
-16.6625017814217	1.42620179366784	1\\
-18.8169715210963	2.29040632696061	1\\
0.596009852112165	1.02511255833282    5\\
-2.13709275859671	-0.889853564462552    5\\
15.4710071128111	0.610579682826467   3\\
-9.01792211391009	2.60235462937461         3.6\\
6.55587645031129	0.478265197642075    5\\
12.1623916761271	1.55254313646863   3\\
14.17342311664	0.933580523259207   3\\
-2.1816456927356	-0.0607979823126644    5\\
1.97527355758178	-0.186221009891955    5\\
-0.71197777236988	1.51547738153152    5\\
20.2409243606416	1.51760562017255   3\\
-9.30538967278889	1.82274702568521         3.6\\
15.4542607039897	1.68716899234405   3\\
-20.2544416695237	0.776970463551071	1\\
2.17164317097379	0.348417453483848    5\\
18.546042613234	1.13134433433068   3\\
-1.63377214861198	0.0233836114569938    5\\
-5.18880612934132	0.0735722801439829    5\\
20.272293735638	1.62950588032582   3\\
15.3137376955454	2.59547372884512   3\\
-11.5770161103664	2.78530405972726	1\\
-13.6578040906098	2.53764945924901	1\\
-22.2594718812781	1.45952736923485	1\\
-2.55311086439217	-1.89757348850271    5\\
-16.6147545232877	1.38623296795023	1\\
-11.9336108227027	0.850271133692976	1\\
19.10038941662	0.746267755092501   3\\
18.5140798912944	1.16196731602621   3\\
-14.0820026152571	2.46231536925236	1\\
17.0033969680974	0.0923861891419633   3\\
-13.7716433866066	1.66702728548603	1\\
-21.5465114288155	2.27393038216814	1\\
-19.266892542561	1.5063570186254	1\\
17.957293797529	1.20526329536001   3\\
0.855078802458656	-1.45519140817519    5\\
15.0913649928761	1.80037076338443   3\\
14.4283409108264	1.59556331296322   3\\
-18.3592790884494	1.39610074503546	1\\
-0.908852166537197	-0.179818610218128    5\\
1.38930662361828	-1.0126894039102    5\\
-20.8874869632382	3.5370459111741	1\\
19.8398389888065	0.473323900134359   3\\
-3.56518578628586	0.0694002705786866    5\\
-14.9472070463345	2.09648397420584	1\\
16.4166007335949	1.28367285196227   3\\
-0.903231390652451	-0.574647079937109    5\\
0.451360804117812	-0.445534987556561    5\\
3.09547678566735	-0.226532505853323    5\\
-18.6723732578719	1.45088340704962	1\\
18.4097167174584	1.25606409458656   3\\
18.0066547612642	0.673202883509209   3\\
21.2377358237767	0.996552217540614   3\\
-8.84599216092668	1.46455780243804         3.6\\
1.74237052222676	0.20474176535925    5\\
-14.2643656566544	2.0359986713898	1\\
13.9932823959732	2.2810748776201   3\\
18.1689305463369	-0.476035595381123   3\\
21.3094273166437	1.43598121181515   3\\
20.4371710628645	1.35885558856491   3\\
-17.9319893325714	1.67540933733375	1\\
-19.703206349711	1.81469320025894	1\\
0.789278993735694	-0.276628818102428    5\\
-1.88560086380028	-1.41052388252961    5\\
-12.3124098407243	1.80779432894724	1\\
-0.937783966196865	0.646631939551512    5\\
-10.8927388023256	2.37111067588025	1\\
-9.05820744933176	1.04164084227071	1\\
-19.256797375881	1.60506823682092	1\\
18.363715910536	-0.362376451908442   3\\
5.31405479380758	-0.652492053763823    5\\
19.1583659796918	1.31306296323931   3\\
-16.3258057443609	0.777287963460357	1\\
-19.7900590331145	2.41352568053095	1\\
-18.9170914580381	1.8718152143672	1\\
8.99536136719112	1.76274859833112     2.3\\
16.5642088765996	0.911737483580825   3\\
11.0343990909932	0.368501798509585    5\\
-0.28034321414595	0.391086743553475    5\\
-2.04916101500488	1.17082765129311    5\\
-21.5399446430118	1.82042502121844	1\\
-13.9949450541028	2.9352749074465	1\\
16.922682720907	0.388459571640427   3\\
-8.80493044061983	1.53633185203394         3.6\\
16.4939570946887	1.52600447864613   3\\
17.9241467458037	1.20419497507464   3\\
-23.04895664004	1.63976967797421	1\\
-17.7696745813826	1.18979757417922	1\\
23.166335934286	-0.582644707088908   3\\
-2.27230023287371	0.71458874712357    5\\
-0.287675662479843	-0.711813345667602    5\\
-0.360363973361282	-0.197989109684111    5\\
27.331540620025	0.757557933311197   3\\
13.1827156854743	2.45728397123268   3\\
14.3027949751858	0.7553324891235   3\\
-15.9994364607955	1.08418419031775	1\\
3.37260053176413	1.31896457301187    5\\
5.5439344918532	-0.19210767319462    5\\
-5.05968541613884	-0.335334018370392    5\\
18.0339424932215	2.32661820778628   3\\
8.60700215498597	1.1120486202389     2.3\\
-7.26158307013133	2.01704240998788    5\\
-16.6084518187844	1.98483496094024	1\\
2.47546508347088	-0.952371623973633    5\\
-16.2460980696874	1.86558534829802	1\\
-8.86791320426481	3.36046530970861	1\\
-20.16845812789	1.48182070413641	1\\
-22.3692119170163	3.52921770601893	1\\
-0.698131614729261	1.47799550815875    5\\
-1.33272594858178	-0.362846898895077    5\\
-17.5905176335567	2.08671647451577	1\\
-14.8453168268134	2.02048846239236	1\\
-12.4930155399117	1.25966754217746	1\\
15.568653564821	1.19830887297365   3\\
0.324827405578217	0.100716870566698    5\\
4.70763340065599	-0.261582709569126    5\\
-8.9790121341234	1.2592543853217         3.6\\
-2.43227676137703	-1.00529674280893    5\\
-3.42556792212039	0.276213723448114    5\\
12.9191146184821	1.91661960198528   3\\
-10.3397290342613	2.57818687141448	1\\
17.1953199307291	0.3874005091279   3\\
-9.78350537989637	1.46400178823938	1\\
-1.39637905539436	-0.225707022587075    5\\
-2.04607986438169	-0.215557883234501    5\\
26.7269907349407	1.0647149207353   3\\
-18.3641829752053	2.9967222947365	1\\
9.35861088149241	0.184108569870373    5\\
0.365726348718405	-1.1978742231361    5\\
22.1650585189586	1.06285225424231   3\\
16.6361575816641	0.475635932289157   3\\
22.4456245787494	1.09664925459472   3\\
2.52960646636026	-0.388887344581937    5\\
-17.4351872477026	2.73216253164856	1\\
16.4884784363945	2.07652272592363   3\\
-1.2072541057224	0.374308590663764    5\\
-0.3456293494986	-0.145244366246396    5\\
16.129887365198	0.0136479432515966   3\\
-19.3957506309308	1.13951918220314	1\\
-24.0369772023756	1.06459681474729	1\\
17.9361581501116	1.71637095613658   3\\
21.039049657626	0.351829545338599   3\\
-24.3538721392332	3.26132825906002	1\\
-18.0242540116346	0.190996792906936	1\\
8.9568917100496	0.738627180656787     2.3\\
0.294461030552158	-0.427908025119613    5\\
-18.6920665750535	2.76421624144904	1\\
16.0513510574982	1.16630018930676   3\\
-15.6202347031393	0.783212942891168	1\\
0.0953110219656994	-1.70005083538807    5\\
-18.1616523223102	2.44001838489488	1\\
0.74440013857161	0.125261145615376    5\\
19.3366085813951	-0.124950052767057   3\\
-16.9543559749638	1.7393735908863	1\\
-1.08475494287602	1.26553147567952    5\\
-19.9074165473818	1.75178430304521	1\\
-13.2038863617528	2.5053026473487	1\\
-12.7609370672405	0.90319601193845	1\\
-12.8565793539235	2.63698792513635	1\\
-12.5603979879597	1.80454105330398	1\\
-21.1158322955079	1.53149922906964	1\\
-16.4299803605004	2.71289957054009	1\\
15.5510029351612	2.13430535609685   3\\
-5.78790560912164	0.940689180084404    5\\
-12.4079712589374	2.59085067590129	1\\
0.741995569408901	1.35097729821223    5\\
-14.3245048265129	3.36997359881242	1\\
-8.8245412536474	0.644504466138986    5\\
18.2027621888613	2.15003988355839   3\\
16.5285380101286	0.656943261440391   3\\
0.0456950789312266	-1.57757798407271    5\\
-2.04961962072843	1.10318927672836    5\\
-21.7081763745371	0.84136398202266	1\\
-18.6581535233444	3.10952054457128	1\\
0.8621464551415	1.16432658152399    5\\
18.3125820819632	0.706939370903543   3\\
-15.9421314045719	0.590360840174064	1\\
-1.07139662572615	1.02027722819588    5\\
-15.8023465839867	1.95446643338996	1\\
-14.794783132387	2.38082430885078	1\\
-3.13764780321976	0.965134649490343    5\\
0.144893956518119	0.275211885948681    5\\
1.61730418631774	-1.2610760554924    5\\
16.2402719973496	0.604288317188662   3\\
-19.0114324045954	2.41811473613716	1\\
15.9130048249204	1.52833459773368   3\\
-0.403811873747295	1.06905385667546    5\\
-19.337960273146	1.89886097994714	1\\
20.5611472972269	0.217867377187204   3\\
17.8022392993506	0.699417957078192   3\\
15.7789368617815	0.630457904166283   3\\
-19.7713156159225	2.36150066513532	1\\
3.27068173235815	1.99413329272505    5\\
2.34080673117722	-1.22547354678251    5\\
1.44082177941281	0.427748841757203    5\\
5.27461449560866	0.665681917910806    5\\
-13.7175163130788	1.82584031331161	1\\
20.7882273234874	1.09037313101011   3\\
-17.0422211463866	0.87364680950016	1\\
18.3774073265841	1.31004470392943   3\\
-0.382707098787698	0.0854458716008148    5\\
-16.2420973542429	1.19496295885525	1\\
1.21549675330816	1.76344114924328    5\\
-19.4104519538838	4.14543345619423	1\\
-2.71521332697903	-0.102543354291735    5\\
-0.379766638050199	-0.855147739074738    5\\
15.8316594558143	0.272073501902906   3\\
1.04457360506462	0.125183188338292    5\\
0.962720049706309	-0.0800567896487775    5\\
-25.8489569715911	2.79875211428775	1\\
-18.3812639685904	1.7586807693835	1\\
-0.870901740780474	0.499120192804166    5\\
15.0945414408655	1.55191224157982   3\\
8.69371609590603	2.81899678836542     2.3\\
13.8211720815551	1.85736428078591   3\\
-6.51317518254709	-0.942033261883492    5\\
-5.36453330437023	0.066848381796742    5\\
6.10534226133907	-0.955176071431222    5\\
-19.1687777013071	1.62710609771755	1\\
0.195999663312141	-0.212289209885246    5\\
12.6152921700468	2.06837899542438   3\\
18.2950063728581	-0.430654941012518   3\\
19.4028271244944	0.927556874797489   3\\
-8.4334129488841	2.75632512624446         3.6\\
-15.8253944662369	1.68223737191408	1\\
-16.8370940110564	0.854853059585934	1\\
-6.1738763898852	0.0526790421663295    5\\
-0.635265042724267	0.300794393628996    5\\
13.641355203277	1.00873810257832   3\\
19.2574938476782	0.683866176061944   3\\
-22.2539461423157	3.01546795186151	1\\
-12.3660000572511	1.46661867594727	1\\
-17.8536260489373	2.64910637072127	1\\
-20.0068609608008	1.44615028101763	1\\
-18.6640354368031	3.64209241242855	1\\
16.5363360864435	-0.273283420489044   3\\
15.9605422002912	0.14748529314626   3\\
2.04476138674395	-0.417954355091849    5\\
15.610420766984	1.63589437139189   3\\
-6.03344592998271	-0.0635857588602965    5\\
2.67072088228364	-0.478585218055148    5\\
-15.7485879890396	0.736956534802068	1\\
2.38507038121179	-0.582062748571733    5\\
19.2161825075766	0.81836688418762   3\\
-20.1060973892671	2.19092272494124	1\\
18.5807489052253	2.36815805625131   3\\
-21.9004488771623	1.98921828484528	1\\
0.708044040305444	-0.435635586393627    5\\
-11.2549783018629	2.31122336690269	1\\
24.5690905173359	2.10040148839898   3\\
-2.57823083051533	1.40005582001389    5\\
9.56559457487811	0.957885503999155     2.3\\
-21.7961066930105	2.51678067987402	1\\
-19.6878591171879	2.3580626609474	1\\
-17.1171250701597	0.793253876026244	1\\
-21.7472411913366	2.31476596263571	1\\
-0.293399002597656	0.432826955761419    5\\
15.2701354453626	0.44354746509279   3\\
23.8133662612195	-0.191602757600895   3\\
25.6653274924032	2.16872531282788   3\\
-14.0607500225465	2.92555277320955	1\\
-22.3842291018574	2.06782043743519	1\\
2.95165487208015	1.88212968797264    5\\
15.5862457224728	0.184979417130069   3\\
-15.7266731682233	2.63279975541401	1\\
-2.91370012284536	0.093849066181965    5\\
};

\end{axis}

\end{tikzpicture}%
			\end{subfigure}
			\pause
			\visible<2->{
			\begin{subfigure}{.7\linewidth}
				\ \\
			\end{subfigure}
			\begin{subfigure}{.7\linewidth}
				% This file was created by matlab2tikz.
%
%The latest updates can be retrieved from
%  http://www.mathworks.com/matlabcentral/fileexchange/22022-matlab2tikz-matlab2tikz
%where you can also make suggestions and rate matlab2tikz.
%
\begin{tikzpicture}[scale = .7]

\begin{axis}[%
width=4.521in,
height=0.904in,
at={(0.758in,1.812in)},
scale only axis,
axis on top,
xmin=0.5,
xmax=1000.5,
xtick={53.1315789473684,140.850877192982,228.570175438596,316.289473684211,404.008771929825,491.728070175439,579.447368421053,667.166666666667,754.885964912281,842.605263157895,930.324561403509},
xticklabels={{-25},{-20},{-15},{-10},{-5},{0},{5},{10},{15},{20},{25}},
ymin=0.5,
ymax=200.5,
ytick={22.7222222222222,67.1666666666667,111.611111111111,156.055555555556,200.5},
yticklabels={{-2},{0},{2},{4},{6}},
axis x line*=bottom,
axis y line*=left,
axis background/.style={fill=white},
legend style={legend cell align=left, align=left, draw=white!15!black}
]
\addplot [forget plot] graphics [xmin=0.0, xmax=1000.5, ymin=0.0, ymax=200.5] {../kmeansMATLAB/GMMdataclassreg-2.png};
\end{axis}

\end{tikzpicture}%
			\end{subfigure}
			}
		\end{figure}
	\end{frame}

	\begin{frame}{$K$-means Algorithm}
		Start with data \( \mathcal{X} = \{\bm x^{1},\ldots,\bm x^N\} \) and $K$ starting `means', \( \{\bm m_1,\ldots,\bm m_K\}. \)
		
		\ \\
		
		\alert<2>{\textbf{Assignment:}} For each data point, \(\bm x^{n}\), set 
			\(\hat{k}_n = \argmin_k d(\bm x^{n},\bm m_k)\). Set \alert<3>{\( \rho_i^n = \delta_i^{\hat{k}_n} \) (Hard responsibility)}.
		
		\ \\
		
		\alert<4>{\textbf{Update:}} Let \(N_k = \sum_{n=1}^{N} \rho^n_k\) and 
			\[\bm m_k^{new} = \frac{\sum_{n=1}^{N} \rho_k^n \bm x^{(n)}}{N_k}.\]
		
		If \(d(\bm m_k ,\bm m_k^{new})\) is small for every \( k \), stop.  Otherwise set \(\bm m_k = \bm m_k^{new}\) and \alert<5>{repeat}. See \citet{MacKay2002} for more discussion.
		
	\end{frame}

	\begin{frame}{$K$-means Example}
		\begin{center}
		\only<1>{\begin{figure}\centering\input{../kmeansMATLAB/kmeans_slides/Kmeans5_1}
		\end{figure}}
	\mode<beamer>{
		\only<2>{\begin{figure}\input{../kmeansMATLAB/kmeans_slides/Kmeans5_2}
		\end{figure}}
		\only<3>{\begin{figure}% This file was created by matlab2tikz.
%
\begin{tikzpicture}[%
scale=.9
]

\begin{axis}[%
width=4.521in,
height=0.714in,
at={(0.758in,1.907in)},
scale only axis,
colormap={mymap}{[1pt] rgb(0pt)=(0,0.447,0.741); rgb(1pt)=(0.466,0.674,0.188); rgb(2pt)=(0.635,0.078,0.184); rgb(3pt)=(0.929,0.694,0.125); rgb(4pt)=(0.494,0.184,0.556)},
xmin=-28,
xmax=29,
ymin=-3,
ymax=6,
axis background/.style={fill=white},
axis x line*=bottom,
axis y line*=left
]
\addplot[scatter, only marks, mark=*, mark size=0.7906pt, scatter src=explicit, scatter/use mapped color={mark options={}, draw=mapped color, fill=mapped color}] table[row sep=crcr, meta=color]{%
x	y	color\\
-2.127	0.73415	4\\
0.12293	0.34333	3\\
2.6322	-0.25433	3\\
-1.7516	0.1583	3\\
4.2744	0.84681	4\\
-13.039	2.5839	5\\
-12.896	1.5214	5\\
8.8887	1.1665	4\\
17.612	0.028451	3\\
-5.3224	0.6551	4\\
-0.61235	1.2855	4\\
0.90233	1.9645	5\\
0.87487	0.14228	3\\
-22.138	0.83279	4\\
0.58648	-0.83253	2\\
0.70504	-0.51508	2\\
-1.0445	-0.14726	3\\
16.332	0.47711	3\\
18.018	1.764	5\\
-19.363	2.0115	5\\
17.455	0.31954	3\\
-20.99	3.6559	5\\
-1.1849	-1.1651	2\\
15.418	0.83152	4\\
17.299	0.050292	3\\
-8.9338	2.7918	5\\
1.2671	-0.86747	2\\
-1.2844	-0.71335	2\\
9.1051	1.8378	5\\
-16.685	2.6888	5\\
3.6319	-0.81409	2\\
18.581	0.83276	4\\
-15.376	0.98107	4\\
3.1471	-0.32248	3\\
-2.6661	-0.81637	2\\
13.737	-0.16353	3\\
-21.418	1.3191	4\\
7.1211	0.43766	3\\
19.508	0.043772	3\\
-24.166	2.2257	5\\
-16.955	1.3529	4\\
2.5199	-0.89007	2\\
-8.1501	-0.3264	3\\
17.199	1.2645	4\\
-19.87	2.0373	5\\
-1.5221	0.47455	3\\
16.326	0.95318	4\\
-4.1181	-0.19492	3\\
-16.053	1.7766	5\\
-21.148	1.9617	5\\
10.417	0.41623	3\\
-20.448	1.6465	5\\
-16.841	2.7393	5\\
-19.593	3.2643	5\\
-2.0857	1.5886	5\\
19.63	-0.57401	2\\
18.567	1.9852	5\\
14.725	1.1378	4\\
13.274	-0.15107	3\\
19.755	2.1933	5\\
-18.993	1.8712	5\\
17.471	0.59715	4\\
-6.6079	0.76265	4\\
-17.52	2.2071	5\\
-17.113	2.0397	5\\
-15.975	1.573	5\\
-16.893	3.0714	5\\
-14.307	3.0438	5\\
17.527	0.9732	4\\
1.9193	-0.30523	3\\
-18.928	0.90504	4\\
19.744	0.19442	3\\
1.851	0.1218	3\\
-20.698	2.0605	5\\
-15.707	1.1417	4\\
-14.337	3.0377	5\\
-22.781	1.2982	4\\
18.61	0.68326	4\\
23.292	1.994	5\\
-16.81	1.1239	4\\
-19.846	2.0056	5\\
-0.2042	1.2462	4\\
20.958	0.61439	4\\
-1.9219	0.10002	3\\
13.394	-0.1251	3\\
-17.514	1.9608	5\\
-17.122	3.3289	5\\
0.14245	0.093343	3\\
16.354	0.72815	4\\
3.9861	1.1496	4\\
18.845	0.022903	3\\
-8.9976	1.5202	5\\
2.6355	-0.93966	2\\
1.1235	1.4347	4\\
-17.787	1.4423	4\\
11.848	0.43308	3\\
9.2291	0.97955	4\\
22.844	1.096	4\\
0.87247	-0.49324	3\\
14.928	1.4909	4\\
-20.522	2.037	5\\
2.3555	-1.0241	2\\
-19.117	2.8662	5\\
-11.65	0.35501	3\\
18.041	0.19723	3\\
13.05	-0.00067892	3\\
-5.5607	0.65167	4\\
-18.566	1.8204	5\\
-14.565	1.7472	5\\
23.238	1.057	4\\
3.6443	0.72558	4\\
21.178	1.2841	4\\
18.894	1.6512	5\\
-20.956	2.2944	5\\
9.0575	2.1157	5\\
14.434	0.12862	3\\
-13.3	1.186	4\\
-15.355	2.2855	5\\
-20.443	2.3651	5\\
3.3269	0.14394	3\\
2.5296	-0.35928	3\\
3.5743	0.66374	4\\
15.974	1.6563	5\\
-20.694	1.9705	5\\
-20.899	1.9207	5\\
8.0972	-0.61117	2\\
23.653	0.041456	3\\
-2.168	-0.86138	2\\
21.442	0.82289	4\\
15.261	0.93323	4\\
-5.516	-1.4643	2\\
18.21	1.8717	5\\
18.156	0.17916	3\\
6.2589	0.077333	3\\
-14.157	0.70826	4\\
6.7601	2.2874	5\\
19.394	1.2668	4\\
20.989	2.1131	5\\
2.2116	1.2636	4\\
22.391	1.0396	4\\
6.1711	0.49958	3\\
13.847	1.6351	5\\
5.0109	-1.9477	1\\
15.31	0.61724	4\\
4.1737	-0.19945	3\\
24.841	-0.029906	3\\
-16.18	2.8186	5\\
14.343	-0.032705	3\\
-17.479	1.7669	5\\
-9.052	1.8202	5\\
-15.651	2.1286	5\\
-6.8739	-0.72605	2\\
-0.96147	0.6593	4\\
1.1156	0.67873	4\\
15.688	1.8731	5\\
-2.2067	-1.8339	1\\
2.9011	-0.053823	3\\
0.030776	-0.50569	2\\
-14.245	1.9968	5\\
-19.329	1.3108	4\\
-0.24889	-1.0336	2\\
13.79	2.5835	5\\
-6.3124	0.2848	3\\
-18.507	1.8311	5\\
18.032	0.44777	3\\
21.565	0.58578	4\\
23.953	1.2824	4\\
4.9779	-0.60092	2\\
7.4764	1.0538	4\\
20.104	0.54245	4\\
20.977	1.8035	5\\
6.1444	0.49385	3\\
4.9805	-0.99259	2\\
-6.3149	-0.52842	2\\
23.912	1.2239	4\\
-0.3008	-1.7044	1\\
12.885	1.5924	5\\
13.336	1.9947	5\\
0.18064	0.0022266	3\\
-18.022	1.2523	4\\
-3.417	0.98246	4\\
19.572	2.2865	5\\
-0.42698	-0.26639	3\\
16.695	1.4707	4\\
-2.2984	0.042803	3\\
1.4802	-0.036955	3\\
0.21509	0.45436	3\\
3.7119	0.66793	4\\
-23.261	2.5821	5\\
-4.7926	-0.15593	3\\
16.191	2.4757	5\\
-9.5381	1.2114	4\\
-19.665	0.99481	4\\
-5.1286	-0.52033	2\\
-15.053	0.49003	3\\
12.614	1.7112	5\\
14.775	1.0133	4\\
-1.3005	-0.43194	3\\
3.2342	-1.3026	2\\
2.4511	0.31748	3\\
13.258	2.2227	5\\
-8.8497	2.1718	5\\
-16.818	1.4558	4\\
16.453	0.57047	4\\
7.3399	0.36447	3\\
-17.725	1.8224	5\\
17.397	1.3497	4\\
-15.792	2.7605	5\\
16.792	0.81172	4\\
23.614	0.84199	4\\
22.458	2.4921	5\\
13.322	2.2329	5\\
11.106	0.62047	4\\
-0.9846	-1.1514	2\\
-0.91146	0.47512	3\\
1.5972	1.2391	4\\
-12.863	2.0545	5\\
-1.2232	-0.058934	3\\
-17.192	2.9956	5\\
-2.8052	-0.0090214	3\\
-17.627	1.6993	5\\
-14.189	1.836	5\\
-1.1821	0.18301	3\\
13.871	1.1584	4\\
6.2683	0.22557	3\\
-12.672	1.3712	4\\
-20.241	0.55605	4\\
18.41	-0.18864	3\\
-26.038	2.9527	5\\
-18.221	2.0181	5\\
-17.314	2.128	5\\
0.2527	0.02121	3\\
-4.8296	-0.21708	3\\
-16.663	1.4262	4\\
-18.817	2.2904	5\\
0.59601	1.0251	4\\
-2.1371	-0.88985	2\\
15.471	0.61058	4\\
-9.0179	2.6024	5\\
6.5559	0.47827	3\\
12.162	1.5525	5\\
14.173	0.93358	4\\
-2.1816	-0.060798	3\\
1.9753	-0.18622	3\\
-0.71198	1.5155	5\\
20.241	1.5176	5\\
-9.3054	1.8227	5\\
15.454	1.6872	5\\
-20.254	0.77697	4\\
2.1716	0.34842	3\\
18.546	1.1313	4\\
-1.6338	0.023384	3\\
-5.1888	0.073572	3\\
20.272	1.6295	5\\
15.314	2.5955	5\\
-11.577	2.7853	5\\
-13.658	2.5376	5\\
-22.259	1.4595	4\\
-2.5531	-1.8976	1\\
-16.615	1.3862	4\\
-11.934	0.85027	4\\
19.1	0.74627	4\\
18.514	1.162	4\\
-14.082	2.4623	5\\
17.003	0.092386	3\\
-13.772	1.667	5\\
-21.547	2.2739	5\\
-19.267	1.5064	5\\
17.957	1.2053	4\\
0.85508	-1.4552	2\\
15.091	1.8004	5\\
14.428	1.5956	5\\
-18.359	1.3961	4\\
-0.90885	-0.17982	3\\
1.3893	-1.0127	2\\
-20.887	3.537	5\\
19.84	0.47332	3\\
-3.5652	0.0694	3\\
-14.947	2.0965	5\\
16.417	1.2837	4\\
-0.90323	-0.57465	2\\
0.45136	-0.44553	3\\
3.0955	-0.22653	3\\
-18.672	1.4509	4\\
18.41	1.2561	4\\
18.007	0.6732	4\\
21.238	0.99655	4\\
-8.846	1.4646	4\\
1.7424	0.20474	3\\
-14.264	2.036	5\\
13.993	2.2811	5\\
18.169	-0.47604	3\\
21.309	1.436	4\\
20.437	1.3589	4\\
-17.932	1.6754	5\\
-19.703	1.8147	5\\
0.78928	-0.27663	3\\
-1.8856	-1.4105	2\\
-12.312	1.8078	5\\
-0.93778	0.64663	4\\
-10.893	2.3711	5\\
-9.0582	1.0416	4\\
-19.257	1.6051	5\\
18.364	-0.36238	3\\
5.3141	-0.65249	2\\
19.158	1.3131	4\\
-16.326	0.77729	4\\
-19.79	2.4135	5\\
-18.917	1.8718	5\\
8.9954	1.7627	5\\
16.564	0.91174	4\\
11.034	0.3685	3\\
-0.28034	0.39109	3\\
-2.0492	1.1708	4\\
-21.54	1.8204	5\\
-13.995	2.9353	5\\
16.923	0.38846	3\\
-8.8049	1.5363	5\\
16.494	1.526	5\\
17.924	1.2042	4\\
-23.049	1.6398	5\\
-17.77	1.1898	4\\
23.166	-0.58264	2\\
-2.2723	0.71459	4\\
-0.28768	-0.71181	2\\
-0.36036	-0.19799	3\\
27.332	0.75756	4\\
13.183	2.4573	5\\
14.303	0.75533	4\\
-15.999	1.0842	4\\
3.3726	1.319	4\\
5.5439	-0.19211	3\\
-5.0597	-0.33533	3\\
18.034	2.3266	5\\
8.607	1.112	4\\
-7.2616	2.017	5\\
-16.608	1.9848	5\\
2.4755	-0.95237	2\\
-16.246	1.8656	5\\
-8.8679	3.3605	5\\
-20.168	1.4818	4\\
-22.369	3.5292	5\\
-0.69813	1.478	4\\
-1.3327	-0.36285	3\\
-17.591	2.0867	5\\
-14.845	2.0205	5\\
-12.493	1.2597	4\\
15.569	1.1983	4\\
0.32483	0.10072	3\\
4.7076	-0.26158	3\\
-8.979	1.2593	4\\
-2.4323	-1.0053	2\\
-3.4256	0.27621	3\\
12.919	1.9166	5\\
-10.34	2.5782	5\\
17.195	0.3874	3\\
-9.7835	1.464	4\\
-1.3964	-0.22571	3\\
-2.0461	-0.21556	3\\
26.727	1.0647	4\\
-18.364	2.9967	5\\
9.3586	0.18411	3\\
0.36573	-1.1979	2\\
22.165	1.0629	4\\
16.636	0.47564	3\\
22.446	1.0966	4\\
2.5296	-0.38889	3\\
-17.435	2.7322	5\\
16.488	2.0765	5\\
-1.2073	0.37431	3\\
-0.34563	-0.14524	3\\
16.13	0.013648	3\\
-19.396	1.1395	4\\
-24.037	1.0646	4\\
17.936	1.7164	5\\
21.039	0.35183	3\\
-24.354	3.2613	5\\
-18.024	0.191	3\\
8.9569	0.73863	4\\
0.29446	-0.42791	3\\
-18.692	2.7642	5\\
16.051	1.1663	4\\
-15.62	0.78321	4\\
0.095311	-1.7001	1\\
-18.162	2.44	5\\
0.7444	0.12526	3\\
19.337	-0.12495	3\\
-16.954	1.7394	5\\
-1.0848	1.2655	4\\
-19.907	1.7518	5\\
-13.204	2.5053	5\\
-12.761	0.9032	4\\
-12.857	2.637	5\\
-12.56	1.8045	5\\
-21.116	1.5315	5\\
-16.43	2.7129	5\\
15.551	2.1343	5\\
-5.7879	0.94069	4\\
-12.408	2.5909	5\\
0.742	1.351	4\\
-14.325	3.37	5\\
-8.8245	0.6445	4\\
18.203	2.15	5\\
16.529	0.65694	4\\
0.045695	-1.5776	1\\
-2.0496	1.1032	4\\
-21.708	0.84136	4\\
-18.658	3.1095	5\\
0.86215	1.1643	4\\
18.313	0.70694	4\\
-15.942	0.59036	4\\
-1.0714	1.0203	4\\
-15.802	1.9545	5\\
-14.795	2.3808	5\\
-3.1376	0.96513	4\\
0.14489	0.27521	3\\
1.6173	-1.2611	2\\
16.24	0.60429	4\\
-19.011	2.4181	5\\
15.913	1.5283	5\\
-0.40381	1.0691	4\\
-19.338	1.8989	5\\
20.561	0.21787	3\\
17.802	0.69942	4\\
15.779	0.63046	4\\
-19.771	2.3615	5\\
3.2707	1.9941	5\\
2.3408	-1.2255	2\\
1.4408	0.42775	3\\
5.2746	0.66568	4\\
-13.718	1.8258	5\\
20.788	1.0904	4\\
-17.042	0.87365	4\\
18.377	1.31	4\\
-0.38271	0.085446	3\\
-16.242	1.195	4\\
1.2155	1.7634	5\\
-19.41	4.1454	5\\
-2.7152	-0.10254	3\\
-0.37977	-0.85515	2\\
15.832	0.27207	3\\
1.0446	0.12518	3\\
0.96272	-0.080057	3\\
-25.849	2.7988	5\\
-18.381	1.7587	5\\
-0.8709	0.49912	3\\
15.095	1.5519	5\\
8.6937	2.819	5\\
13.821	1.8574	5\\
-6.5132	-0.94203	2\\
-5.3645	0.066848	3\\
6.1053	-0.95518	2\\
-19.169	1.6271	5\\
0.196	-0.21229	3\\
12.615	2.0684	5\\
18.295	-0.43065	3\\
19.403	0.92756	4\\
-8.4334	2.7563	5\\
-15.825	1.6822	5\\
-16.837	0.85485	4\\
-6.1739	0.052679	3\\
-0.63527	0.30079	3\\
13.641	1.0087	4\\
19.257	0.68387	4\\
-22.254	3.0155	5\\
-12.366	1.4666	4\\
-17.854	2.6491	5\\
-20.007	1.4462	4\\
-18.664	3.6421	5\\
16.536	-0.27328	3\\
15.961	0.14749	3\\
2.0448	-0.41795	3\\
15.61	1.6359	5\\
-6.0334	-0.063586	3\\
2.6707	-0.47859	3\\
-15.749	0.73696	4\\
2.3851	-0.58206	2\\
19.216	0.81837	4\\
-20.106	2.1909	5\\
18.581	2.3682	5\\
-21.9	1.9892	5\\
0.70804	-0.43564	3\\
-11.255	2.3112	5\\
24.569	2.1004	5\\
-2.5782	1.4001	4\\
9.5656	0.95789	4\\
-21.796	2.5168	5\\
-19.688	2.3581	5\\
-17.117	0.79325	4\\
-21.747	2.3148	5\\
-0.2934	0.43283	3\\
15.27	0.44355	3\\
23.813	-0.1916	3\\
25.665	2.1687	5\\
-14.061	2.9256	5\\
-22.384	2.0678	5\\
2.9517	1.8821	5\\
15.586	0.18498	3\\
-15.727	2.6328	5\\
-2.9137	0.093849	3\\
};
\addplot[only marks, mark=x, mark options={}, mark size=3.0619pt, draw=red] table[row sep=crcr]{%
x	y\\
0.015225	-1.7769\\
1.3439	-0.89291\\
5.1983	0.047219\\
2.8809	1.0242\\
-7.1991	2.2042\\
};
\end{axis}
\end{tikzpicture}%
		\end{figure}}
		\only<4>{\begin{figure}\input{../kmeansMATLAB/kmeans_slides/Kmeans5_4}
		\end{figure}}
		\only<5>{\begin{figure}\input{../kmeansMATLAB/kmeans_slides/Kmeans5_5}
		\end{figure}}
		\only<6>{\begin{figure}\input{../kmeansMATLAB/kmeans_slides/Kmeans5_6}
		\end{figure}}
		\only<7>{\begin{figure}\input{../kmeansMATLAB/kmeans_slides/Kmeans5_7}
		\end{figure}}
		\only<8>{\begin{figure}\input{../kmeansMATLAB/kmeans_slides/Kmeans5_8}
		\end{figure}}
		\only<9>{\begin{figure}\input{../kmeansMATLAB/kmeans_slides/Kmeans5_9}
		\end{figure}}
		\only<10>{\begin{figure}\input{../kmeansMATLAB/kmeans_slides/Kmeans5_10}
		\end{figure}}
		\only<11>{\begin{figure}% This file was created by matlab2tikz.
%
\begin{tikzpicture}[%
scale=.9
]

\begin{axis}[%
width=4.521in,
height=0.714in,
at={(0.758in,1.907in)},
scale only axis,
colormap={mymap}{[1pt] rgb(0pt)=(0,0.447,0.741); rgb(1pt)=(0.466,0.674,0.188); rgb(2pt)=(0.635,0.078,0.184); rgb(3pt)=(0.929,0.694,0.125); rgb(4pt)=(0.494,0.184,0.556)},
xmin=-28,
xmax=29,
ymin=-3,
ymax=6,
axis background/.style={fill=white},
axis x line*=bottom,
axis y line*=left
]
\addplot[scatter, only marks, mark=*, mark size=0.7906pt, scatter src=explicit, scatter/use mapped color={mark options={}, draw=mapped color, fill=mapped color}] table[row sep=crcr, meta=color]{%
x	y	color\\
-2.127	0.73415	1\\
0.12293	0.34333	2\\
2.6322	-0.25433	2\\
-1.7516	0.1583	1\\
4.2744	0.84681	4\\
-13.039	2.5839	5\\
-12.896	1.5214	5\\
8.8887	1.1665	4\\
17.612	0.028451	3\\
-5.3224	0.6551	1\\
-0.61235	1.2855	2\\
0.90233	1.9645	2\\
0.87487	0.14228	2\\
-22.138	0.83279	5\\
0.58648	-0.83253	2\\
0.70504	-0.51508	2\\
-1.0445	-0.14726	2\\
16.332	0.47711	3\\
18.018	1.764	3\\
-19.363	2.0115	5\\
17.455	0.31954	3\\
-20.99	3.6559	5\\
-1.1849	-1.1651	2\\
15.418	0.83152	3\\
17.299	0.050292	3\\
-8.9338	2.7918	1\\
1.2671	-0.86747	2\\
-1.2844	-0.71335	2\\
9.1051	1.8378	4\\
-16.685	2.6888	5\\
3.6319	-0.81409	2\\
18.581	0.83276	3\\
-15.376	0.98107	5\\
3.1471	-0.32248	2\\
-2.6661	-0.81637	1\\
13.737	-0.16353	3\\
-21.418	1.3191	5\\
7.1211	0.43766	4\\
19.508	0.043772	3\\
-24.166	2.2257	5\\
-16.955	1.3529	5\\
2.5199	-0.89007	2\\
-8.1501	-0.3264	1\\
17.199	1.2645	3\\
-19.87	2.0373	5\\
-1.5221	0.47455	2\\
16.326	0.95318	3\\
-4.1181	-0.19492	1\\
-16.053	1.7766	5\\
-21.148	1.9617	5\\
10.417	0.41623	4\\
-20.448	1.6465	5\\
-16.841	2.7393	5\\
-19.593	3.2643	5\\
-2.0857	1.5886	1\\
19.63	-0.57401	3\\
18.567	1.9852	3\\
14.725	1.1378	3\\
13.274	-0.15107	3\\
19.755	2.1933	3\\
-18.993	1.8712	5\\
17.471	0.59715	3\\
-6.6079	0.76265	1\\
-17.52	2.2071	5\\
-17.113	2.0397	5\\
-15.975	1.573	5\\
-16.893	3.0714	5\\
-14.307	3.0438	5\\
17.527	0.9732	3\\
1.9193	-0.30523	2\\
-18.928	0.90504	5\\
19.744	0.19442	3\\
1.851	0.1218	2\\
-20.698	2.0605	5\\
-15.707	1.1417	5\\
-14.337	3.0377	5\\
-22.781	1.2982	5\\
18.61	0.68326	3\\
23.292	1.994	3\\
-16.81	1.1239	5\\
-19.846	2.0056	5\\
-0.2042	1.2462	2\\
20.958	0.61439	3\\
-1.9219	0.10002	1\\
13.394	-0.1251	3\\
-17.514	1.9608	5\\
-17.122	3.3289	5\\
0.14245	0.093343	2\\
16.354	0.72815	3\\
3.9861	1.1496	4\\
18.845	0.022903	3\\
-8.9976	1.5202	1\\
2.6355	-0.93966	2\\
1.1235	1.4347	2\\
-17.787	1.4423	5\\
11.848	0.43308	4\\
9.2291	0.97955	4\\
22.844	1.096	3\\
0.87247	-0.49324	2\\
14.928	1.4909	3\\
-20.522	2.037	5\\
2.3555	-1.0241	2\\
-19.117	2.8662	5\\
-11.65	0.35501	5\\
18.041	0.19723	3\\
13.05	-0.00067892	3\\
-5.5607	0.65167	1\\
-18.566	1.8204	5\\
-14.565	1.7472	5\\
23.238	1.057	3\\
3.6443	0.72558	4\\
21.178	1.2841	3\\
18.894	1.6512	3\\
-20.956	2.2944	5\\
9.0575	2.1157	4\\
14.434	0.12862	3\\
-13.3	1.186	5\\
-15.355	2.2855	5\\
-20.443	2.3651	5\\
3.3269	0.14394	2\\
2.5296	-0.35928	2\\
3.5743	0.66374	4\\
15.974	1.6563	3\\
-20.694	1.9705	5\\
-20.899	1.9207	5\\
8.0972	-0.61117	4\\
23.653	0.041456	3\\
-2.168	-0.86138	1\\
21.442	0.82289	3\\
15.261	0.93323	3\\
-5.516	-1.4643	1\\
18.21	1.8717	3\\
18.156	0.17916	3\\
6.2589	0.077333	4\\
-14.157	0.70826	5\\
6.7601	2.2874	4\\
19.394	1.2668	3\\
20.989	2.1131	3\\
2.2116	1.2636	2\\
22.391	1.0396	3\\
6.1711	0.49958	4\\
13.847	1.6351	3\\
5.0109	-1.9477	4\\
15.31	0.61724	3\\
4.1737	-0.19945	4\\
24.841	-0.029906	3\\
-16.18	2.8186	5\\
14.343	-0.032705	3\\
-17.479	1.7669	5\\
-9.052	1.8202	1\\
-15.651	2.1286	5\\
-6.8739	-0.72605	1\\
-0.96147	0.6593	2\\
1.1156	0.67873	2\\
15.688	1.8731	3\\
-2.2067	-1.8339	1\\
2.9011	-0.053823	2\\
0.030776	-0.50569	2\\
-14.245	1.9968	5\\
-19.329	1.3108	5\\
-0.24889	-1.0336	2\\
13.79	2.5835	3\\
-6.3124	0.2848	1\\
-18.507	1.8311	5\\
18.032	0.44777	3\\
21.565	0.58578	3\\
23.953	1.2824	3\\
4.9779	-0.60092	4\\
7.4764	1.0538	4\\
20.104	0.54245	3\\
20.977	1.8035	3\\
6.1444	0.49385	4\\
4.9805	-0.99259	4\\
-6.3149	-0.52842	1\\
23.912	1.2239	3\\
-0.3008	-1.7044	2\\
12.885	1.5924	3\\
13.336	1.9947	3\\
0.18064	0.0022266	2\\
-18.022	1.2523	5\\
-3.417	0.98246	1\\
19.572	2.2865	3\\
-0.42698	-0.26639	2\\
16.695	1.4707	3\\
-2.2984	0.042803	1\\
1.4802	-0.036955	2\\
0.21509	0.45436	2\\
3.7119	0.66793	4\\
-23.261	2.5821	5\\
-4.7926	-0.15593	1\\
16.191	2.4757	3\\
-9.5381	1.2114	1\\
-19.665	0.99481	5\\
-5.1286	-0.52033	1\\
-15.053	0.49003	5\\
12.614	1.7112	3\\
14.775	1.0133	3\\
-1.3005	-0.43194	2\\
3.2342	-1.3026	2\\
2.4511	0.31748	2\\
13.258	2.2227	3\\
-8.8497	2.1718	1\\
-16.818	1.4558	5\\
16.453	0.57047	3\\
7.3399	0.36447	4\\
-17.725	1.8224	5\\
17.397	1.3497	3\\
-15.792	2.7605	5\\
16.792	0.81172	3\\
23.614	0.84199	3\\
22.458	2.4921	3\\
13.322	2.2329	3\\
11.106	0.62047	4\\
-0.9846	-1.1514	2\\
-0.91146	0.47512	2\\
1.5972	1.2391	2\\
-12.863	2.0545	5\\
-1.2232	-0.058934	2\\
-17.192	2.9956	5\\
-2.8052	-0.0090214	1\\
-17.627	1.6993	5\\
-14.189	1.836	5\\
-1.1821	0.18301	2\\
13.871	1.1584	3\\
6.2683	0.22557	4\\
-12.672	1.3712	5\\
-20.241	0.55605	5\\
18.41	-0.18864	3\\
-26.038	2.9527	5\\
-18.221	2.0181	5\\
-17.314	2.128	5\\
0.2527	0.02121	2\\
-4.8296	-0.21708	1\\
-16.663	1.4262	5\\
-18.817	2.2904	5\\
0.59601	1.0251	2\\
-2.1371	-0.88985	1\\
15.471	0.61058	3\\
-9.0179	2.6024	1\\
6.5559	0.47827	4\\
12.162	1.5525	3\\
14.173	0.93358	3\\
-2.1816	-0.060798	1\\
1.9753	-0.18622	2\\
-0.71198	1.5155	2\\
20.241	1.5176	3\\
-9.3054	1.8227	1\\
15.454	1.6872	3\\
-20.254	0.77697	5\\
2.1716	0.34842	2\\
18.546	1.1313	3\\
-1.6338	0.023384	1\\
-5.1888	0.073572	1\\
20.272	1.6295	3\\
15.314	2.5955	3\\
-11.577	2.7853	5\\
-13.658	2.5376	5\\
-22.259	1.4595	5\\
-2.5531	-1.8976	1\\
-16.615	1.3862	5\\
-11.934	0.85027	5\\
19.1	0.74627	3\\
18.514	1.162	3\\
-14.082	2.4623	5\\
17.003	0.092386	3\\
-13.772	1.667	5\\
-21.547	2.2739	5\\
-19.267	1.5064	5\\
17.957	1.2053	3\\
0.85508	-1.4552	2\\
15.091	1.8004	3\\
14.428	1.5956	3\\
-18.359	1.3961	5\\
-0.90885	-0.17982	2\\
1.3893	-1.0127	2\\
-20.887	3.537	5\\
19.84	0.47332	3\\
-3.5652	0.0694	1\\
-14.947	2.0965	5\\
16.417	1.2837	3\\
-0.90323	-0.57465	2\\
0.45136	-0.44553	2\\
3.0955	-0.22653	2\\
-18.672	1.4509	5\\
18.41	1.2561	3\\
18.007	0.6732	3\\
21.238	0.99655	3\\
-8.846	1.4646	1\\
1.7424	0.20474	2\\
-14.264	2.036	5\\
13.993	2.2811	3\\
18.169	-0.47604	3\\
21.309	1.436	3\\
20.437	1.3589	3\\
-17.932	1.6754	5\\
-19.703	1.8147	5\\
0.78928	-0.27663	2\\
-1.8856	-1.4105	1\\
-12.312	1.8078	5\\
-0.93778	0.64663	2\\
-10.893	2.3711	5\\
-9.0582	1.0416	1\\
-19.257	1.6051	5\\
18.364	-0.36238	3\\
5.3141	-0.65249	4\\
19.158	1.3131	3\\
-16.326	0.77729	5\\
-19.79	2.4135	5\\
-18.917	1.8718	5\\
8.9954	1.7627	4\\
16.564	0.91174	3\\
11.034	0.3685	4\\
-0.28034	0.39109	2\\
-2.0492	1.1708	1\\
-21.54	1.8204	5\\
-13.995	2.9353	5\\
16.923	0.38846	3\\
-8.8049	1.5363	1\\
16.494	1.526	3\\
17.924	1.2042	3\\
-23.049	1.6398	5\\
-17.77	1.1898	5\\
23.166	-0.58264	3\\
-2.2723	0.71459	1\\
-0.28768	-0.71181	2\\
-0.36036	-0.19799	2\\
27.332	0.75756	3\\
13.183	2.4573	3\\
14.303	0.75533	3\\
-15.999	1.0842	5\\
3.3726	1.319	2\\
5.5439	-0.19211	4\\
-5.0597	-0.33533	1\\
18.034	2.3266	3\\
8.607	1.112	4\\
-7.2616	2.017	1\\
-16.608	1.9848	5\\
2.4755	-0.95237	2\\
-16.246	1.8656	5\\
-8.8679	3.3605	1\\
-20.168	1.4818	5\\
-22.369	3.5292	5\\
-0.69813	1.478	2\\
-1.3327	-0.36285	2\\
-17.591	2.0867	5\\
-14.845	2.0205	5\\
-12.493	1.2597	5\\
15.569	1.1983	3\\
0.32483	0.10072	2\\
4.7076	-0.26158	4\\
-8.979	1.2593	1\\
-2.4323	-1.0053	1\\
-3.4256	0.27621	1\\
12.919	1.9166	3\\
-10.34	2.5782	1\\
17.195	0.3874	3\\
-9.7835	1.464	1\\
-1.3964	-0.22571	2\\
-2.0461	-0.21556	1\\
26.727	1.0647	3\\
-18.364	2.9967	5\\
9.3586	0.18411	4\\
0.36573	-1.1979	2\\
22.165	1.0629	3\\
16.636	0.47564	3\\
22.446	1.0966	3\\
2.5296	-0.38889	2\\
-17.435	2.7322	5\\
16.488	2.0765	3\\
-1.2073	0.37431	2\\
-0.34563	-0.14524	2\\
16.13	0.013648	3\\
-19.396	1.1395	5\\
-24.037	1.0646	5\\
17.936	1.7164	3\\
21.039	0.35183	3\\
-24.354	3.2613	5\\
-18.024	0.191	5\\
8.9569	0.73863	4\\
0.29446	-0.42791	2\\
-18.692	2.7642	5\\
16.051	1.1663	3\\
-15.62	0.78321	5\\
0.095311	-1.7001	2\\
-18.162	2.44	5\\
0.7444	0.12526	2\\
19.337	-0.12495	3\\
-16.954	1.7394	5\\
-1.0848	1.2655	2\\
-19.907	1.7518	5\\
-13.204	2.5053	5\\
-12.761	0.9032	5\\
-12.857	2.637	5\\
-12.56	1.8045	5\\
-21.116	1.5315	5\\
-16.43	2.7129	5\\
15.551	2.1343	3\\
-5.7879	0.94069	1\\
-12.408	2.5909	5\\
0.742	1.351	2\\
-14.325	3.37	5\\
-8.8245	0.6445	1\\
18.203	2.15	3\\
16.529	0.65694	3\\
0.045695	-1.5776	2\\
-2.0496	1.1032	1\\
-21.708	0.84136	5\\
-18.658	3.1095	5\\
0.86215	1.1643	2\\
18.313	0.70694	3\\
-15.942	0.59036	5\\
-1.0714	1.0203	2\\
-15.802	1.9545	5\\
-14.795	2.3808	5\\
-3.1376	0.96513	1\\
0.14489	0.27521	2\\
1.6173	-1.2611	2\\
16.24	0.60429	3\\
-19.011	2.4181	5\\
15.913	1.5283	3\\
-0.40381	1.0691	2\\
-19.338	1.8989	5\\
20.561	0.21787	3\\
17.802	0.69942	3\\
15.779	0.63046	3\\
-19.771	2.3615	5\\
3.2707	1.9941	2\\
2.3408	-1.2255	2\\
1.4408	0.42775	2\\
5.2746	0.66568	4\\
-13.718	1.8258	5\\
20.788	1.0904	3\\
-17.042	0.87365	5\\
18.377	1.31	3\\
-0.38271	0.085446	2\\
-16.242	1.195	5\\
1.2155	1.7634	2\\
-19.41	4.1454	5\\
-2.7152	-0.10254	1\\
-0.37977	-0.85515	2\\
15.832	0.27207	3\\
1.0446	0.12518	2\\
0.96272	-0.080057	2\\
-25.849	2.7988	5\\
-18.381	1.7587	5\\
-0.8709	0.49912	2\\
15.095	1.5519	3\\
8.6937	2.819	4\\
13.821	1.8574	3\\
-6.5132	-0.94203	1\\
-5.3645	0.066848	1\\
6.1053	-0.95518	4\\
-19.169	1.6271	5\\
0.196	-0.21229	2\\
12.615	2.0684	3\\
18.295	-0.43065	3\\
19.403	0.92756	3\\
-8.4334	2.7563	1\\
-15.825	1.6822	5\\
-16.837	0.85485	5\\
-6.1739	0.052679	1\\
-0.63527	0.30079	2\\
13.641	1.0087	3\\
19.257	0.68387	3\\
-22.254	3.0155	5\\
-12.366	1.4666	5\\
-17.854	2.6491	5\\
-20.007	1.4462	5\\
-18.664	3.6421	5\\
16.536	-0.27328	3\\
15.961	0.14749	3\\
2.0448	-0.41795	2\\
15.61	1.6359	3\\
-6.0334	-0.063586	1\\
2.6707	-0.47859	2\\
-15.749	0.73696	5\\
2.3851	-0.58206	2\\
19.216	0.81837	3\\
-20.106	2.1909	5\\
18.581	2.3682	3\\
-21.9	1.9892	5\\
0.70804	-0.43564	2\\
-11.255	2.3112	5\\
24.569	2.1004	3\\
-2.5782	1.4001	1\\
9.5656	0.95789	4\\
-21.796	2.5168	5\\
-19.688	2.3581	5\\
-17.117	0.79325	5\\
-21.747	2.3148	5\\
-0.2934	0.43283	2\\
15.27	0.44355	3\\
23.813	-0.1916	3\\
25.665	2.1687	3\\
-14.061	2.9256	5\\
-22.384	2.0678	5\\
2.9517	1.8821	2\\
15.586	0.18498	3\\
-15.727	2.6328	5\\
-2.9137	0.093849	1\\
};
\addplot[only marks, mark=x, mark options={}, mark size=3.0619pt, draw=red] table[row sep=crcr]{%
x	y\\
-5.2511	0.49027\\
0.69538	-0.0025604\\
17.85	1.0259\\
7.0615	0.51938\\
-17.65	1.952\\
};
\end{axis}
\end{tikzpicture}%
		\end{figure}}
		\only<12>{\begin{figure}\input{../kmeansMATLAB/kmeans_slides/Kmeans5_12}
		\end{figure}}
		\only<13>{\begin{figure}\input{../kmeansMATLAB/kmeans_slides/Kmeans5_13}
		\end{figure}}
		\only<14>{\begin{figure}\input{../kmeansMATLAB/kmeans_slides/Kmeans5_14}
		\end{figure}}
		\only<15>{\begin{figure}\input{../kmeansMATLAB/kmeans_slides/Kmeans5_15}
		\end{figure}}
%		\only<16>{\begin{figure}\input{../kmeansMATLAB/kmeans_slides/Kmeans5_16}
%		\end{figure}}
%		\only<17>{\begin{figure}\input{../kmeansMATLAB/kmeans_slides/Kmeans5_17}
%		\end{figure}}
	}
		\end{center}
	
	\end{frame}

	\begin{frame}{EM Algorithm on Gaussian Mixture}
		\begin{figure}
%			\centering
			\begin{subfigure}{.9\linewidth}
				\includegraphics[scale=.4]{../kmeansMATLAB/EM3GMM}
				\subcaption{AIC\( \approx 5030 \), BIC\( \approx 5102\)}
			\end{subfigure}
			\pause
			\visible<2->{
				\begin{subfigure}{.9\linewidth}
					\ \\
				\end{subfigure}
				\begin{subfigure}{.9\linewidth}
					\includegraphics[scale=.399]{../kmeansMATLAB/EM5GMM}
					\subcaption{AIC\( \approx 4994 \), BIC\( \approx 5116\)}
				\end{subfigure}
			}
		\end{figure}
	\end{frame}

	\begin{frame}{EM Algorithm on Non-Gaussian Data}
		\only<1>{\begin{figure}
					\centering
					% This file was created by matlab2tikz.
%
%The latest updates can be retrieved from
%  http://www.mathworks.com/matlabcentral/fileexchange/22022-matlab2tikz-matlab2tikz
%where you can also make suggestions and rate matlab2tikz.
%
\begin{tikzpicture}[%
scale=.7
]

\begin{axis}[%
width=4.521in,
height=2.26in,
at={(0.758in,1.134in)},
scale only axis,
colormap={mymap}{[1pt] rgb(0pt)=(0,0.447,0.741); rgb(1pt)=(0.929,0.694,0.125); rgb(2pt)=(0.466,0.674,0.188); rgb(3pt)=(0.635,0.078,0.184)},
xmin=-8,
xmax=8,
ymin=-4,
ymax=4,
axis background/.style={fill=white},
axis x line*=bottom,
axis y line*=left
]
\addplot[scatter, only marks, mark=*, mark size=0.9682pt, scatter src=explicit, scatter/use mapped color={mark options={}, draw=mapped color, fill=mapped color}] table[row sep=crcr, meta=color]{%
x	y	color\\
-2.8442	1.3251	1\\
-3.8156	1.8954	1\\
-1.4036	0.11607	1\\
-5.169	1.3588	1\\
-5.3902	1.5769	1\\
-6.6113	0.222	1\\
-2.3326	1.2749	1\\
-2.3995	1.1318	1\\
-4.1597	1.8971	1\\
-2.351	0.98832	1\\
-3.634	1.3206	1\\
-3.4238	1.7049	1\\
-3.7744	1.6932	1\\
-3.1386	1.4724	1\\
-3.7969	1.4525	1\\
-1.0081	-1.1842	1\\
-6.0268	0.83239	1\\
-6.6834	0.13795	1\\
-5.9932	0.98843	1\\
-5.9303	0.87383	1\\
-4.8495	1.2407	1\\
-4.3359	1.5675	1\\
-4.4348	1.9095	1\\
-2.9515	1.4242	1\\
-3.2344	1.6191	1\\
-2.5862	1.3948	1\\
-1.3682	0.18383	1\\
-1.9025	1.0208	1\\
-6.1695	0.83613	1\\
-6.9063	-0.39936	1\\
-2.3457	1.2756	1\\
-6.6712	0.042098	1\\
-3.8666	1.6622	1\\
-3.9291	1.7423	1\\
-1.1284	-0.40136	1\\
-4.1384	1.6901	1\\
-4.9864	1.4415	1\\
-3.5313	1.4068	1\\
-3.5117	1.536	1\\
-3.86	1.7541	1\\
-5.1691	1.3432	1\\
-6.0969	0.89454	1\\
-3.374	1.9901	1\\
-6.2374	1.1679	1\\
-5.8459	1.4076	1\\
-1.7756	0.76036	1\\
-4.8218	1.7092	1\\
-4.28	1.8309	1\\
-2.3489	1.1966	1\\
-4.5943	1.7723	1\\
-2.1581	1.1941	1\\
-4.6811	1.4997	1\\
-2.1816	0.92965	1\\
-2.7944	0.95977	1\\
-4.4239	1.6094	1\\
-5.5983	1.2271	1\\
-5.8612	1.1677	1\\
-4.5358	1.6908	1\\
-4.858	1.5848	1\\
-4.8025	1.9418	1\\
-2.2238	1.1692	1\\
-4.5896	1.4916	1\\
-2.0029	1.0528	1\\
-5.0309	1.4185	1\\
-2.2399	0.93109	1\\
-5.0138	1.3081	1\\
-3.6481	1.5239	1\\
-2.2425	1.5833	1\\
-4.8125	1.7008	1\\
-4.8582	1.3943	1\\
-1.229	-0.22942	1\\
-2.2717	1.1512	1\\
-4.8465	1.4907	1\\
-2.7601	1.2058	1\\
-4.7099	1.584	1\\
-2.6623	1.537	1\\
-1.6841	0.64306	1\\
-6.4953	0.407	1\\
-2.1588	0.88922	1\\
-1.6412	0.47118	1\\
-3.9051	1.589	1\\
-1.3914	0.15198	1\\
-3.6901	1.7564	1\\
-6.1146	0.81575	1\\
-6.8587	-0.43353	1\\
-4.6587	1.9064	1\\
-2.6493	1.343	1\\
-3.3722	1.4481	1\\
-3.1882	1.6101	1\\
-5.113	1.1714	1\\
-3.7127	1.5741	1\\
-6.5242	0.40529	1\\
-5.4285	1.3689	1\\
-5.5241	1.3391	1\\
-2.9908	1.4769	1\\
-4.0283	1.5489	1\\
-6.0125	0.98322	1\\
-4.019	1.6083	1\\
-5.3103	1.0476	1\\
-5.2706	1.2313	1\\
-2.4041	1.0753	1\\
-3.7735	1.8544	1\\
-3.2171	1.7029	1\\
-2.3733	1.1722	1\\
-3.6408	1.6985	1\\
-2.9342	1.8293	1\\
-2.7014	1.7072	1\\
-1.9681	0.70193	1\\
-3.0716	1.7044	1\\
-1.6879	0.19592	1\\
-3.9548	2.1441	1\\
-4.0957	1.6087	1\\
-1.4254	0.23977	1\\
-4.9908	1.6829	1\\
-4.0057	1.6191	1\\
-2.3184	1.3431	1\\
-3.7011	1.5131	1\\
-1.6955	0.60742	1\\
-2.1514	1.3907	1\\
-3.4223	2.1288	1\\
-5.4575	1.3789	1\\
-2.9057	1.2641	1\\
-6.2808	0.61261	1\\
-3.5143	1.6496	1\\
-2.4926	1.1562	1\\
-3.5761	1.9486	1\\
-2.3799	0.93988	1\\
-1.2106	-0.016956	1\\
-2.3122	1.1726	1\\
-3.4704	1.7256	1\\
-2.7728	1.4076	1\\
-3.1658	1.9706	1\\
-6.2058	0.42854	1\\
-4.8171	1.4774	1\\
-1.9005	0.44094	1\\
-4.0872	1.4154	1\\
-2.8775	1.4798	1\\
-5.8093	1.2808	1\\
-3.871	2.045	1\\
-2.8456	1.4656	1\\
-6.258	0.63142	1\\
-3.314	1.4114	1\\
-4.3624	1.8682	1\\
-5.7112	1.0051	1\\
-4.0751	1.6848	1\\
-6.7099	0.14182	1\\
-6.8314	-0.4632	1\\
-6.0399	0.8466	1\\
-5.7863	1.1572	1\\
-6.0349	0.52886	1\\
-6.5931	0.086756	1\\
-3.4288	1.6672	1\\
-6.4863	0.32331	1\\
-3.7799	1.6745	1\\
-2.5272	1.2789	1\\
-4.0045	1.6093	1\\
-3.814	1.8149	1\\
-4.7445	1.7656	1\\
-6.3832	0.45002	1\\
-4.0876	1.3217	1\\
-2.2571	1.2032	1\\
-1.3973	0.3945	1\\
-4.8431	1.6268	1\\
-6.5147	0.43272	1\\
-3.6683	1.7471	1\\
-3.2591	1.512	1\\
-4.4513	1.6919	1\\
-6.0039	1.0514	1\\
-1.218	-0.088455	1\\
-6.9155	-0.32857	1\\
-4.9913	1.7034	1\\
-6.2136	1.0074	1\\
-5.2175	1.545	1\\
-3.3714	1.8504	1\\
-3.7245	1.8288	1\\
-5.3459	1.4415	1\\
-2.5437	1.5629	1\\
-2.8066	1.4563	1\\
-2.7628	1.365	1\\
-4.8345	1.7743	1\\
-2.1312	0.824	1\\
-2.7397	1.7397	1\\
-1.0473	-0.69083	1\\
-2.1105	1.3271	1\\
-1.9857	0.99214	1\\
-3.8941	1.9629	1\\
-6.1755	0.88066	1\\
-4.7312	1.7577	1\\
-5.4247	1.5771	1\\
-5.0894	1.9082	1\\
-2.9265	1.3883	1\\
-5.0765	1.1002	1\\
-5.6439	0.76485	1\\
-4.8037	1.6888	1\\
-4.3209	1.692	1\\
-3.1963	1.6835	1\\
-5.7308	1.1352	1\\
-3.3724	1.8174	1\\
-1.1774	-1.4848	2\\
0.88342	-0.68934	2\\
-1.7575	-1.6124	2\\
-1.6018	-1.5614	2\\
-3.3661	-1.1981	2\\
-3.617	-0.76187	2\\
-0.7189	-2.0024	2\\
-1.8289	-1.7486	2\\
-0.35347	-1.3456	2\\
-2.9734	-1.0685	2\\
1.384	0.48623	2\\
-1.0893	-1.6567	2\\
0.53335	-1.4798	2\\
-1.7846	-1.328	2\\
-1.6574	-1.426	2\\
-0.022579	-1.515	2\\
-0.2587	-1.4564	2\\
-3.2997	-0.80985	2\\
-3.7642	-0.38113	2\\
-3.2697	-0.85209	2\\
-3.6491	-0.62966	2\\
-1.2142	-1.7054	2\\
-2.1407	-1.5717	2\\
-2.0584	-1.769	2\\
-3.5907	-0.39135	2\\
0.12718	-1.3192	2\\
-0.47297	-1.4367	2\\
-1.5124	-1.8656	2\\
0.49696	-1.292	2\\
-2.9445	-1.2488	2\\
-0.51948	-1.6381	2\\
-0.64573	-1.3812	2\\
-0.64416	-1.4986	2\\
-2.7105	-1.2571	2\\
-0.68732	-1.9186	2\\
-2.3336	-1.7219	2\\
-2.9793	-1.1678	2\\
-3.4064	-0.9381	2\\
0.52045	-0.80069	2\\
-2.9006	-0.98983	2\\
-1.678	-1.7963	2\\
-3.4131	-0.68708	2\\
-1.9218	-1.3626	2\\
-4.325	1.5873	2\\
-3.255	-0.67907	2\\
-3.6602	-0.38116	2\\
-3.154	-1.0415	2\\
-1.7577	-1.4945	2\\
-1.9765	-1.5914	2\\
1.4051	-0.0081048	2\\
-2.4368	-1.418	2\\
-1.8299	-1.6514	2\\
-3.7976	-0.48384	2\\
-2.2645	-1.1976	2\\
-3.5653	-0.6603	2\\
-1.2727	-1.1533	2\\
0.2452	-1.3104	2\\
-0.53925	-1.4801	2\\
-2.6621	-1.0732	2\\
-3.8508	0.2141	2\\
1.0275	-0.49033	2\\
1.4616	0.051868	2\\
-1.0927	-1.5297	2\\
-3.1187	-1.3749	2\\
0.73285	-0.92983	2\\
-2.4741	-1.4203	2\\
-2.2505	-1.2585	2\\
0.29709	-0.93085	2\\
-3.6169	-0.6511	2\\
-4.0788	0.27554	2\\
-0.50819	-1.6485	2\\
-0.54591	-1.4626	2\\
-1.1432	-1.3109	2\\
-0.36647	-1.5086	2\\
0.5819	-1.0692	2\\
0.4237	-1.2316	2\\
-2.431	-1.3487	2\\
0.17655	-1.0249	2\\
-0.98799	-1.2549	2\\
-2.3165	-1.5235	2\\
-3.3939	-0.84776	2\\
-0.096896	-1.5599	2\\
-2.0309	-1.2954	2\\
-1.2405	-1.7137	2\\
-2.8157	-1.4028	2\\
-3.7806	0.001707	2\\
-0.99128	-1.5185	2\\
1.2091	2.0241	3\\
3.4011	0.79048	3\\
2.8299	1.431	3\\
3.2088	1.1008	3\\
-0.076565	1.6474	3\\
-0.37309	1.3484	3\\
-0.81957	0.8701	3\\
2.3026	1.6843	3\\
4.3254	-0.90839	3\\
2.7076	1.4511	3\\
-0.11652	1.5149	3\\
2.8859	0.99866	3\\
-0.64366	1.1541	3\\
-0.99146	0.23254	3\\
2.6906	1.4503	3\\
-0.12446	1.6123	3\\
1.9584	1.7579	3\\
3.6395	0.43188	3\\
4.7882	-1.4651	4\\
5.6964	-1.2378	4\\
1.5036	-0.0099777	4\\
5.2617	-1.1197	4\\
6.836	0.35437	4\\
5.0482	-1.5675	4\\
1.5615	-0.21711	4\\
3.5956	-1.9019	4\\
3.0271	-1.4031	4\\
4.7538	-1.7376	4\\
4.8088	-1.669	4\\
4.7964	-1.7267	4\\
3.283	-1.4979	4\\
2.8474	-1.6254	4\\
2.8647	-1.6896	4\\
5.5792	-1.1401	4\\
1.3174	-0.12774	4\\
3.5874	-1.5992	4\\
4.5364	-1.722	4\\
2.7535	-1.3749	4\\
4.9309	-2.0648	4\\
3.8574	-1.6034	4\\
2.2906	-0.79465	4\\
4.3218	-1.4748	4\\
3.4679	-1.4703	4\\
3.1953	-1.638	4\\
2.6531	-1.3444	4\\
4.1228	-1.6486	4\\
1.3219	-0.032685	4\\
3.2171	-1.7621	4\\
2.9475	-1.5797	4\\
2.3305	-1.2275	4\\
5.2464	-1.525	4\\
2.6145	-1.1935	4\\
5.0473	-1.0036	4\\
5.7649	-1.3906	4\\
3.6871	-1.658	4\\
4.5509	-1.9375	4\\
5.2234	-1.1787	4\\
3.8382	-1.4136	4\\
5.0477	-1.8436	4\\
1.8548	-0.40845	4\\
5.9532	-1.194	4\\
2.2151	-1.0659	4\\
1.3333	0.14542	4\\
6.2828	-0.61705	4\\
3.8746	-1.2382	4\\
4.8782	-1.0181	4\\
3.284	-1.5856	4\\
2.1706	-1.459	4\\
1.0009	1.2991	4\\
5.2003	-1.6191	4\\
3.4318	-1.7987	4\\
3.607	-1.8997	4\\
4.7966	-1.5767	4\\
1.6805	-0.18128	4\\
2.6225	-1.5082	4\\
6.0019	-0.99051	4\\
5.1501	-1.2375	4\\
1.8328	-0.53182	4\\
1.1375	0.49942	4\\
2.8975	-1.2849	4\\
4.4258	-1.4955	4\\
5.7155	-1.2375	4\\
4.1873	-1.6084	4\\
2.031	-1.0232	4\\
4.606	-1.8181	4\\
1.9659	-0.77819	4\\
2.2569	-1.1282	4\\
2.7518	-1.4654	4\\
1.8948	-0.77893	4\\
3.4625	-1.4539	4\\
2.3487	-1.1261	4\\
3.4338	-1.6072	4\\
3.0966	-1.5048	4\\
3.1954	-1.5799	4\\
3.0205	-1.6905	4\\
6.11	-0.67522	4\\
5.1721	-1.1798	4\\
4.1181	-1.8045	4\\
1.6322	-0.15666	4\\
1.4866	-0.37314	4\\
1.25	0.052265	4\\
6.9704	0.69995	4\\
5.7599	-1.2881	4\\
4.5102	-1.8469	4\\
2.2374	-1.0568	4\\
1.9994	-0.61612	4\\
5.06	-1.401	4\\
6.4605	-0.24553	4\\
1.4316	0.018094	4\\
3.1209	-2.0682	4\\
3.8315	-1.7288	4\\
1.2351	0.20652	4\\
5.5598	-1.1191	4\\
3.1862	-1.8485	4\\
6.7195	-0.16778	4\\
3.2313	-1.4417	4\\
4.8034	-1.6137	4\\
2.0045	-0.82948	4\\
2.8022	-1.6426	4\\
1.9706	-1.2361	4\\
4.7719	-1.4467	4\\
6.1911	-0.51053	4\\
2.8575	-1.1475	4\\
4.8547	-1.8821	4\\
3.019	-1.8024	4\\
4.204	-1.9878	4\\
6.0643	-0.8852	4\\
4.2402	-1.8135	4\\
2.1476	-0.79695	4\\
2.4152	-1.5382	4\\
3.6613	-2.0293	4\\
3.6182	-1.6345	4\\
2.4015	-1.0001	4\\
4.36	-1.8389	4\\
6.1499	-1.0455	4\\
3.5012	-1.6881	4\\
3.5928	-1.9995	4\\
1.1381	0.40362	4\\
2.5478	-1.1594	4\\
1.7078	-0.59192	4\\
3.2401	-1.984	4\\
5.0199	-1.6287	4\\
6.4221	-0.53058	4\\
6.646	0.249	4\\
3.7666	-1.5079	4\\
3.3668	-1.6438	4\\
5.522	-1.0234	4\\
4.9548	-1.7529	4\\
5.437	-1.4031	4\\
5.4584	-1.0391	4\\
1.6191	-0.2247	4\\
5.0468	-1.3654	4\\
3.8171	-1.5963	4\\
2.5713	-1.4552	4\\
2.6832	-1.2466	4\\
6.1674	-0.9398	4\\
2.4905	-1.1777	4\\
1.8857	-0.77197	4\\
4.4272	-1.4416	4\\
6.7749	0.074739	4\\
4.0511	-1.5723	4\\
4.6102	-1.4787	4\\
2.2907	-1.3451	4\\
6.5532	-0.24386	4\\
4.2495	-1.3702	4\\
5.0281	-1.4958	4\\
2.2701	-1.2127	4\\
5.58	-1.1589	4\\
4.7654	-1.6557	4\\
1.9869	-0.83566	4\\
4.0795	-1.7259	4\\
3.1187	-1.5713	4\\
4.2182	-1.5999	4\\
3.5024	-1.386	4\\
3.6217	-1.5857	4\\
2.1531	-0.73443	4\\
3.3608	-2.0368	4\\
5.4391	-1.4221	4\\
6.6075	-0.1364	4\\
1.9811	-1.086	4\\
6.9206	0.35933	4\\
6.4247	-0.34316	4\\
4.336	-1.5586	4\\
3.4036	-1.5822	4\\
2.6746	-1.3123	4\\
5.5464	-1.4449	4\\
2.6815	-1.4943	4\\
3.0976	-1.5824	4\\
5.6262	-1.3586	4\\
4.8539	-1.5095	4\\
5.6464	-0.73482	4\\
4.9817	-1.6313	4\\
3.2123	-1.4652	4\\
3.0016	-1.7359	4\\
5.4879	-1.0371	4\\
2.7216	-1.6078	4\\
6.0522	-0.84649	4\\
5.4787	-1.434	4\\
4.5702	-1.5429	4\\
4.0065	-1.6977	4\\
4.7205	-1.621	4\\
6.9612	2.3709	4\\
3.7754	-1.8176	4\\
1.5472	-0.43669	4\\
6.7213	0.57377	4\\
5.1713	-1.6912	4\\
2.7619	-1.322	4\\
6.9723	0.67786	4\\
5.0659	-1.2981	4\\
4.5828	-1.5773	4\\
2.5423	-1.1916	4\\
2.1085	-1.0843	4\\
1.3752	-0.10373	4\\
3.1706	-1.2886	4\\
2.1249	-1.4968	4\\
};
\end{axis}

\begin{axis}[%
width=5.833in,
height=4.375in,
at={(0in,0in)},
scale only axis,
xmin=0,
xmax=1,
ymin=0,
ymax=1,
axis line style={draw=none},
ticks=none,
axis x line*=bottom,
axis y line*=left
]
\end{axis}
\end{tikzpicture}%
				 \end{figure}
		}
	\mode<beamer>{
		\only<2>{\begin{figure}
					\centering
					\includegraphics[width=0.9\linewidth]{../kmeansMATLAB/EM4Crescent}
				 \end{figure}
		}
		\only<3>{\begin{figure}
					\centering
					\includegraphics[width=0.9\linewidth]{../kmeansMATLAB/EM4CrescentClusterRegions}
				 \end{figure}
		}
		}
	\end{frame}
	
	\section{Dynamic Responsibility}
	\begin{frame}{Responsibility Requirements}
		Necessary items
			\begin{itemize}
				\item Data \( \{\mathcal X,\mathcal T\} = (\bm x^{n},t^{n})\; n=1,\ldots,N \)
				\item Distributions \( f_k(\bm x,\bm\gt_k) \; k=1,\ldots,K\)
				\item Parameter matrix \( F = \left(f_i(\bm x^j,\bm\gt_i)\right)_i^j = (F_i^{j})\)
				\item Mixture probabilities \( \bm\pi_0= \left(\pi_1,\ldots,\pi_K\right)\in S_K \)
			\end{itemize}
		\pause
		\begin{definition}[Probability Simplex]
				\begin{equation*}
				S_K:=\left\{\{\pi_k\}_{k=1}^{K}:0\leq \pi_k\leq 1; \sum_{k=1}^{K}\pi_k =1\right\}.
				\end{equation*}
		\end{definition}
	\end{frame}

	\begin{frame}{Bayes' Rule Estimation of Mixture Probabilities}
		\begin{align*}
		P(t^n=k|\bm x^n, \bm\Theta) &=\dfrac{ P(\bm x^n|t^n=k,\bm\Theta)P(t^n=k|
			 \bm\Theta) }{P(\bm x^n|\bm\Theta)}\\
		&=\dfrac{f_k(\bm x^n,\bm\gt_k)\pi_k} {\sum_{i}\pi_{i}f_{i}(\bm x^n,\bm\gt_i)}
		\end{align*}
	\end{frame}

	\begin{frame}{Responsibility}
		\onslide<1->{Start with rational maps 
			\begin{align*}
			r_i(\bm\pi)=\frac 1N\mathlarger{\sum}_n \frac{\pi_i f_i(\bm x^n,\bm\gt_k)} {\sum_{k}\pi_{k}f_{k}(\bm x^n,\bm\gt_k)}\;\; i=1,\ldots,K
			\end{align*}
		}
		\onslide<2->{
			\begin{definition}<2->[Responsibility Map]
				\begin{equation}\label{map}
				R:S_K\rightarrow S_K: R(\pi_1,\pi_2,\ldots,\pi_K)=(r_1(\bm\pi),r_2(\bm\pi),\ldots,r_K(\bm\pi)).
				\end{equation}
			\end{definition}
			When necessary, write \( R_F(\bm\pi) \) to emphasize dependence on \( K\times N \) parameter matrix \( F \).
		}
	\end{frame}

	\begin{frame}[fragile]{Dynamic Responsibility}
		\begin{algorithm}[H]
			\caption{Dynamic Responsibility Algorithm}\label{ratioAlg}
			\begin{algorithmic}[1]
				\Require $F$ a $K\times N$ matrix
				\Require $\bm\pi_0$, $\ge$ \Comment{$\ge$ creates halt condition}
				\Procedure{Iteration}{$F,\bm\pi_0,\ge$}
				\State $n \gets 1$, $\bm\pi_n \gets R_F(\bm\pi_0)$ 
				\State $orbit \gets \{\bm\pi_0,\bm\pi_1\}$
				\While{$|\bm\pi_n-\bm\pi_{n-1}|>\ge|\bm\pi_n|$}
				\State $\bm\pi_{n+1} \gets R_F(\bm\pi_n)$
				\State $orbit \gets \{\bm\pi_0,\ldots,\bm\pi_{n+1}\}$
				\State $n\gets n+1$
				\EndWhile
				\State \textbf{return} $orbit$ \Comment{at this point $\bm\pi_{n-1}\approx\hat{\bm\pi}$}
				\EndProcedure
			\end{algorithmic}
		\end{algorithm}
	\end{frame}

	\begin{frame}{Lyapunov function}
		For $\bm\pi\in\R_+^K$ the positive orthant of \( \R^K \), let 
		\[\ell_F(\bm\pi)={\dfrac{1}{N}}\sum_{n=1}^{N}\log\left(\sum_{k=1}^{K}\pi_kF_k^{n}\right)\]
		\pause
		\begin{lemm}\label{lyapunovLem}
			\( -\elpi{F} \) is a Lyapunov function for dynamic responsibility. In other words,
			\[ \elpi[R_F(\bm\pi)]{F}\geq \elpi{F} \]
			With equality if and only if \( R_F(\bm\pi)=\bm\pi. \)
		\end{lemm}
		\pause
		Note that if \( F \) has full rank, \( -\ell_F \) is \textit{strictly} convex.
	\end{frame}

	\begin{frame}{Main Theorem}
		\begin{theorem}[Convergence of \DR]
			If \( F \) has full rank, and \( \bm\pi_0\in \op{Int}S_K\) then the orbit \( \bm\pi^n = R_F^n(\bm\pi_0) \) converges to \( \hat{\bm\pi}_F, \) the unique maximizing fixed point of \( \elpi{F} \) on \( S_K.\) Moreover, \( \hat{\bm\pi}_F \)  depends differentiably on \( F \).
		\end{theorem}
	\end{frame}

	\begin{frame}{Dynamic Responsibility Example}
	\only<1>{If $F=(F_{i}^j)$ has linearly independent rows, the interior of $S_K$ converges to one point.
		\begin{center}
			\includegraphics[scale=.4]{Full_rank_seed203.png} 
		\end{center}
	}
	\mode<beamer>
	\only<2>{ In this case, convergence happens very quickly. (about 5 iterations)
			\begin{center}
				\includegraphics[scale=.4]{Full_rank_img_seed203.png} 
			\end{center}
		}
	\mode<all>
	\end{frame}
	
	\section{Responsible Softmax}
	\begin{frame}{Calculate \( F \)}
		\Ryan{Needs motivation\\}
		If \( F = e^{\bm A} \) for some \( \bm A =(A_i^j) \) and \( \mu_i = \ln(\pi_i) \), then 
		\[	r_i(\bm\pi)=\frac 1N\mathlarger{\sum}_n \frac{\pi_i F_i^n} {\sum_{k}\pi_{k}F_{k}^n} = \frac 1N \mathlarger{\sum}_n \alert<3>{\frac{\exp({A_i^n+\mu_i})}{\sum_k \exp({A_k^n+\mu_k})}}\]
		\onslide<2->{
		The softmax function is given by the Gibbs Distribution
		\begin{equation*}
		\gs_i(\bm x) = \alert<3>{\frac{\exp({x_i})}{\sum_k \exp({x_k})}}.
		\end{equation*}
		}
		\onslide<4>{This establishes a connection with modern neural networks.}
	\end{frame}	
	
	\begin{frame}{Neural Network Output}
%		\Ryan{Needs motivation}
		Neural networks take in data, and output guesses of cluster assignments.
		\begin{center}
			\(F=(F_i^{j});\;\) \(\;\bm\pi^n=R_F^n(\bm\pi_0);\;\) \(\;\bm\pi^n\rightarrow\hat{\bm\pi} \) as \( n\rightarrow\oo \)
		\end{center}
		\uncover<+->{\[Y(F,\hat{\bm\pi})=\left(\frac{\hat{\pi}_iF_{i}^{j}}{\sum_{k=1}^{K}\hat{\pi}_kF_{k}^{j}}\right)_{i=1,\ldots,K}^{j=1,\ldots,N}\]
		The entry \( Y_i^j \) represents the probability that \( \bm x^j\) comes from cluster \( i \).\\
		}
		\uncover<+->{For some \( F \), it may be that \( \hat{\bm\pi}_F\in\partial S_K \). To prevent this, stop at some finite \(n=C<\oo\) and use \(Y(F,\bm\pi^C)\) as the output.\\ See \citet{NealHintonEM1999} for inspiration.}
	\end{frame}
	
	\begin{frame}{Layer Diagram}
		\begin{figure}
			\centering
				\usetikzlibrary{decorations.pathreplacing}
\begin{tikzpicture}[scale = .7]
% Layer boxes and labels
\draw  [fill = teal!20, fill opacity = .8, thick ](-7.5,0) rectangle (-6.5,-1);
\node at (-7,-0.5) {\scriptsize$\bm X$};
\draw  [fill = blue!20, fill opacity = .8, thick ](-5.1,2.5) rectangle (-2.9,-3.5);
\node at (-4,-0.5) {\scriptsize\(\bm A =\bm W\cdot \bm X\)};
\draw  [fill = violet!20, fill opacity = .8, thick ](-1.5,1.6) rectangle (-0.14,-2.6);
\node at (-.815,-0.5) {\scriptsize$\bm F = e^{\bm A}$};
\draw  [fill = red!20, fill opacity = .8, thick ](.88,1.5) rectangle (2.55,-2.5);
\node at (1.75,-0.5) {\scriptsize$\bm Y\!\!\left(\!\bm F,\bm\pi^C\!\right)$};
\draw  [fill = orange!50, fill opacity = .8, thick ](4.5,0) rectangle (6,-1);
\node at (5.25,-0.5) {\scriptsize$L(\bm Y,\bm T)$\normalsize};
% Connections between first and second layer
\node (v1) at (-6.5,-0.5) {};
\node (v2) at (-5.1,2.5) {};
\node (v3) at (-5.1,2) {};
\node (v4) at (-5.1,1.5) {};
\node at (-5.5,-0.5) {{$\vdots$}};
\node (v6) at (-5.1,-3) {};
\node (v7) at (-5.1,-3.5) {};
\node (v5) at (-5.1,-2.5) {};
\draw [->] (v1) edge (v2);
\draw [->] (v1) edge (v3);
\draw [->] (v1) edge (v4);
\draw [->] (v1) edge (v5);
\draw [->] (v1) edge (v6);
\draw [->] (v1) edge (v7);
% Connections between second & 3rd layer
\node (v8) at (-2.95,2.5) {};
\node (v11) at (-2.95,2) {};
\node (v15) at (-2.95,-3) {};
\node (v12) at (-2.95,-3.5) {};
\node (v9) at (-1.5,1.5) {};
\node (v10) at (-1.5,1) {};
\node (v14) at (-1.5,-2) {};
\node (v13) at (-1.5,-2.5) {};
\draw [->] (v8) edge (v9);
\draw [->] (v8) edge (v10);
\draw [->] (v11) edge (v9);
\draw [->] (v11) edge (v10);
\draw [->] (v12) edge (v13);
\draw [->] (v12) edge (v14);
\draw [->] (v15) edge (v13);
\draw [->] (v15) edge (v14);
\draw [->,opacity = .5] (v8) edge (v14);
\draw [->,opacity = .5] (v8) edge (v13);
\draw [->,opacity = .5] (v11) edge (v14);
\draw [->,opacity = .5] (v11) edge (v13);
\draw [->,opacity = .5] (v15) edge (v9);
\draw [->,opacity = .5] (v15) edge (v10);
\draw [->,opacity = .5] (v12) edge (v9);
\draw [->,opacity = .5] (v12) edge (v10);
\node at (-2.5,-0.5) {{$\vdots$}};
% Connections between 3rd and 4th layer
\node (v16) at (-0.2,1) {};
\node (v18) at (-0.2,0.5) {};
\node (v20) at (-0.2,-1.5) {};
\node (v22) at (-0.2,-2) {};
\node (v23) at (1,-2) {};
\node (v21) at (1,-1.5) {};
\node (v19) at (1,0.5) {};
\node (v17) at (1,1) {};
\draw [->] (v16) edge (v17);
\draw [->] (v18) edge (v19);
\draw [->] (v20) edge (v21);
\draw [->] (v22) edge (v23);
\node at (0.375,-0.5) {{$\vdots$}};
% Connections between 4th and 5th layer
\node (v25) at (4.5,-0.5) {};
\node (v24) at (2.5,1) {};
\node (v26) at (2.5,0.5) {};
\node (v27) at (2.5,-1.5) {};
\node (v28) at (2.5,-2) {};
\draw [->] (v24) edge (v25);
\draw [->] (v26) edge (v25);
\draw [->] (v27) edge (v25);
\draw [->] (v28) edge (v25);
\node at (3.25,-0.5) {{$\vdots$}};
% Layer labels
\draw [decorate,  decoration={brace,  amplitude=5pt}] (-5.5,3)--(-2.5,3) node[above, xshift =-28.7]{\small Fully Connected Layer};
\draw [decorate,  decoration={brace,  amplitude=5pt}] (-2,2)--(3.5,2) node[above, xshift =-60.7]{{\small Responsible Softmax Layer}\normalsize};
\draw (5.25,.5) node [above]{{\small Loss Function}};
\end{tikzpicture}
		\end{figure}
		\only<2>{\begin{center}\( L(\bm Y,\bm{T}) = -\sum_n\sum_k T_k^n\log(Y_k^n) \)
			\end{center}}
	\end{frame}
	
	\begin{frame}{Backpropagation}
		The goal is to use gradient descent to learn parameters of the network.
		\begin{enumerate}
			\item[]\textbf{Option 1:} Automatic differentiation
			\item[]\textbf{Option 2:} Direct calculation
		\end{enumerate}
		\onslide<2>{
			\begin{align}
				D\hat{\bm\pi}_F&=D_{\bm\pi}R\cdot D\hat{\bm\pi}_F+D_FR\nonumber\\
				D\hat{\bm\pi}_F&=\left(I-D_{\bm\pi}R\right)^{-1}\cdot D_{F}R\label{eqn:dPidF}
			\end{align}
			In practice, equation \eqref{eqn:dPidF} is too much.  An approximation may be used instead.
			\only<2>{\( \left(I-D_{\bm\pi}R\right)^{-1}\approx I+DR+DR^2+\ldots+DR^C \)}
		}
	\end{frame}

	\begin{frame}{Setting the Hyperparameter $C$}
%		\only<1>{Recall the dynamic responsibility example. Convergence happened in 5 steps.}
%		\only<2->{
			Let \( a_n = d(\bm\pi_{n+1},\bm\pi_n) \).
			\begin{figure}
				\centering
				\includegraphics[width=0.7\linewidth]{log_dist_nVn}
				\caption{Plot of \(\log(a_n)\) for several \(F\). Each curve represents a different parameter matrix \(F\).}
			\end{figure}
%		}
	\end{frame}

	\section{Basic Experiments}
	\begin{frame}{Experiments with GMM}
		\only<1>{
			\begin{figure}
%				\centering
				\includegraphics[width=0.65\linewidth]{sample2}
				\caption{A sample of data generated from a GMM to test the \RS layer.}
				\label{fig:sample2}
			\end{figure}
		}
	\mode<beamer>
		\only<2>{
				\begin{figure}
					\centering
					\includegraphics[width=0.9\linewidth]{netClassRegions}
					%				\caption{Classification regions for different Neural nets. Net 1 uses the standard softmax layer. Nets 2 and 3 use a \RS layer with \(C=1,4\) respectively. The last layer has fixed weights.}
					\label{fig:netclassregions}
				\end{figure}
%		}
%		\only<3>{	
			
			\begin{table}
			\centering
			\resizebox{.5\linewidth}{!}{
				\begin{tabular}{|l|l|}
					\hline
					\textbf{Net}  & \textbf{Classification layer}\\ \hline
					Net \#1   & Softmax     \\ \hline
					Net \#2   & Responsibility Softmax; \( C=1 \) \\ \hline
					Net \#3   & Responsibility Softmax; \( C=4 \) \\ \hline
					Net \#4   & Fixed Weight Softmax \\ 
					\hline
				\end{tabular}
			}
		\end{table}
		}
		\only<3>{
				\begin{figure}
					\centering
					\includegraphics[width=0.9\linewidth]{deepandWide8_edit}
%					\caption{Various nets trained on similar data. Nets 2 through 5 have $C$ values of $1,4,8,16$ respectively.}
%					\label{fig:deepandwide8edit}
				\end{figure}
				\begin{table}
				\centering
				\resizebox{.45\linewidth}{!}{
					\begin{tabular}{|l|l|}
						\hline
						\textbf{Net}  & \textbf{Classification layer}\\ \hline
						Net \#1   & Softmax     \\ \hline
						Net \#2   & Responsibility Softmax; \( C=1 \) \\ \hline
						Net \#3   & Responsibility Softmax; \( C=4 \) \\ \hline
						Net \#4   & Responsibility Softmax; \( C=8 \) \\ \hline
						Net \#5   & Responsibility Softmax; \( C=16 \) \\ \hline
						Net \#6   & Fixed Weight Softmax \\ 
						\hline
					\end{tabular}
				}
			\end{table}
		}
	\mode<all>
	\end{frame}
	
	\begin{frame}{Non-Gaussian Data Set}
		\only<1>{Recall the performance of the EM algorithm on Crescent data
			\begin{figure}
				\centering
				\includegraphics[width=0.9\linewidth]{../kmeansMATLAB/EM4Crescent}
			\end{figure}
		}
		\mode<beamer>
		\only<2>{
			\begin{figure}
				\centering
				\includegraphics[width=0.9\linewidth]{../kmeansMATLAB/RSCrescentClassification}
				\caption{Classification regions for neural nets trained on crescent data. Hyperparametes are as in GMM example.}
			\end{figure}
		}
		\only<3>{
	
%		}
%		\only<4>{
%	\begin{columns}
%		\begin{column}{.62\textwidth}
			\begin{figure}
%				\centering
				\includegraphics[scale = .28]{../kmeansMATLAB/RSCrescentConfusion}
%				\caption{Confusion matrices for neural nets with \RS layers trained on crescent data.}
			\end{figure}
%		\end{column}
%		
%		\begin{column}{.38\textwidth}
			\begin{table}
%				\centering
				\resizebox{.45\linewidth}{!}{
					\begin{tabular}{|l|l|}
						\hline
						\textbf{Net}  & \textbf{Classification layer}\\ \hline
						Net \#1   & Softmax     \\ \hline
						Net \#2   & Responsibility Softmax \( C=1 \) \\ \hline
						Net \#3   & Responsibility Softmax \( C=4 \) \\ \hline
						Net \#4   & Fixed Weight Softmax \\ 
						\hline
					\end{tabular}
				}
			\end{table}
%		\end{column}
%	\end{columns}
		}
	

\mode<all>
	\end{frame}

	\begin{frame}{Experiments with MNIST}
		\begin{figure}
			% This file was created by matlab2tikz.
%
%The latest updates can be retrieved from
%  http://www.mathworks.com/matlabcentral/fileexchange/22022-matlab2tikz-matlab2tikz
%where you can also make suggestions and rate matlab2tikz.
%
\begin{tikzpicture}[scale = .45]

\begin{axis}[%
width=2.573in,
height=2.207in,
at={(0.999in,3.777in)},
scale only axis,
point meta min=0,
point meta max=27.25,
axis on top,
xmin=0.5,
xmax=9.5,
y dir=reverse,
ymin=0.5,
ymax=9.5,
axis background/.style={fill=white},
title style={font=\bfseries},
title={MNIST net 1 Confusion},
legend style={legend cell align=left, align=left, draw=white!15!black}
]
\addplot [forget plot] graphics [xmin=0.5, xmax=9.5, ymin=0.5, ymax=9.5] {benfordConfusion2-1.png};
\end{axis}

\begin{axis}[%
width=2.573in,
height=2.207in,
at={(4.384in,3.777in)},
scale only axis,
point meta min=0,
point meta max=34.15,
axis on top,
xmin=0.5,
xmax=9.5,
y dir=reverse,
ymin=0.5,
ymax=9.5,
axis background/.style={fill=white},
title style={font=\bfseries},
title={MNIST net 2 Confusion},
legend style={legend cell align=left, align=left, draw=white!15!black}
]
\addplot [forget plot] graphics [xmin=0.5, xmax=9.5, ymin=0.5, ymax=9.5] {benfordConfusion2-2.png};
\end{axis}

\begin{axis}[%
width=2.573in,
height=2.207in,
at={(0.999in,0.712in)},
scale only axis,
point meta min=0,
point meta max=29.25,
axis on top,
xmin=0.5,
xmax=9.5,
y dir=reverse,
ymin=0.5,
ymax=9.5,
axis background/.style={fill=white},
title style={font=\bfseries},
title={MNIST net 3 Confusion},
legend style={legend cell align=left, align=left, draw=white!15!black}
]
\addplot [forget plot] graphics [xmin=0.5, xmax=9.5, ymin=0.5, ymax=9.5] {benfordConfusion2-3.png};
\end{axis}

\begin{axis}[%
width=2.573in,
height=2.207in,
at={(4.384in,0.712in)},
scale only axis,
point meta min=0,
point meta max=37.55,
axis on top,
xmin=0.5,
xmax=9.5,
y dir=reverse,
ymin=0.5,
ymax=9.5,
axis background/.style={fill=white},
title style={font=\bfseries},
title={MNIST net 4 Confusion},
legend style={legend cell align=left, align=left, draw=white!15!black}
]
\addplot [forget plot] graphics [xmin=0.5, xmax=9.5, ymin=0.5, ymax=9.5] {benfordConfusion2-4.png};
\end{axis}

\end{tikzpicture}%
		\end{figure}
	\end{frame}
	

	\begin{frame}{Conclusions}
		We have shown that:
		\begin{itemize}
			\item \textbf{Dynamic responsibility} has nice convergence properties; converges to a MLE.
			\item The \textbf{\RS} layer uses \DR and gives cluster responsibilities.
			\item Using a \RS layer gives better results when working with imbalanced data. It also works when we do not have distributions for the mixture populations.
%			\item The hyperparameter \( C \) should be small in general.
		\end{itemize}
	\end{frame}

	\begin{frame}{Future Work}
		Future work:
		\begin{itemize}
			\item Use \RS with other neural nets, LSTM, VAE, Deductron etc.
			\item Use \RS with nonparametric models (\textit{e.g.} Gaussian processes).
			\item Obtain constructive bounds on convergence rates.
			\item Explore the relationship between hessian of \( \ell_F \) and Fisher Information matrix.
		\end{itemize}
	\end{frame}
	
	\begin{frame}{References}
	    \bibliographystyle{apalike}
		\bibliography{dissertationBib}
%		\printbibliography
	\end{frame}
\appendix
\begin{frame}{EM algorithm for GMM}
	\mode<beamer>
	\only<1>{\begin{align*}
		&\mathcal{X} = \{\bm x^{1},\bm x^{2},\ldots,\bm x^{N}\}  \\
		&f_k(\bm x) \sim \mathcal{N}(\bm \mu_k,\bm \Sigma_k),\;\; k=1,\ldots,K\\
		&p(t^n=k) = \pi_{k}, \;\; \sum_{k} \pi_{k} = 1	
		\end{align*}
	}
\mode<all>
	\only<2>{\begin{enumerate}
			\item \textbf{Expectation} step: Set 
			\begin{equation*}
			\rho_k^n = \frac{\pi_k f_k(\bm x^{(n)})}{\sum_{j=1}^K \pi_j f_j(\bm x^{(n)})}
			\end{equation*}
			\item \textbf{Maximization} step: Set
			\begin{align*}
			&N_k = \sum_{n=1}^{N}\rho_k^{n},&	\bm \mu_{k}^{new} &= \dfrac{1}{N_k}\sum_{n=1}^{N} \rho_k^{n}\bm x^{(n)}\\
			&\pi_k^{new} = \dfrac{N_k}{N},&			
			\bm \Sigma_k^{new} &= \dfrac{1}{N_k}\sum_{n=1}^{N}\rho_k^{n} (\bm x^{(n)}-\bm \mu_{k}^{new})(\bm x^{(n)}-\bm \mu_{k}^{new})^{\intercal}
			\end{align*}
			\item Repeat steps 1 and 2 until convergence.
		\end{enumerate}
		See \citet{BishopBook} for more details.
	}
\end{frame}

\begin{frame}{Proof of Lyapunov Lemma}
\mode<beamer>
	\only<1>{
		\begin{lemm}\label{diffDef}
			The map $R_F(\bm\pi)$ as defined in equation \eqref{map} satisfies
			\[R_F(\bm\pi)=\left(\pi_i\cdot\eval{\frac{\partial\ell_F}{\partial\pi_i}}_{\bm\pi}\right)_{1\leq i\leq K}\]
		\end{lemm}	
	}
\mode<all>
	\only<2-5>{		
		\begin{align*}
		\ell_{F}(R_F(\bm\pi))-\elpi{F} &= \frac{1}{N}\mathlarger{\mathlarger{\sum}_{n=1}^{N}}\log\left\{\frac{\sum_{i=1}^{K}\pi_iF_{i}^{n}\pdv{\ell}{\pi_i}}{\sum_{k=1}^{K}\pi_kF_{k}^{n}}\right\}\\
		&\geq\alert<3>{\mathlarger{\sum}_{n=1}^{N}\sum_{i=1}^{K}\frac 1N \frac{\pi_iF_{i}^{n}} {\sum_{k=1}^{K}\pi_kf_{kn}} \log\left(\pdv{\ell}{\pi_i}\right)}\\
		&=\alert<4>{\mathlarger{\sum}_{i=1}^{K}\sum_{n=1}^{N}}\frac 1N \frac{\pi_iF_{i}^{n}} {\sum_{k=1}^{K}\pi_kf_{kn}} \log\left(\pdv{\ell}{\pi_i}\right)\\
		&= \alert<5>{\sum_{i=1}^{K} r_i(\bm\pi)\log\left(\frac{r_i(\bm\pi)} {\pi_i}\right)\geq 0}
		\end{align*}
	}
\end{frame}

\begin{frame}[fragile]{Confusion for GMM data}
			\only<1>{
				\begin{table}[ht]
					\renewcommand{\arraystretch}{1.4}
					\centering
					%\captionsetup[subtable]{position=top}
				\subcaptionbox{Confusion table for GMM Net \#1.\label{table:GMMconfusion1}}{
				\begin{tabular}{|c|c|c|c|c|}
					\hline
					30.253±.001 & 0.0 & .027±.001   & 0.0 & 0.0         \\ \hline
					1.680±.000  & 0.0 & 0.0         & 0.0 & 0.0         \\ \hline
					.328±.004   & 0.0 & 32.206±.008 & 0.0 & .706±.006   \\ \hline
					0.0         & 0.0 & .033±.001   & 0.0 & 3.207±.001  \\ \hline
					0.0         & 0.0 & .021±.001   & 0.0 & 31.539±.001 \\ \hline
				\end{tabular}
				}
				\caption[Confusion matrices with error estimates for GMM nets \#1-\#4]{The nets were tested on a set of samples drawn independently from the training set. Values are reported as percentages for clarity. Test data sample size \(N=2500\) for all runs. Error intervals are 95\% confidence standard error. An entry of \( 0.0 \) indicates that all values were zero to 3 decimal places.}\label{table:GMMconfusion}
				\end{table}
			}
	\mode<beamer>
			\only<2>{
				\begin{table}[ht]
					\renewcommand{\arraystretch}{1.4}
					\centering
					%\captionsetup[subtable]{position=top}
				\subcaptionbox{Confusion table for GMM Net \#2.\label{table:GMMconfusion2}}{
				\begin{tabular}{|c|c|c|c|c|}
					\hline
					30.165±.004 & .101±.004 & .014±.001   & 0.0       & 0.0         \\ \hline
					1.616±.003  & .063±.003 & 0.0         & 0.0       & 0.0         \\ \hline
					.398±.006   & .114±.003 & 31.739±.010 & .330±.008 & .659±.009   \\ \hline
					0.0         & 0.0       & .031±.001   & .333±.012 & 2.875±.012  \\ \hline
					0.0         & 0.0       & .012±.000   & .082±.004 & 31.466±.004 \\ \hline
				\end{tabular}
				}
				\caption[Confusion matrices with error estimates for GMM nets \#1-\#4]{The nets were tested on a set of samples drawn independently from the training set. Values are reported as percentages for clarity. Test data sample size \(N=2500\) for all runs. Error intervals are 95\% confidence standard error. An entry of \( 0.0 \) indicates that all values were zero to 3 decimal places.}\label{table:GMMconfusion2}
				\end{table}
			}
			\only<3>{
				\begin{table}[ht]
					\renewcommand{\arraystretch}{1.4}
					\centering
					%\captionsetup[subtable]{position=top}
				\subcaptionbox{Confusion table for GMM Net \#3.\label{table:GMMconfusion3}}{
				\begin{tabular}{|c|c|c|c|c|}
					\hline
					29.897±.010 & .374±.010 & .009±.001   & .000±.001  & 0.0         \\ \hline
					1.273±.011  & .406±.011 & .001±.001   & 0.0        & 0.0         \\ \hline
					.658±.016   & .595±.017 & 29.916±.036 & 1.329±.031 & .743±.018   \\ \hline
					0.0         & .000±.001 & .013±.001   & 1.221±.027 & 2.006±.027  \\ \hline
					0.0         & 0.0       & 0.0         & .340±.009  & 31.220±.009 \\ \hline
				\end{tabular}
				}
				\caption[Confusion matrices with error estimates for GMM nets \#1-\#4]{The nets were tested on a set of samples drawn independently from the training set. Values are reported as percentages for clarity. Test 	data sample size \(N=2500\) for all runs. Error intervals are 95\% confidence standard error. An entry of \( 0.0 \) indicates that all values were zero to 3 decimal places.}\label{table:GMMconfusion3}
				\end{table}
			}
			\only<4>{
				\begin{table}[ht]
					\renewcommand{\arraystretch}{1.4}
					\centering
					%\captionsetup[subtable]{position=top}
				\subcaptionbox{Confusion table for GMM Net \#4.\label{table:GMMconfusion4}}{
				\begin{tabular}{|c|c|c|c|c|}
					\hline
					26.842±.035 & 3.438±.035 & 0.0         & 0.0        & 0.0         \\ \hline
					.044±.006   & 1.636±.006 & 0.0         & 0.0        & 0.0         \\ \hline
					.075±.003   & 1.841±.024 & 28.737±.037 & 2.540±.025 & .047±.002   \\ \hline
					0.0         & 0.0        & .027±.001   & 3.122±.004 & .092±.004   \\ \hline
					0.0         & 0.0        & 0.0         & 2.463±.023 & 29.097±.023 \\ \hline
				\end{tabular}
			}
			\caption[Confusion matrices with error estimates for GMM nets \#1-\#4]{The nets were tested on a set of samples drawn independently from the training set. Values are reported as percentages for clarity. Test data sample size \(N=2500\) for all runs. Error intervals are 95\% confidence standard error. An entry of \( 0.0 \) indicates that all values were zero to 3 decimal places.}\label{table:GMMconfusion4}
			\end{table}
		}
	\mode<all>
	\end{frame}

	\begin{frame}{Per class precision and recall for GMM data}
		\only<1>{
		\begin{table}[ht]
			\renewcommand{\arraystretch}{1.4}
			\centering
			\subcaptionbox{Precision and Recall table for GMM Net \#1.\label{table:GMMprecRec1}}[.45\linewidth]{
				\begin{tabular}{|c|c|c|}
					\hline
					\multicolumn{3}{|c|}{\textbf{GMM Net 1}} \\ \hline
					Class      & Precision      & Recall     \\ \hline
					1          & 0.936          & 0.999      \\ \hline
					2          & 0.000          & 0.000      \\ \hline
					3          & 0.998          & 0.966      \\ \hline
					4          & 0.000          & 0.000      \\ \hline
					5          & 0.979          & 0.999      \\ \hline
				\end{tabular}
			}
			\caption[Per class precision and recall for GMM nets \#1-\#4]{This table shows per class precision and recall for GMM nets trained and tested on the same data as in table \ref{table:GMMconfusion}}\label{table:GMMprecRec}
		\end{table}
	}
	\mode<beamer>
	\only<2>{
		\begin{table}[ht]
			\renewcommand{\arraystretch}{1.4}
			\centering
			\subcaptionbox{Precision and Recall table for GMM Net \#2.\label{table:GMMprecRec2}}[.45\linewidth]{
				\begin{tabular}{|c|c|c|}
					\hline
					\multicolumn{3}{|c|}{\textbf{GMM Net 2}} \\ \hline
					Class      & Precision      & Recall     \\ \hline
					1          & 0.935          & 0.997      \\ \hline
					2          & 0.209          & 0.027      \\ \hline
					3          & 0.998          & 0.953      \\ \hline
					4          & 0.401          & 0.090      \\ \hline
					5          & 0.898          & 0.997      \\ \hline
				\end{tabular}
			}
			\caption[Per class precision and recall for GMM nets \#1-\#4]{This table shows per class precision and recall for GMM nets trained and tested on the same data as in table \ref{table:GMMconfusion}}\label{table:GMMprecRec2}
		\end{table}
	}
	\only<3>{
		\begin{table}[ht]
			\renewcommand{\arraystretch}{1.4}
			\centering
			\subcaptionbox{Precision and Recall table for GMM Net \#3.\label{table:GMMprecRec3}}[.45\linewidth]{
				\begin{tabular}{|c|c|c|}
					\hline
					\multicolumn{3}{|c|}{\textbf{GMM Net 3}} \\ \hline
					Class      & Precision      & Recall     \\ \hline
					1          & 0.934          & 0.988      \\ \hline
					2          & 0.279          & 0.194      \\ \hline
					3          & 0.999          & 0.894      \\ \hline
					4          & 0.385          & 0.364      \\ \hline
					5          & 0.918          & 0.989      \\ \hline
				\end{tabular}
			}
			\caption[Per class precision and recall for GMM nets \#1-\#4]{This table shows per class precision and recall for GMM nets trained and tested on the same data as in table \ref{table:GMMconfusion}}\label{table:GMMprecRec3}
		\end{table}
	}
	\only<4>{
		\begin{table}[ht]
			\renewcommand{\arraystretch}{1.4}
			\centering
			\subcaptionbox{Precision and Recall table for GMM Net \#4.\label{table:GMMprecRec4}}[.45\linewidth]{
				\begin{tabular}{|c|c|c|}
					\hline
					\multicolumn{3}{|c|}{\textbf{GMM Net 4}} \\ \hline
					Class      & Precision      & Recall     \\ \hline
					1          & 0.988          & 0.930      \\ \hline
					2          & 0.289          & 0.882      \\ \hline
					3          & 0.999          & 0.868      \\ \hline
					4          & 0.380          & 0.969      \\ \hline
					5          & 0.996          & 0.922      \\ \hline
				\end{tabular}
			}
			\caption[Per class precision and recall for GMM nets \#1-\#4]{This table shows per class precision and recall for GMM nets trained and tested on the same data as in table \ref{table:GMMconfusion}}\label{table:GMMprecRec4}
		\end{table}
	}
\mode<all>
\end{frame}

\end{document}