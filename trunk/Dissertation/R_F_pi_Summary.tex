\documentclass[12pt,letterpaper]{amsbook}
%%%%%%%%%%%%%%%%%%%%%%%%%%%%%%%%%%%%%%%%%%%%%%%%%%%%%%%%%%%%%%%%%%%%%%%%%%%%%%%%%%%%%%%%%%%%%%%%%
\setlength{\topmargin}{-0.5in}        %%%  This sets all the spacing stuff to use the page more
\setlength{\oddsidemargin}{-0.3in}    %%%  efficiently than the normal "article" setup would.
\setlength{\evensidemargin}{-0.3in}   %%%  It's OK to play with these some.
\setlength{\textheight}{9.5in}     %%%
\setlength{\textwidth}{7.2in}     %%%
\setlength{\headsep}{.2in}          %%%
\setlength{\headheight}{0.2in}       %%%
\setlength{\footskip}{.1in}         %%%

%%%%%%%%%%%%%%%%%%%%%%%%%%%%%%%%%%%%%%%%%%%%%%%%%%%%%%%%%%%%%%%%%%%%%%%%%%%%%%%%%%%%%%%%%%%%%%%%
\newif\iffinal\finalfalse
\newif\ifdraft
	\iffinal
		\draftfalse
	\else
		\drafttrue
\fi

\usepackage{amsmath, amsthm, amssymb, amsfonts, mathtools, mathrsfs, bm, bbm}

\usepackage[square, sort, comma, numbers]{natbib}
\usepackage{ifthen}
\usepackage{relsize}
\usepackage{graphicx}  % Required for including images
\usepackage{booktabs} % Top and bottom rules for tables
%\usepackage{caption}
\usepackage{subcaption}
\usepackage{textcomp}
\usepackage{physics}
\usepackage{array}
\usepackage{tikz,pgfplots}
\usetikzlibrary{automata, positioning}
\tikzset{every state/.style={minimum size=0pt}}
%\usepgfplotslibrary{external} 
%\tikzexternalize

\usepackage[toc]{appendix}
\usepackage{fancyvrb}
\usepackage{listings}

\usepackage{pdfpages}

%Settings for the listings package. Setup is currently for MATLAB defaults
\definecolor{dkGrey}{gray}{.2}
\lstset{% general command to set parameter(s)
	basicstyle          = \scriptsize\singlespace\ttfamily, % print whole listing small
	language			= matlab,
	numbers             = left,
	numberstyle         = \tiny,
	stepnumber          = 1,
	keywordstyle        =\color{blue}\bfseries,
	identifierstyle     =\color{dkGrey}, 
	commentstyle        =\color{green}, 
	stringstyle         =\color{purple},
	showspaces          = false,
	showstringspaces    = false,
	showtabs            = false,
	frame               = single,
	tabsize             = 2,
%	captionpos          = b,
	breaklines          = true,
	breakatwhitespace   = false,
%	fancyvrb            = true,
	}

\usepackage{placeins}
\usepackage{float}
\usepackage[linktoc = page,
			hidelinks
			]{hyperref}

%\usepackage[lofdepth,lotdepth]{subfig}
% Packages to have pseudocode must be included after hyperref!!!

\usepackage{setspace}
\usepackage{algorithm}
\usepackage{algpseudocode}
\usepackage{chngcntr}


\DeclareMathAlphabet{\mathpzc}{OT1}{pzc}{m}{it}

\def \ga{\alpha} \def \gb{\beta}  \def \gd{\delta} \def \gw{\omega} \def \gW{\Omega}
\def \gt{\theta} \def \gp{\phi} \def \ge{\epsilon} \def \gs{\sigma}
\def \gl{\lambda} \def \gz{\zeta} \def \gr{\rho} \def \GT{\Theta}

%\def \gg{\gamma}

\def \BF{\mathbb{F}}
\def \<{\langle} \def \>{\rangle}
\newcommand{\overbar}[1]{\mkern 1.5mu\overline{\mkern-1.5mu#1\mkern-1.5mu}\mkern 1.5mu}

\newcommand{\oo}{\infty}

\newcommand{\fr}[1]{\mathfrak{#1}}
\renewcommand{\op}[1]{\operatorname{#1}}
\newcommand{\Unit}[1]{{#1}^{\times}}
\newcommand{\cc}[1]{\overline{#1}}
\newcommand{\KX}[1]{\ifthenelse{\equal{#1}{1}}{$K[x]$}{$K[x_1,x_2,\ldots,x_{#1}]$}}

\newcommand{\Ryan}[1]{\ifdraft\textcolor{red}{Ryan says: #1}\fi}
\newcommand{\Marek}[1]{\ifdraft\textcolor{blue}{Marek says: #1}\fi}
\newcommand{\Dave}[1]{\ifdraft\textcolor{plum}{Dave says: #1}\fi}
\newcommand{\Clay}[1]{\ifdraft\textcolor{green}{Clay says: #1}\fi}
\newcommand{\Rob}[1]{\ifdraft\textcolor{orange}{Robert says: #1}\fi}

\newcommand{\R}{\mathbb R}
\newcommand{\HHH}{\mathbb H}
\newcommand{\RR}{\mathcal R}
\newcommand{\SSS}{\mathcal S}
\newcommand{\SSSS}{\mathfrak S}
\newcommand{\CC}{\textrm{C}}
\newcommand{\charr}{\textrm{char}}
\newcommand{\Supp}{\textrm{Supp}}
\newcommand{\CM}{\mathcal M}
\newcommand{\HH}{\textrm{H}}
\newcommand{\htt}{\textrm{ht}}
\newcommand{\Imm}{\textrm{Im}}
\newcommand{\ds}{\displaystyle}

\newcommand{\agmax}{\textrm{argmax}}
\DeclareRobustCommand{\pder1}[2]{\frac{\partial {#1}}{\partial {#2}}}

\newcommand{\N}{\mathbb N}
\newcommand{\ac}{\mathfrak{a}}
\newcommand{\mc}{\mathfrak{m}}
\newcommand{\Pc}{\mathfrak{P}}
\newcommand{\pc}{\mathfrak{p}}

%\newcommand{\MM}{\textrm{M}}
\newcommand{\PG}{\textrm{P}\Gamma_1}
\newcommand{\BA}{\mathbb{A}}
\newcommand{\QQ}{\mathbb Q}
\newcommand{\ZZ}{\mathbb Z}
\newcommand{\KK}{\mathbb K}
\newcommand{\BC}{\mathbb C}
\newcommand{\Prj}{\mathbb P}
\newcommand{\ri}{\mathcal{O}}
\newcommand{\FS}{\mathfrak F}
\newcommand{\Norm}{\textrm{N}}
\newcommand{\End}{\textrm{End}}
\newcommand{\Cl}{\textrm{Cl}}
\newcommand{\Qbar}{\overline{\Q}}

\DeclareMathOperator{\Gal}{Gal}
\DeclareMathOperator{\Frob}{Frob}
\DeclareMathOperator{\GL}{GL}
\DeclareMathOperator{\SL}{SL}
\DeclareMathOperator{\PGL}{PGL}
\DeclareMathOperator{\Aut}{Aut}
\DeclareMathOperator{\Hom}{Hom}
\DeclareMathOperator{\Stab}{Stab}
\DeclareMathOperator{\Fix}{Fix}
\DeclareMathOperator{\Inn}{Inn}
\DeclareMathOperator{\Bil}{Bil}
\DeclareMathOperator{\disc}{Disc}


\DeclareFontFamily{U}{wncy}{}
\DeclareFontShape{U}{wncy}{m}{n}{<->wncyr10}{}
\DeclareSymbolFont{mcy}{U}{wncy}{m}{n}
\DeclareMathSymbol{\Sh}{\mathord}{mcy}{"58} 

\theoremstyle{definition}
\newtheorem{defn}{Definition}[chapter]%[section]

\theoremstyle{remark}
\newtheorem{rk}[defn]{Remark}
\newtheorem{ex}{Exercise}
\newtheorem{eg}[defn]{Example}
\newtheorem*{soln}{Solution}
\newtheorem{experiment}[defn]{Experiment}
\newtheorem{calc}[defn]{Calculation}

\theoremstyle{plain}
\newtheorem{thm}{Theorem}[chapter]
\newtheorem{lemm}[thm]{Lemma}
\newtheorem{prop}[thm]{Proposition}
\newtheorem{cor}[thm]{Corollary}

%\numberwithin{defn}{section}

\newcommand{\Matrix}[1]{\begin{bmatrix} #1 \end{bmatrix}}
 \newcommand{\Vector}[1]{\begin{pmatrix} #1 \end{pmatrix}}

% \newcommand*{\norm}[1]{\mathopen\| #1 \mathclose\|}% use instead of $\|x\|$
% \newcommand*{\abs}[1]{\mathopen| #1 \mathclose|}% use instead of $\|x\|$
 \newcommand*{\normLR}[1]{\left\| #1 \right\|}% use instead of $\|x\|$

 \newcommand*{\SET}[1]  {\ensuremath{\mathcal{#1}}}
 \newcommand*{\FUN}[1]  {\ensuremath{\mathcal{#1}}}
 \newcommand*{\MAT}[1]  {\ensuremath{\boldsymbol{#1}}}
 \newcommand*{\VEC}[1]  {\ensuremath{\boldsymbol{#1}}}
 \newcommand*{\CONST}[1]{\ensuremath{\mathit{#1}}}
 
% \newcommand{\bra}{\langle}
% \newcommand{\ket}{\rangle}
 %%%%%%%%%%%%%%%%%%%%%%%%%%%%%%%%%%%%%%%%%%%%%%%%%%%%%%%
 % commands for quick dissertation writing
 %%%%%%%%%%%%%%%%%%%%%%%%%%%%%%%%%%%%%%%%%%%%%%%%%%%%%%%
 \newcommand*{\Rpi}[2][\bm\pi]{ \(R_{#2}({#1})\)} 
 \newcommand*{\elpi}[2][\bm\pi]{\ell_{#2}({#1})}
 \newcommand*{\RS}{responsible softmax }
 \newcommand*{\DR}{dynamic responsibility }
 %%%%%%%%%%%%%%%%%%%%%%%%%%%%%%%%%%%%%%%%%%%%%%%%%%%%%%%

 \DeclareMathOperator*{\argmax}{arg\,max}
 \DeclareMathOperator*{\diag}{diag}
 \DeclareMathOperator*{\argmin}{arg\,min}
 \DeclareMathOperator*{\vectorize}{vec}
 \DeclareMathOperator*{\reshape}{reshape}

 %-----------------------------------------------------------------------------
 % Differentiation
 \newcommand*{\Nabla}[1]{\nabla_{\!#1}}

 \renewcommand*{\d}{\mathrm{d}}
% \newcommand*{\dd}{\partial}

 \newcommand*{\At}[2]{\ensuremath{\left.#1\right|_{#2}}}
 \newcommand*{\AtZero}[1]{\At{#1}{\pp=\VEC 0}}

 \newcommand*{\diffp}[2]{\ensuremath{\frac{\dd #1}{\dd #2}}}
 \newcommand*{\diffpp}[3]{\ensuremath{\frac{\dd^2 #1}{\dd #2 \dd #3}}}
 \newcommand*{\diffppp}[4]{\ensuremath{\frac{\dd^3 #1}{\dd #2 \dd #3 \dd #4}}}
 \newcommand*{\difff}[2]{\ensuremath{\frac{\d #1}{\d #2}}}
 \newcommand*{\diffff}[3]{\ensuremath{\frac{\d^2 #1}{\d #2 \d #3}}}
 \newcommand*{\difffp}[3]{\ensuremath{\frac{\dd\d #1}{\d #2 \dd #3}}}
 \newcommand*{\difffpp}[4]{\ensuremath{\frac{\dd^2\d #1}{\d #2 \dd #3 \dd #4}}}

 \newcommand*{\diffpAtZero}[2]{\ensuremath{\AtZero{\diffp{#1}{#2}}}}
 \newcommand*{\diffppAtZero}[3]{\ensuremath{\AtZero{\diffpp{#1}{#2}{#3}}}}
 \newcommand*{\difffAt}[3]{\ensuremath{\At{\difff{#1}{#2}}{#3}}}
 \newcommand*{\difffAtZero}[2]{\ensuremath{\AtZero{\difff{#1}{#2}}}}
 \newcommand*{\difffpAtZero}[3]{\ensuremath{\AtZero{\difffp{#1}{#2}{#3}}}}
 \newcommand*{\difffppAtZero}[4]{\ensuremath{\AtZero{\difffpp{#1}{#2}{#3}{#4}}}}

 %-----------------------------------------------------------------------------
 % Defined
 % How should the defined operator look like (:= or ^= ==)
 % (I want back my :=, it is so much better than ^= because (1) it has a
 % direction and (2) everyone here uses it.)
 %
 % Use :=
 \newcommand*{\defined}{\ensuremath{\mathrel{\mathop{:}}=}}
 \newcommand*{\definedRight}{\ensuremath{=\mathrel{\mathop{:}}}}
 % Use ^=
 %\newcommand*{\defined}{\ensuremath{\triangleq}}
 %\newcommand*{\definedRight}{\ensuremath{\triangleq}}
 % Use = with three bars
 %\newcommand*{\defined}{\ensuremath{?}}
 %\newcommand*{\definedRight}{\ensuremath{?}}

%-----------------------------------------------------------------------------
 % Domains
 \newcommand*{\D}{\mathcal{D}}
 \newcommand*{\I}{\mathcal{I}}

 %-----------------------------------------------------------------------------
 % Texture coordinates
 \newcommand*{\rr}{\VEC{r}}

 %-----------------------------------------------------------------------------
 % Parameters
 \newcommand*{\pt}{\VEC{\tau}}
 \newcommand*{\pr}{\VEC{\rho}}
 \newcommand*{\pp}{\VEC{p}}
% \newcommand*{\qq}{\VEC{q}}
 \newcommand*{\xx}{\VEC{x}}
 \newcommand*{\deltaq}{\Delta \qq}
 \newcommand*{\deltap}{\Delta \pp}
 \newcommand*{\zz}{\VEC{z}}
 \newcommand*{\pa}{\VEC{\alpha}}
 \newcommand*{\qa}{\VEC{\alpha}}
% \newcommand*{\pb}{\VEC{\beta}}

 %-----------------------------------------------------------------------------
 % Optimal appearance parameters
 \newcommand*{\pbh}[1]{\ensuremath{\hat{\pb}({#1})}}

 %-----------------------------------------------------------------------------
 % Warp basis
 \newcommand*{\M}[1]{\ensuremath{M({#1})}}
 \newcommand*{\LL}[1]{\ensuremath{L({#1})}}

 %-----------------------------------------------------------------------------
 % Matrices of the texture model
 \newcommand*{\AM}[1]{\ensuremath{\Lambda(#1)}}               % Lambda(beta) 
 \newcommand*{\AMr}[2]{\ensuremath{\Lambda(#1; #2)}}        % Lambda(r, beta)

 \newcommand*{\As}{A}         % Continuous Basis symbol
 \newcommand*{\afs}{a}        % Continuous mean symbol
 \newcommand*{\Ab}[1]{\As(#1)}         % Continuous Basis
 \newcommand*{\af}[1]{\afs(#1)}        % Continuous mean


 %-----------------------------------------------------------------------------
 % Matrices of the shape model
 \newcommand*{\MU}{\VEC{\mu}}
 \newcommand*{\MM}{\MAT{M}}

 %-----------------------------------------------------------------------------
 %% The project out matrix and operator
 \newcommand*{\INT}{\MAT{P}}
 \newcommand*{\INTf}{P}

 %-----------------------------------------------------------------------------
 % The identity matrix
 \newcommand*{\EYEtwo}{\Matrix{1 & 0\\0&1}}
 \newcommand*{\EYE}{\MAT E}
 \newcommand*{\EYEf}{E}

 % Wether to use subscripts or brackets for some function arguments
 % can be decided by commenting out the corresponding functions underneath
 %-----------------------------------------------------------------------------
 % Mapping
 \newcommand*{\Cs}[1]{\ensuremath{C^{#1}}} % C symbol
 \newcommand*{\C}[2]{\ensuremath{C^{#1}(#2)}} % Use C with brackets

 %-----------------------------------------------------------------------------
 % Objective function
 \newcommand*{\Fs}{\ensuremath{F}}              % F symbol
 \newcommand*{\F}[1]{\ensuremath{\Fs(#1)}}       % Use F with brackets    F(q)

 %-----------------------------------------------------------------------------
 % Approximated objective functions
 \newcommand*{\FFs}{\tilde{F}}                     % ~F symbol
 \newcommand*{\FF}[1]{\ensuremath{\FFs(#1)}}       % Use ~F with brackets    F(q)

 %-----------------------------------------------------------------------------
 % residual function
 \newcommand*{\es}{\ensuremath{f}}              % R symbol

 \newcommand*{\e}[1]{\ensuremath{\es(#1)}}         % R(q)
 \newcommand*{\er}[2]{\ensuremath{\es(#1; #2)}}    % R(r; q)

 %-----------------------------------------------------------------------------
 % Approximated residual functions
 \newcommand*{\ees}{\tilde{f}}                       % ~R symbol
 \newcommand*{\ee}[1]{\ensuremath{\ees(#1)}}       % ~R(q)
 \newcommand*{\eer}[2]{\ensuremath{\ees(#2; #1)}}  % ~R(r; q)

 %-----------------------------------------------------------------------------
 % Warps
 \newcommand*{\Vs}{\ensuremath{V}}
 \newcommand*{\VLins}{\ensuremath{\Vs^{\text{Ortho}}}}
 \newcommand{\VModels}{\ensuremath{\Vs^{\text{Model}}}}
 \newcommand*{\Ws}{\ensuremath{W}}

 \newcommand{\V}[1]{\ensuremath{\Vs(#1)}}
 \newcommand{\VModel}[1]{\ensuremath{\VModels(#1)}}
 \newcommand{\Vr}[2]{\ensuremath{\Vs(#1; #2)}}
 \newcommand{\VInvr}[2]{\ensuremath{\Vs^{-1}(#1; #2)}}
 \newcommand{\VrLin}[2]{\ensuremath{\VLins(#1; #2)}}
 \newcommand{\W}[1]{\ensuremath{\Ws(#1)}}
 \newcommand{\Winv}[1]{\ensuremath{\Ws^{-1}(#1)}}
 \newcommand{\Wr}[2]{\ensuremath{\Ws(#1; #2)}}
 \renewcommand{\arraystretch}{0.65}
 
%-----------------------------------------------------------

% set equation numbering to include section and subsection numbers
\numberwithin{equation}{chapter}

\makeatletter
\let\OldStatex\Statex
\renewcommand{\Statex}[1][3]{%
  \setlength\@tempdima{\algorithmicindent}%
  \OldStatex\hskip\dimexpr#1\@tempdima\relax}
\makeatother
\title{What do I know about the responsibility map \Rpi F?}
\begin{document}
\maketitle
\chapter{Definitions and Basics}
	Let \( F = F^i_n \) be a \( K\times N \) (\( K\ll N \)) matrix with non-negative entries. Then define 
	\begin{equation}\label{eqn:ellDef}
		\elpi F := \frac{1}{N}\sum_{n=1}^{N} \log(\sum_{k=1}^{K} \pi_kF^k_n).
	\end{equation}
	Note that when considered as a function \( \ell_F:\R^K\rightarrow \R \), \( \elpi F \) is undefined at \( \bm\pi = \bm 0 \). This is also true for any \( \bm\pi \) where any of the sums \( \sum_k \pi_k F_n^k
	\leq 0 \). This can be partially remediated by requiring \( \sum_i F^i_n >0\;\forall n \), but the problem persists in this situation.  To fully fix these problems, \( \elpi F \) must be considered as a map \( \ell_F:U\subset\R^K\rightarrow\R \) for some carefully chosen subspace \( U\subset \R \).
	
	To better understand the function \( \elpi F \), define \( \mu_i = \log(\pi_i) \) for \( \pi_i>0 \) and \( \mu_i = -\oo \) when \( \pi_i = 0 \). This defines a change of coordinates \( \phi_i:\mu_i\mapsto \pi_i = e^{\mu_i} \) with \( e^{-\oo}:=0 \). While more work will be needed to make this precise, the correct idea is present. Thus \( \tilde{\ell}_F(\bm\mu) := \elpi[e^{\bm\mu}]{F} \) is defined for all \( \bm\mu\in\R^K \).  Note that \( \bm\phi:\R^K\rightarrow\R^K \) given by \( \bm\phi(\bm\mu) = \left(\phi_i(\mu_i)\right)_i \) has image \( \bm\phi(\R^K) = \R^K_{>0} \), the strictly positive orthant of \( \R^K \). Also, \( \bm\phi \) is a smooth diffeomorphism onto its image.
	
	Define \( A_K := \left\{\bm x\in\R^K|\<\bm x,\mathbbm 1_K\> = 1\right\}\) with \( \mathbbm 1_K  = (1,1,\ldots,1)\) the vector of all ones. Let \( S_K := A_K\bigcap \R^{K}_{\geq 0} \) be the subset of \( A_K \) consisting of non-negative coordinates. In other words, \( S_K \) is the standard probability simplex.  One benefit of defining \( A_K,\;S_K \) this way is that the barycentric coordinates for \( S_K \) as a simplex naturally extend to \( A_K \), and coincide with the standard euclidean coordinates for both of these sets as subspaces of \( \R^K \).  
	
	Since this paper partially seeks to classify behavior of  \( \elpi{F} \) on \( S_K \), discussion of the relationship between the coordinate map \( \bm\phi \)  and the compact set \( S_K \) naturally arises.  Even though \( S_K \) is compact as described above, \( \bm\phi^{-1}(S_K) \) is not compact.  Though compactification of this set is possible, most of what is needed for this discussion can be done without compactification of \( \R^K \).
	
	In general, the behavior of the set \( \bm\phi^{-1}(S_K) = \left\{\bm\mu\in\R^K|\sum_i e^{\mu_i} = 1\right\} \) is difficult to describe. One important distinction arises when \( S_K \) is considered as a manifold with boundary \( \partial S_K \).  In set notation,
	\begin{equation}\label{eqn:invImgBoundary}
	\bm\phi^{-1}(\partial S_K) = \left\{\bm\mu\in \{\R\cup \{-\oo\}\}^K | \mu_i=-\oo \text{ for at least one } i\right\}.
	\end{equation}
	Under the inverse image of \( \bm\phi \) the boundary gets mapped to a `degenerate set' in that for every point \( \bm\pi^{\ast}\in\partial S_K \), the set \( \bm\phi^{-1}(\bm\pi^{\ast}) \) consists entirely of points \( \bm\mu \) that belong in \( \{\R\cup \{-\oo\}\}^K \), i.e. some coordinates of \( \bm\mu \) lie at \( -\oo. \)
	
	To make equation \eqref{eqn:invImgBoundary} more precise, consider the following argument. Since all the points in \( \partial S_K \) are limit points of \( S_K \), if \( \bm\pi^{\ast}\in\partial S_K, \) there is a series of points \( \bm\pi^n \) such that \( \bm\pi^n\rightarrow\bm\pi^{\ast} \) as \( n\rightarrow\oo \).  Here without loss of generality, it may be assumed that \( \bm\pi^n\in\op{Int}S_K\;\forall n \).  Points \( \bm\pi^{\ast}\in\partial S_K \) are classified by \( \pi^{\ast}_{i_m} = 0 \) for some set of indices \( \{i_1,\ldots,i_M\},\;M<K \). Thus sequences \( \bm\pi^n\rightarrow\bm\pi^{\ast} \) always have \( \pi^{n}_{i_m}\rightarrow 0 \) as \( n\rightarrow\oo. \) 
	
	For any point \( \bm\pi^{\ast}\in\partial S_K \) define such sequences as \textit{boundary sequence(s)} for \( \bm\pi^{\ast} \). Boundary points may then be defined as equivalence classes of boundary sequences. Equivalence of two boundary sequences \( \bm\pi^n,\;\bm\pi^m \) in this sense is given by \( \bm\pi^n\sim\bm\pi^m \) if \( |\pi^n_i -\pi^m_i|\rightarrow 0 \) as \( n,m\rightarrow\oo \) for all \( i=1,\ldots,K. \)
	
	Under inverse image of the map \( \bm\phi \), boundary sequences in \( S_K \) translate to sequences \( \bm\mu^{n}\subset\R^K\) such that \( \mu^n_{i_m} \rightarrow -\oo\) as \( n\rightarrow\oo \) for some set of indices \( i_m,\;m=1,\ldots,M<K \). Thus the preimage \( \bm\phi^{-1}(\bm\pi^{\ast}) \) for \( \bm\pi^{\ast}\in \partial S_K \) is more precisely defined as the equivalence class of preimage of boundary sequences for \( \bm\pi^{\ast}\! \), where two sequences \( \bm\mu^n\!,\;\bm\mu^m \) are equivalent if \( \bm\pi^n:=\bm\phi(\bm\mu^n),\;\bm\pi^m:=\bm\phi(\bm\mu^m)\) are equivalent boundary sequences in \( S_K \).
	
	Another required definition is that of \textit{face} and \textit{degeneracy} maps on simplicies. Broadly speaking, degeneracy maps act like projections of facets of \( S_K \) onto lower dimensional simplices \( S_{M},\;M<K \). On the other hand, face maps embed lower dimensional simplicies \( S_{M},\;M<K \) into facets of higher dimensional simplicies \( S_K \). For a general overview, the interested reader may refer to nLab \Ryan{citation, \url{https://ncatlab.org/nlab/show/simplex\#BarycentricCoordinates}}. 
	
	For the purposes of this paper, define the face map as follows.  Given \(\bm\pi = (\pi_1, \ldots, \pi_{K-1}) \in S_{K-1}\)  and \( 1\leq i\leq K \), the $i$th face inclusion is the subspace inclusion
	\begin{equation}\label{eqn:faceMapDefn}
	\gd_i:S_{K-1}\hookrightarrow S_K:(\pi_1, \ldots, \pi_{K-1})\mapsto (\pi_1, \ldots,\pi_{i-1},0,\pi_i,\ldots, \pi_{K-1}).
	\end{equation}
	The map \( \gd_i \) is induced by a similar inclusion map \( \R^{K-1}\hookrightarrow\R^K \), which is exactly the same on coordinates.  
	
	For a set of indices \( I=\{i_1<i_2<\ldots<i_M\}\) with \(M<K\) the definition of \( \gd_i \) may be extended to a collection of inclusions \( \Delta_I:S_{K-M} \hookrightarrow S_K \) given by \( \Delta_I = \gd_{i_M}^{K}\circ\gd_{i_{M-1}}^{K-1}\circ\ldots\circ\gd_1^{K-M+1} \), where \( \gd_i^{k} \) represents the $i$th facemap \( \gd_i^k:S_{k-1}\hookrightarrow S_k \).  It is important to note here that \( \Im{\gd_i}\subsetneq\partial S_K \) and that this also holds for the maps \( \Delta_{i_1<i_2< \ldots <i_M } \). In fact, it is the case that \(\partial S_K = \bigcup_{i=1}^{K}\Im{\gd_i}\) and if \( i<j \), \( \Im{\gd_i}\cap\Im{\gd_j} = \Im{\Delta_{i<j}} \).
		
	With the face maps thus defined, consider the degeneracy map.\( \Psi \)
\[  \left[ \mu=2 , \pi_{1}={{\left(2\,f_{2,1}-f_{1,1}\right)\,f
		_{2,2}-f_{1,2}\,f_{2,1}}\over{\left(2\,f_{2,1}-2\,f_{1,1}\right)\,f
		_{2,2}-2\,f_{1,2}\,f_{2,1}+2\,f_{1,1}\,f_{1,2}}} , \pi_{2}=-{{f_{1,1
		}\,f_{2,2}+f_{1,2}\,f_{2,1}-2\,f_{1,1}\,f_{1,2}}\over{\left(2\,f_{2,
			1}-2\,f_{1,1}\right)\,f_{2,2}-2\,f_{1,2}\,f_{2,1}+2\,f_{1,1}\,f_{1,2
}}} \right]  \]

\[ \left[ \mu=2 , \pi_{1}={{\left(2\,f_{2}(x_{1})-f_{1}(x_{1})
		\right)\,f_{2}(x_{2})-f_{2}(x_{1})\,f_{1}(x_{2})}\over{\left(2\,f_{2
		}(x_{1})-2\,f_{1}(x_{1})\right)\,f_{2}(x_{2})+\left(2\,f_{1}(x_{1})-
		2\,f_{2}(x_{1})\right)\,f_{1}(x_{2})}} , \pi_{2}=-{{f_{1}(x_{1})\,f
		_{2}(x_{2})+\left(f_{2}(x_{1})-2\,f_{1}(x_{1})\right)\,f_{1}(x_{2})
	}\over{\left(2\,f_{2}(x_{1})-2\,f_{1}(x_{1})\right)\,f_{2}(x_{2})+
		\left(2\,f_{1}(x_{1})-2\,f_{2}(x_{1})\right)\,f_{1}(x_{2})}}
\right]  \]
\end{document}