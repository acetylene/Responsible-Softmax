%!TEX root = ../Dissertation_RC.tex

\subsection{$K$-means algorithm}\label{kmeans}
The $K$-means algorithm has been in use for several decades.  Though the first 
mention of the algorithm by name was given by MacQueen in 1967 \cite{
macqueen1967kmeans} the idea had been around for some time(insert steinhaus ref).  The standard  
algorithm was used at Bell Labs in 1957 \cite{Lloyd82} for pulse code 
modulation.  Pollard \cite{pollard1981,pollard1982} showed that the $K$-means 
algorithm is consistent in a very precise sense.  Today the algorithm is used 
in many applications \cite{AutoClass1,AutoClass2}, and can be found in many 
good books on machine learning.
\cite{Bishop1995,MacKay2002,BishopBook,hastie09esl,MML_2019}

The outline below is primarily compiled from chapters 20 and 22 of MacKay
\cite{MacKay2002}, though the Bishop and Deisenroth books 
\cite{BishopBook,MML_2019} are also a good reference.  

The \(K\)-means algorithm is used for vector quantization and for data 
clustering.  It is so named because it separates data points into $K$ distinct 
groups, each characterized by a `mean' \(\bm m_k,\, k= 1,\ldots,K\).  In the 
case that the \(K\)-means algorithm is being used for clustering, these means 
are the cluster centers and each data point is assigned to the closest mean.
In this situation it is the case that 
\[\bm m_k = \frac{\sum_{n=1}^{N} r_k^n \bm x^{(n)}}{N_k}\]
where 
\begin{equation*}
r^n_k = \begin{cases}
			1 & \text{if } \bm x^{(n)} \text{ is closest to } \bm m_k\\
			0 & \text{otherwise}
		\end{cases}
\end{equation*}
and \(N_k = \sum_{n=1}^{N} r^n_k\).  In other words, the means \(\bm m_k\) are 
literally the means of the assignment clusters.

Of course we cannot understand `closest' without first defining a distance 
function  on the underlying data space.  For the original \(K\)-means 
algorithm, the distance was the manhattan distance, though we will use a 
scaled square of the euclidean distance:
\[d(\bm x, \bm y) = \frac 12 \sum_i (x_i-y_i)^2.\]
It is worth noting that the choice of distance in this sense is somewhat 
arbitrary. In most descriptions of the algorithm, euclidean distance is used 
to aid visualization.

To implement the \(K\)-means algorithm, start with \(K\) distinct means.  
A common practice is to use randomly sampled data points 
\(\{\bm m_1 = \bm x^{(n_1)},\ldots,\bm m_K = \bm x^{(n_K)}\}\). Then iteratively 
do the following:
\begin{enumerate}
	\item For each data point, \(\bm x^{(n)}\), set 
	\(\hat{k}_n = \argmin_k d(\bm x^{(n)},\bm m_k)\).
	\item Set \(r_{\hat{k}_n}^{n} = 1\), for each \(n\). Set all other 
	\(r_k^n = 0\)
	\item Calculate \(N_k = \sum_{n=1}^{N} r^n_k\) and 
	\[\bm m_k^{new} = \frac{\sum_{n=1}^{N} r_k^n \bm x^{(n)}}{N_k}.\]
	If \(N_k = 0\), \(\bm m_k^{new} = \bm m_k\).
	\item If \(d(\bm m_k ,\bm m_k^{new})\) is within a predefined tolerance for every \( k \), 
	stop.  Otherwise set \(\bm m_k = \bm m_k^{new}\) and repeat at step 1. Equivalently, convergence happens when no assignments \( \hat{k}_n \) change. 
\end{enumerate}
%algorithm? soft k means? generalized k means? ADD algorithm block?
In the literature, it is common to see steps one and two listed as the 
assignment step, and steps three and four as the update step.  

The easiest way to see that this algorithm terminates is to recognize that the 
function \[L := \sum_{n=1}^{N} d(\bm x^{(n)},\bm m_{\hat{k}_n})\]
either stays the same or decreases at each update step.  In this sense, \(L\) 
acts as a Lyapunov function for the \(K\)-means algorithm.

One problem with k-means clustering that is particularly relevant to this 
dissertation comes when the clusters do not have equal representation in the data.
As an example, \cite{xiong2009kmeansvalid} give examples of data sets where kmeans produces equal sized clusters, even though their true sizes relative to the number of data points is different. %\Ryan{Still working on examples, have a few good references though.}

%In this case, the cluster means are often slightly off center and some data points are be inappropriately labeled. %voronoi diagrams? 
One way to address the problem is with \textit{soft responsibility} 
\begin{equation}\label{softResp}
r_k^{(n)} = \frac{\exp(-\gb d(\bm x^{(n)},\bm m_k))}
{\sum_i \exp(-\gb d(\bm x^{(n)},\bm m_i))}.
\end{equation}
The idea behind soft responsibility is that each cluster center is partially 
responsible for each data point.  The amount of responsibility \(r_k^{(n)}\) 
of \(\bm m_k\) for the data point \(\bm x^{(n)}\) ought to be inversly 
proportional to \(d(\bm x^{(n)},\bm m_k)\). That is, cluster centers closer to 
data points have greater responsibility for those data points.  

The factor \(\gb\) included here is an inverse temperature, or stiffness 
parameter, and it can be set at the beginning or iteratively.  In futher 
refinements of the soft $K$-means algorithm, each cluster center has its own
\(\gb_k\), and at each iteration \(\gb_k = \dfrac{1}{\gs_k^2}\) where 
\(\gs_k^{2}\) is the weighted sample variance of the data points assigned to 
cluster $k$.
\begin{equation}\label{isoSigma}
\gs_k^{2} = \frac{\sum_n r_k^{(n)}d(\bm x^{(n)},\bm m_k)^2}{N_k}.
\end{equation}
Further refinements to this algorithm can be found in MacKay's book, and were 
also developed in the software \textit{AutoClass}. \cite{MacKay2002,AutoClass1,AutoClass2}
