%!TEX root = ../Dissertation_RC.tex

\subsection{Expectation Maximization} \label{emAlg}
Upon close inspection, it can be seen that the soft \(K\) means algorithm is 
very similar to the Expectation Maximization (or EM) algorithm for Gaussian 
Mixture Models.  We give a brief overview of expectation maximization
and the relation of this algorithm to responsibility as discussed in section 
\ref{kmeans}. The discussion below roughly follows discussions available in 
Bishop and other sources \cite{MML_2019, BishopBook, hastie09esl}.

EM was first described for the general case in a paper by Dempster et. al. \cite{Dempster77EM}.  
Though the paper mentions that Hartley \cite{hartley1958}, Baum \cite{baum1970} and others \cite{woodbury1970missing,sundberg1974,sundberg1976}
had already used similar techniques in special circumstances.
%add reference to pearsonand mixture models?

The basic idea behind EM is to add hidden or latent variables to a modeling
problem in such a way that maximum likelihood estimation is made easier.  The 
heuristic of this approach is that the latent variables are simply unobserved 
features of the data.  

To be more precise, suppose we are given data $ x $ and we want to fit a model 
with parameters $ \gt $ for the pdf \( p(x|\gt) \) using maximum likelihood 
estimation. In many cases this is an intractable problem that can be simplified
by considering the conditional pdf
\begin{equation}\label{emcond}	
p(x|\gt,z).
\end{equation}

Now as \( z \) are latent variables, we must place a prior \( p(z) \) on the 
distribution of \( z. \) Using \ref{emcond}, and the law of total probability 
we may write
\begin{equation}\label{emtotprob}
p(x|\gt)=\int_{\mathcal{Z}} p(x|\gt,z)p(z)\;dz.
\end{equation}
Where the integral is taken over the space of possible latent variables.

In practice, the integral in \ref{emtotprob} can easily become intractable.  The trick is
to choose \( z \) and \( p(z) \) in a manner that avoids this difficulty. 
The EM algorithm is an iterative procedure that relies on repeating two steps until convergence.  It is the use of these two steps which help us choose \( z \) and \( p(z) \). These steps are also from which the algorithm receives its name.  

While EM is more broadly defined and used than what we will discuss, we will consider the case where we are using EM to fit a Gaussian mixture model for simplicity.  In this case, let 
\begin{align}
	\mathcal{D} &= \{\bm x^{(1)},\bm x^{(2)},\ldots,\bm x^{(N)}\} \nonumber \\
	f_k(\bm x) &\sim \mathcal{N}(\bm \mu_k,\bm \Sigma_k)\;\; k=1,\ldots,K\\
	 p(z=k) &=\pi_{k},   \;\; \sum_{k} \pi_{k} = 1	
\end{align}

Then to use the EM algorithm, set parameters to some initial values: \( \pi_{k}=\pi_{k}^0,\; \mu_{k}=\mu_{k}^0,\; \Sigma_{k}=\Sigma_{k}^0\). Then apply the following steps:
\begin{enumerate}
	\item \textit{Expectation} step: Set 
		\begin{equation}\label{emResp}
		 r_k^n = \frac{\pi_k f_k(\bm x^{(n)})}{\sum_{j=1}^K \pi_j f_j(\bm x^{(n)})}
		\end{equation}
	\item \textit{Maximization} step: Set
	\begin{align}
		N_k &= \sum_{n=1}^{N}r_k^{n}\\
		\pi_k^{new} &= \dfrac{N_k}{N}\\
		\bm \mu_{k}^{new} &= \dfrac{1}{N_k}\sum_{n=1}^{N} r_k^{n}\bm x^{(n)}\\
		\bm \Sigma_k^{new} &= \dfrac{1}{N_k}\sum_{n=1}^{N}r_k^{n} (\bm x^{(n)}-\bm \mu_{k}^{new})(\bm x^{(n)}-\bm \mu_{k}^{new})^{\intercal}
	\end{align}
	\item If all of \( \pi_k, \;\bm\mu_k,\;\bm\Sigma_k \) are close enough to their new counterparts, stop.  Otherwise, set \( \pi_k = \pi_k^{new}, \;\bm\mu_k = \bm\mu_{k}^{new}, \;\bm\Sigma_k = \bm\Sigma_k^{new}\), then repeat steps 1 and 2.
\end{enumerate}

We note that this is a specific implementation of the EM algorithm in the case we suspect that a Gaussian mixture model works well for the underlying data.  Other changes that may be made are choosing different parameterized distributions \( f_k(\bm x) \), or even adapting the algorithm for completely different models. For example the Baum-Welch algorithm \cite{baum1970}, is tailored  specifically to using EM on Hidden Markov Models. More information about generalizations of EM may be found in Hastie et. al. \citep[p.276]{hastie09esl}.

Other than ease of explanation, we use the preceding description of EM to illustrate the similarity of EM and \( K \)-means algorithms.  For example, fix \( \bm\Sigma_{k} \) to be isotropic with variance as described in \ref{isoSigma}, then equations \ref{softResp} and \ref{emResp} agree on the definition of responsibility.  %This connection can be explored somewhere else, for now we are just interested in responsibility! \Ryan{Should I mention ELBO bound here?}%nope!